\chapter{Конструкторский раздел}\label{ch:ch2}
\section{Обоснование выбора языка программирования и среды разработки}\label{sec:ch2/sec1}

Для разработки {\ProgModule} понадобится сверхвысокоуровневый язык с кросс-платформенной
стандартной библиотекой, который позволит точно и лаконично описать этапы анализа,
а так же имеющий высокую скорость исполнения, для анализа больших объемов исходного кода и
исполняемых файлов.

{\small
    \setlength{\tabcolsep}{2pt}
    \begin{longtable}{*{5}{| c}|}
        \hline
        \diagbox[width=8cm]{Свойства}{Язык программирования} &
            \makecell{Nim \autocite{nim}} &
            \makecell{Python \autocite{python}} &
            \makecell{Perl \autocite{perl}} &
            \makecell{C/C++} \\
        \hline
            \makecell{Сверхвысокоуровневость} & 
            \greencell{Да} & 
            \greencell{Да} &
            \greencell{Да} &
            \redcell{Нет} \\
        \hline
            \makecell{Компилируется в\\машинный код} & 
            \greencell{Да} & 
            \redcell{Нет} &
            \redcell{Нет} &
            \greencell{Да} \\
        \hline
            \makecell{Количество функции в\\стандартной библиотеке} & 
            5585 & 
            638 &
            1338 &
            1224 \\
        \hline
            \makecell{Портируемость} & 
            \greencell{Есть} & 
            \greencell{Есть} &
            \greencell{Есть} &
            \yellowcell{\makecell{Есть,\\но неудобная}}\\
        \hline
            \makecell{Встроенная\\генерация документации} & 
            \greencell{Есть} & 
            \greencell{Есть} &
            \greencell{Есть} &
            \redcell{Нет}\\
        \hline
            \makecell{Статическая типизация} & 
            \greencell{Есть} & 
            \redcell{Нет} &
            \redcell{Нет} &
            \greencell{Есть}\\
        \hline
            \makecell{Автоматическое\\управление памятью} & 
            \greencell{Есть} & 
            \greencell{Есть} &
            \greencell{Есть} &
            \greencell{Есть} \\
        \hline
            \makecell{Обобщенное программирование} & 
            \greencell{Есть} & 
            \greencell{Есть} &
            \greencell{Есть} &
            \greencell{Есть} \\
        \hline
            \makecell{Опыт использования} & 
            \greencell{Есть} & 
            \greencell{Есть} &
            \redcell{Нет} &
            \greencell{Есть} \\
        \hline
    \caption{\label{table:languages-comparsion}
           Сравнительная таблица языков программирования}
    \end{longtable}
}

Рассмотрим подробно каждый из представленных в таблице языков:
\begin{itemize}
    \item C++ -- мультипарадигменный высокоуровневый язык программирования,
        разработанный в 1983 году Бьёрном Страуструпом. Является практически
        полным надмножеством языка C. Статически типизирован.\\
        Отличается высокой производительностью и неплохой гибкостью при написании кода.
        К минусам языка можно отнести сложность освоения и перегруженность 
        <<наследием>> 80-х годов прошлого века, а так же низкую скорость компиляции,
        по сравнению с предшественником -- C.\\
        Портируемость языка на различные платформы обеспечивается пере- или
        кросс-компиляцией исходного кода под нужную платформу.
        

    \item Python \autocite{python} -- мультипарадигменный сверхвысокоуровневый 
        язык программирования, разработанный в 1991 году Гвидо Ван Россумом.
        Является интерпретируемым языком, имеет слабую динамическую типизацию,
        что позволяет легко писать обобщенный код и использовать мета-программирование,
        но так же ведет к трудноулавливаемым ошибкам. Негативное влияние можно сгладить
        с помощью указания типов при объявлении перемнных и аргументов функций, а так же 
        программы, проверяющей эти типы -- линтера. Например pylint \autocite{pylint} или
        pyflakes \autocite{pyflakes}.\\
        Благодаря своей популярности, python так же портирован на большое количество платформ.
        Большим плюсом языка является его обширная стандартная библиотека, позволяющая легко
        писать комплексные приложения, не прибегая к установке дополнительных библиотек --
        такие программы, как и сам python, следуют философии <<в комплекте с батарейками>>
        (<<batteries included>> \autocite{batteries-included}), суть которой заключается в 
        самодостаточности программ. Помимо этого вместе с python поставляется менеджер
        пакетов pip \autocite{pip}, позволяющий удобно устанавливать требуемые библиотечные модули вместе
        с зависимостями.\\
        К минусам языка можно отнести медлительность эталонного интерпретатора языка -- cpython \autocite{cpython}.
        Код, исполняемый им, в определенных задачах медленнее кода на C в сотни раз. Не смотря на то, что
        есть более быстрые интерпретаторы: PyPy \autocite{pypy}, Jython \autocite{jython}, Iron Python \autocite{iron-python},
        они не смогут достичь скорости исполнения программ, компилируемых в машинный код.\\
        На данный момент существует две, между собой несовместимые, версии языка: 
        python 2, поддержка которого закончилась \DTMdate{2020-01-01} и python 3.

    \item Perl \autocite{perl} -- мультипарадигменный сверхвысокоуровневый 
        язык программирования, разработанный в 1987 году Ларри Уоллом.
        Является интерпретируемым языком, имеет слабую динамическую типизацию.\\
        Полное название языка -- <<Practical Extraction and Report Language>> 
        (<<Практический Язык для Извлечения Данных и Составления Отчётов>>), отражает его суть:
        в языке реализованы обширные возможности для работы с текстом, в синтаксис интегрированы 
        регулярные выражения, как и в языках, которые оказали на него наибольшее влияние --
        AWK \autocite{awk} и sed \autocite{sed}. Но это же и я является его слабой частью, так как
        Perl скорее предназначен для однострочных команд в терминале, как AWK \autocite{awk} и
        sed \autocite{sed}.
        
    \item Nim \autocite{nim} -- мультипарадигменный сверхвысокоуровневый 
        язык программирования, разработанный в 2004 году Андреасом Румпфом.
        Является компилируемым языком, имеет строгую статическую типизацию.\\
        Заметно, что на синтаксис языка повлиял Python, что сделало его
        выразительным и понятным. Язык использует промежуточную компиляцию, которая несколько
        замедляет процесс компиляции программ, но позволяет запускать nim-программы на различных
        платформах. На данный момент поддерживается компиляция в JavaScript \autocite{javascript}
        и оптимизированный C-код с несколькими моделями управления памятью:\\
        \begin{itemize}
            \item Сборщики мусора, основанные на:
                \begin{enumerate}
                    \item Подсчете ссылок
                    \item Подсчет ссылок с оптимизацией move-семантикой \autocite{nim-gc-move}
                    \item Boehm \autocite{boehm-gc}
                    \item Go \autocite{go-gc}
                \end{enumerate}
            \item Ручном освобождении памяти
            \item Модель, в которой вся выделенная память высвобождается только по завершению программы
                (не рекомендуется к использованию)
        \end{itemize}
        Компиляции Nim в C, означает не только высокую скорость работы, но и прозрачный программный интерфейс при взаимодействии с
        C библиотеками. Это значит, что можно писать Nim-код, взаимодействующий с С библиотекой так же, как
        если бы это была Nim-библиотека, в отличие от, например, Python.\\
        Так же вместе с компилятором языка поставляется пакетный менеджер nimble \autocite{nimble} и генератор
        документации из комментариев, написанных на reStructuredText \autocite{restructuredtext}.

\end{itemize}

\textbf{Вывод:} Из всего вышесказанного следует, что для {\ProgModule} лучше всего подойдет язык Nim
благодаря его скорости, выразительности и портируемости на различные платформы.
Кроме того, для подготовки динамического анализа программы будут использованы утилиты, умеющие разбирать
заголовки исполняемого файла, а именно objdump и readelf. Форматирование входных данных для данных утилит
будет осуществляться с помощью Bash-скриптов.

\section{Обоснование выбора среды разработки}\label{sec:ch2/sec2}

Для разработки на Nim существует несколько IDE и огромное количество
текстовых редакторов, часть которых рассмотрим ниже:

{\small
    \setlength{\tabcolsep}{2pt}
    \begin{longtable}{*{6}{| c}|}
        \hline
        \diagbox[width=8cm]{Свойства}{IDE/Редактор} &
            \makecell{Aporia \autocite{aporia-ide}} &
            \makecell{Atom \autocite{atom-ide}} &
            \makecell{Sublime\\Text \autocite{sublime-ide}} &
            \makecell{Visual\\Studio\\Code \autocite{vs-code-ide}} &
            \makecell{Vim \autocite{vim-ide}} \\
        \hline
            \makecell{Поддержка плагинов} & 
            \redcell{Нет} &
            \greencell{Да} & 
            \greencell{Да} &
            \greencell{Да} &
            \greencell{Да} \\
        \hline
            \makecell{Требователен к ресурсам} & 
            \greencell{Нет} & 
            \redcell{Да} & 
            \greencell{Нет} & 
            \redcell{Да} & 
            \greencell{Нет} \\ 
        \hline
            \makecell{Имеет продвинутую систему\\редактирования текста} & 
            \redcell{Нет} &
            \redcell{Нет} &
            \redcell{Нет} &
            \redcell{Нет} &
            \greencell{Да} \\
        \hline
            \makecell{Кросс-платформенность} & 
            \greencell{Есть} & 
            \greencell{Есть} &
            \greencell{Есть} &
            \greencell{Есть} &
            \greencell{Есть} \\
        \hline
            \makecell{Может работать\\без GUI} & 
            \redcell{Нет} &
            \redcell{Нет} &
            \redcell{Нет} &
            \redcell{Нет} &
            \greencell{Да} \\
        \hline
            \makecell{Восстановление после сбоев} & 
            \redcell{Нет} & 
            \greencell{Есть} &
            \greencell{Есть} &
            \greencell{Есть} &
            \greencell{Есть} \\
        \hline
            \makecell{Возможность выделять\\ключевые слова с помощью\\регулярных выражений} & 
            \redcell{Нет} & 
            \greencell{Есть} &
            \greencell{Есть} &
            \greencell{Есть} &
            \greencell{Есть} \\
        \hline
            \makecell{Опыт использования} & 
            \redcell{Нет} &
            \redcell{Нет} &
            \greencell{Есть} &
            \greencell{Есть} &
            \greencell{Есть} \\
        \hline
    \caption{\label{table:ide-comparsion}
           Сравнительная таблица IDE и редакторов кода}
    \end{longtable}
}

Рассмотрим подробно каждый из представленных в таблице редакторов:
\begin{itemize}
    \item Aporia -- простая IDE, написанная на nim, с использованием GTK2.
        В настоящее время не поддерживается, так как большая часть Nim-программистов
        перешла на Visual Studio Code.

    \item Atom -- редактор от GitHub Inc., написан с использованием Electron \autocite{electron} -- фреймворка
        для разработки кросс-платформенных приложений с помощью HTML, JavaScript и CSS. Из-за архитектурных
        и технологических решений, все программы, написанные на данном фреймворке, будут очень требовательны
        к ресурсам.

    \item Sublime Text -- проприетарный текстовый редактор, возможности которого могут быть расширены
        с помощью плагинов на python.

    \item Visual Studio Code -- редактор от Microsoft. Так же, как и Atom, написан с использованием Electron.

    \item Vim -- текстовый редактор с открытым исходным кодом и большими возможностями к
        быстрому редактированию текстов. Является наследником редактора vi, который, в свою
        очередь, создавался с оглядкой на редактор ed. Управление делится на
        режим ввода и режим команд, благодаря чему есть возможность управлять 
        редактором только с помощью клавиатуры, что, при должном умении, повышает скорость
        не только из-за отсутствия необходимости в использовании компьютерной мыши, но и
        более коротким сочетаниям  <<горячих клавиш>>. Легко поддается модифицированию с помощью плагинов.
        Есть под множество платформ.

\end{itemize}

\textbf{Вывод:} Из всего вышесказанного и личного опыта следует, 
что для разработки {\ProgModule} лучше всего подойдет текстовый редактор Vim,
так как он поддерживает добавление плагинов, не требователен к ресурсам и позволяет
очень быстро редактировать текст.

\newpage

\section{Организация передачи информации между компонентами {\ProgModule}}\label{sec:ch2/sec3}
Передача информации между компонентами {\ProgModule} осуществляется посредством
сериализации внутренних структур конкретного модуля в формате JSON.
JSON удобен тем, что является простым для чтения как человеком, так и компьютером,
что позволяет оператору анализировать так же и промежуточные результаты работы, для
вынесения вердикта.

\section{Cхема данных}\label{sec:ch2/sec4}

\begin{figure}[!htbp]
    \centerfloat{
        \tikzset{
    line/.style={draw, -latex'},
%     every join/.style={line},
    u/.style={anchor=south},
    r/.style={anchor=west},
    fxd/.style={text width = 6em},
    it/.style={font={\itshape}},
    bf/.style={font={\bfseries}}

}
\tikzstyle{base} =
    [
        draw,
%         on chain,
%         on grid, именно из-за этой опции у вас node distance было расстоянием между центрами, а не между блоками
%         align=center,
%         minimum width = 5ex,
%         node distance = 6mm and 60mm,
        text badly centered,
        text width=12em,
        minimum height=3ex,
        inner xsep = 1pt,
        inner ysep = 3pt,
    ]
\tikzstyle{coord} =
    [
        coordinate,
%         on chain,
%         on grid
    ]
\tikzstyle{cloud} =
    [
        base,
        ellipse,
%         node distance = 3cm,
%         minimum height = 2em
    ]
\tikzstyle{decision} =
    [
        base,
        diamond,
        aspect=2,
%         node distance = 2cm,
        inner sep = 0pt
    ]
\tikzstyle{block} =
    [
        rectangle,
        base,
        rounded corners,
%         minimum height = 2em
    ]
\tikzstyle{print_block} =
    [
        base,
        tape,
        tape bend top=none,
    ]
\tikzstyle{io} =
    [
        base,
        trapezium,
        trapezium left angle = 70,
        trapezium right angle = 110,
    ]
\tikzstyle{prompt} =
    [
        base,
        trapezium,
        trapezium left angle = 90,
        trapezium right angle = 80,
        shape border rotate = 90
    ]
\tikzstyle{disk file} =
    [
        base,
        cylinder,
        aspect=0.2,
        minimum width=4ex, % то, что у~лежачего цилиндра по вертикали — это ширина
    ]
\tikzstyle{process} =
    [
        rectangle,
        base,
    ]
\makeatletter
\pgfkeys{/pgf/.cd,
    subrtshape w/.initial=2mm,
    cycleshape w/.initial=2mm
}
\pgfdeclareshape{subrtshape}{
    \inheritsavedanchors[from=rectangle]
    \inheritanchorborder[from=rectangle]
    \inheritanchor[from=rectangle]{north}
    \inheritanchor[from=rectangle]{center}
    \inheritanchor[from=rectangle]{west}
    \inheritanchor[from=rectangle]{east}
    \inheritanchor[from=rectangle]{mid}
    \inheritanchor[from=rectangle]{base}
    \inheritanchor[from=rectangle]{south}
    \backgroundpath{
        \southwest \pgf@xa=\pgf@x \pgf@ya=\pgf@y
        \northeast \pgf@xb=\pgf@x \pgf@yb=\pgf@y
        \pgfmathsetlength\pgfutil@tempdima{\pgfkeysvalueof{/pgf/subrtshape w}}
        \def\ppd@offset{\pgfpoint{\pgfutil@tempdima}{0ex}}
        \def\ppd@offsetm{\pgfpoint{-\pgfutil@tempdima}{0ex}}
        \pgfpathmoveto{\pgfqpoint{\pgf@xa}{\pgf@ya}}
        \pgfpathlineto{\pgfqpoint{\pgf@xb}{\pgf@ya}}
        \pgfpathlineto{\pgfqpoint{\pgf@xb}{\pgf@yb}}
        \pgfpathlineto{\pgfqpoint{\pgf@xa}{\pgf@yb}}
        \pgfpathclose
        \pgfpathmoveto{\pgfpointadd{\pgfpoint{\pgf@xa}{\pgf@yb}}{\ppd@offsetm}}
        \pgfpathlineto{\pgfpointadd{\pgfpoint{\pgf@xa}{\pgf@ya}}{\ppd@offsetm}}
        \pgfpathlineto{\pgfpointadd{\pgfpoint{\pgf@xb}{\pgf@ya}}{\ppd@offset}}
        \pgfpathlineto{\pgfpointadd{\pgfpoint{\pgf@xb}{\pgf@yb}}{\ppd@offset}}
        \pgfpathclose
    }
}
\pgfdeclareshape{cyclebegshape}{
    \inheritsavedanchors[from=rectangle]
    \inheritanchorborder[from=rectangle]
    \inheritanchor[from=rectangle]{north}
    \inheritanchor[from=rectangle]{center}
    \inheritanchor[from=rectangle]{west}
    \inheritanchor[from=rectangle]{east}
    \inheritanchor[from=rectangle]{mid}
    \inheritanchor[from=rectangle]{base}
    \inheritanchor[from=rectangle]{south}
    \backgroundpath{
        \southwest \pgf@xa=\pgf@x \pgf@ya=\pgf@y
        \northeast \pgf@xb=\pgf@x \pgf@yb=\pgf@y
        \pgfmathsetlength\pgfutil@tempdima{\pgfkeysvalueof{/pgf/cycleshape w}}
        \pgfpathmoveto{\pgfqpoint{\pgf@xa}{\pgf@ya}}
\pgfpathlineto{\pgfpointadd{\pgfpoint{\pgf@xa}{\pgf@yb}}{\pgfpoint{0ex}{-\pgfutil@tempdima}}}
\pgfpathlineto{\pgfpointadd{\pgfpoint{\pgf@xa}{\pgf@yb}}{\pgfpoint{\pgfutil@tempdima}{0ex}}}
\pgfpathlineto{\pgfpointadd{\pgfpoint{\pgf@xb}{\pgf@yb}}{\pgfpoint{-\pgfutil@tempdima}{0ex}}}
\pgfpathlineto{\pgfpointadd{\pgfpoint{\pgf@xb}{\pgf@yb}}{\pgfpoint{0ex}{-\pgfutil@tempdima}}}
\pgfpathlineto{\pgfqpoint{\pgf@xb}{\pgf@ya}}
        \pgfpathclose
    }
}
\pgfdeclareshape{cycleendshape}{
    \inheritsavedanchors[from=rectangle]
    \inheritanchorborder[from=rectangle]
    \inheritanchor[from=rectangle]{north}
    \inheritanchor[from=rectangle]{center}
    \inheritanchor[from=rectangle]{west}
    \inheritanchor[from=rectangle]{east}
    \inheritanchor[from=rectangle]{mid}
    \inheritanchor[from=rectangle]{base}
    \inheritanchor[from=rectangle]{south}
    \backgroundpath{
        \southwest \pgf@xa=\pgf@x \pgf@ya=\pgf@y
        \northeast \pgf@xb=\pgf@x \pgf@yb=\pgf@y
        \pgfmathsetlength\pgfutil@tempdima{\pgfkeysvalueof{/pgf/cycleshape w}}
        \pgfpathmoveto{\pgfqpoint{\pgf@xb}{\pgf@yb}}
\pgfpathlineto{\pgfpointadd{\pgfpoint{\pgf@xb}{\pgf@ya}}{\pgfpoint{0ex}{\pgfutil@tempdima}}}
\pgfpathlineto{\pgfpointadd{\pgfpoint{\pgf@xb}{\pgf@ya}}{\pgfpoint{-\pgfutil@tempdima}{0ex}}}
\pgfpathlineto{\pgfpointadd{\pgfpoint{\pgf@xa}{\pgf@ya}}{\pgfpoint{\pgfutil@tempdima}{0ex}}}
\pgfpathlineto{\pgfpointadd{\pgfpoint{\pgf@xa}{\pgf@ya}}{\pgfpoint{0ex}{\pgfutil@tempdima}}}
\pgfpathlineto{\pgfqpoint{\pgf@xa}{\pgf@yb}}
        \pgfpathclose
    }
}
\makeatother
\tikzstyle{subroutine} =
    [
        base,
        subrtshape,
    ]
\tikzstyle{cyclebegin} =
    [
        base,
        cyclebegshape,
    ]
\tikzstyle{cycleend} =
    [
        base,
        cycleendshape,
    ]
\tikzstyle{connector} =
    [
        base,
        circle,
    ]

% \small
% \footnotesize
\scriptsize
\renewcommand{\baselinestretch}{0.8}
\sf

\noindent
\resizebox{\linewidth}{!}{
% !TEX encoding = UTF-8 Unicode
% Úτƒ-8 encoded
% http://www.linux.org.ru/forum/general/10357036
% \begin{figure}
% \hspace{-4cm}
% \small
\begin{tikzpicture}[%
    start chain=main_vert going below,    % General flow is top-to-bottom
    start chain=main_horz going right,  
    start chain=rev_vert going above,    
    node distance=1.ex and 1em, % Global setup of box spacing
%     scale=0.7, 
%     every node/.style={scale=0.72}
every on chain/.style=join,
    ] 

        \tikzstyle{fitblock}=[inner sep = 0ex]
        \tikzstyle{shortline}=[draw, thin]
        \tikzstyle{longline}=[shortline,-latex']
        \tikzstyle{revline}=[shortline,latex'-]
        \tikzstyle{nodraw}=[draw=none]

        \tikzset{every join/.style=shortline}
        \node [disk file ] (sources)         [on chain=main_vert                   ] {  Файлы с~исходным кодом};
        \node [disk file ] (makefile)        [right  = of sources ] {  Makefile};
%         \node [prompt    ] (makefile path)   [right  = of makefile                             ] {  Путь до папки с~Makefile};
        \coordinate                          [on chain=main_vert] (main_from_makefile);
%         \coordinate[on chain=main_vert] (main_from_makefile path);
%         \coordinate                          [on chain=main_vert] (no_used);
        \tikzset{every join/.style=shortline} %longline
        \node [process   ] (builder)         [on chain=main_vert              ] {  Сборка};
        \tikzset{every join/.style=shortline}
                
        \node [disk file ] (build log)       [on chain=main_vert        ] {  Файл с~информацией о~сборке};
        \node [process   ] (static analyzer) [on chain=main_vert        ] {  Модуль статического анализа};
        \node [disk file ] (stat result)     [on chain=main_vert        ] {  Результаты статического анализа};
        \node [process   ] (stat parser)     [on chain=main_vert        ] {  Модуль преобразования результатов статического анализа};
        \node [disk file ] (stat json)       [on chain=main_vert                ] {  Преобразованные результаты статического анализа};
        \node [process   ] (aggregator)      [on chain=main_vert        ] {  Модуль агрегирования результатов линковки и~статического анализа};
        \node [disk file ] (aggregator file) [on chain=main_vert           ] {  Агрегированные результаты линковки и~статического анализа};
        \node [process   ] (comparer)        [on chain=main_vert           ] {  Модуль сравнительного анализа};
        \node [disk file ] (summary)         [on chain=main_vert           ] {  Результаты сравнительного анализа};
        \tikzset{every join/.style=nodraw}

        
        \draw [longline] (makefile)   |- (main_from_makefile);
%         \draw [longline] (makefile path)         |- (main_from_makefile path);

%         \draw [longline] (sources)         |- (static analyzer);
         \coordinate                          [left = of static analyzer] (static_from_sources);
        \draw [longline] (sources)     -| (static_from_sources)    -- (static analyzer);
       
        \node [disk file ] (call map)        [right =  of build log                ] {  Файл с~информацией о~линковке};
        \draw [longline] (builder)         -| (call map);
        \draw [longline] (call map)        |- (aggregator);
       
        \node[fit=(call map.north) (sources) (summary), fitblock] (left_vert_base) {};
        
        
        
        % самая широкая часть правой вертикали
        \tikzset{every join/.style=nodraw}
%         \coordinate                          [right = of stat parser.north east-|left_vert_base.east, on chain=main_horz] (right_vert_anchorpoint); % right = of stat result?
        \coordinate                          [right = of stat result.south-|left_vert_base.east, on chain=main_horz] (right_vert_anchorpoint); % right = of stat result?
        \node [disk file ] (gdb script)      [on chain=main_horz] {  Скрипт для GDB};
        \coordinate                          [on chain=main_horz] (center_from_breakpointer);
        \node [disk file ] (modified exe)    [on chain=main_horz] {  Модифицированный исполняемый файл};
        \node[fit=(gdb script) (modified exe), fitblock] (right_vert_cross) {};
       
        % вверх
        \node [process   ] (breakpointer)    [on chain=rev_vert, above = of right_vert_cross] {  Модуль бинарного анализа};
        \draw [longline] (breakpointer)    -| (modified exe);
        \draw [longline] (breakpointer)    -| (gdb script);
%         \tikzset{every join/.style=revline}
%         \tikzset{every join/.style=shortline} %revline
%         \coordinate                          [on chain=rev_vert] (executable_to_breakpointer);
        \tikzset{every join/.style=shortline}
        \node [disk file ] (file executable) [on chain=rev_vert   ] {  Исполняемый\\файл};
        \draw [longline] (builder)         -| (file executable);
        
% %         \node [prompt    ] (executable)      [right = of  file executable          ] {  Путь до исполняемого файла};
%         \node [prompt    ] (executable)      [above = of  file executable.north-|modified exe          ] {  Путь до исполняемого файла};
%         \draw [revline] (executable_to_breakpointer)    -| (executable);            
        
        % вниз
        \tikzset{every join/.style=nodraw}
        \node [process   ] (gdb manager)     [on chain=main_vert, below = of right_vert_cross   ] {  Модуль управления отладчиком};
        \draw [longline] (modified exe)    |- (gdb manager);
        \draw [longline] (gdb script)      |- (gdb manager);

        \tikzset{every join/.style=shortline}
        \node [disk file ] (dyn result)      [on chain=main_vert                ] {  Результаты динамического анализа};
        \node [process   ] (dyn parser)      [on chain=main_vert                ] {  Модуль преобразования результатов динамического анализа};
        \node [disk file ] (dyn json)        [on chain=main_vert                ] {  Преобразованные результаты динамического анализа};
   
        \draw [longline] (dyn json)        |- (comparer);
     
        
       
% другой вариант, по раскладке ближе к исходному       
%         \coordinate[right = of sources, on chain=main_horz] (right_vert_anchorpoint);
%         \node [disk file ] (file executable) [on chain=main_vert, right = of right_vert_anchorpoint             ] {  Исполняемый\\файл};
%         \tikzset{every join/.style=shortline}
%         \coordinate[on chain=main_vert] (executable_to_breakpointer);
%         \node [prompt    ] (executable)      [right = of  file executable          ] {  Путь до исполняемого файла};
%         \draw [revline] (executable_to_breakpointer)    -| (executable);
% 
%         
%         \tikzset{every join/.style=longline}
%         \node [process   ] (breakpointer)    [on chain=main_vert   ] {  Модуль бинарного анализа};% решает последнее
%         \tikzset{every join/.style=nodraw}
%         \coordinate[on chain=main_vert] (center_from_breakpointer);
% %         \node[circle, fill=red] at  (center_from_breakpointer) {};
%       
%         
% %         \node [disk file ] (modified exe)    [right = of center_from_breakpointer, anchor = north west] {  Модифицированный исполняемый файл}; у~цилиндра north west почти на north
%         \node [disk file ] (modified exe)    [right = of center_from_breakpointer, anchor = after bottom] {  Модифицированный исполняемый файл};
%         \node [disk file ] (gdb script)      [left = of center_from_breakpointer|-modified exe] {  Скрипт для GDB};
%         \node [process   ] (gdb manager)     [on chain=main_vert, below = of breakpointer|-modified exe.south              ] {  Модуль управления отладчиком};
%         \tikzset{every join/.style=shortline}
%         \node [disk file ] (dyn result)      [on chain=main_vert                ] {  Результаты динамического анализа};
%         \node [process   ] (dyn parser)      [on chain=main_vert                ] {  Модуль преобразования результатов динамического анализа};
%         \node [disk file ] (dyn json)        [on chain=main_vert                ] {  Преобразованные результаты динамического анализа};
%         
%         
% %         \draw [line] (builder)         -| (build log);
% % 
% % %         \draw [line] (executable)      -| (breakpointer);
% % %         \draw [line] (file executable) -| (breakpointer);
%         \draw [longline] (breakpointer)    -| (modified exe);
%         \draw [longline] (breakpointer)    -| (gdb script);
% % 
%         \draw [longline] (modified exe)    |- (gdb manager);
%         \draw [longline] (gdb script)      |- (gdb manager);
% % %         \draw [-] (gdb manager)        -- (dyn result);
% % %         \draw [-] (dyn result)         -- (dyn parser);
% % %         \draw [-] (dyn parser)         -- (dyn json);
% % 
% % 
% % %         \draw [-] (build log)          -- (static analyzer);
% % %         \draw [-] (static analyzer)    -- (stat result);
% % %         \draw [-] (stat result)        -- (stat parser);
% % %         \draw [-] (stat parser)        -- (stat json);
% % %         \draw [-] (stat json)          -- (aggregator);
% % %         \draw [-] (aggregator)         -- (aggregator file);
% % 
% % %         \draw [-] (aggregator file)    -- (comparer);
%         \draw [longline] (dyn json)        |- (comparer);
% % %         \draw [-] (comparer)           -- (summary);
        
        
% контроль полей
%         \draw [red] (current bounding box.south east) rectangle (current bounding box.north west);
\end{tikzpicture}
}

    }
    \caption{Схема данных {\ProgModule}\label{fig:dataflow}}
\end{figure}
Из схемы данных \autoref{fig:dataflow} видно, что работу {\ProgModule} можно разбить на параллельные 
задачи.


\section{Алгоритм работы программы}\label{sec:ch2/sec5}
Работу {\ProgModule} можно разделить на функциональные этапы:
\begin{enumerate}
    \item Сборка анализиуемой программы 
    \item Статический анализ результатов сборки\label{statical-analysis-stage}
    \item Динамический анализ собранной программы\label{dynamical-analysis-stage}
    \item Сравнительный анализ результатов предыдущих шагов
\end{enumerate}

Причем \autoref{statical-analysis-stage} и \autoref{dynamical-analysis-stage} могут выполняться
одновременно, так как не имеют зависимости по данным.

\begin{figure}[!htbp]
    \centerfloat{
        % !TEX encoding = UTF-8 Unicode
% Úτƒ-8 encoded
% http://www.linux.org.ru/forum/general/10357036
\tikzset{
    pics/.cd,
    line/.style={draw, -latex'},
    every join/.style={line},
    u/.style={anchor=south},
    r/.style={anchor=west},
    fxd/.style={text width = 6em},
    it/.style={font={\small\itshape}},
    bf/.style={font={\small\bfseries}},
}
\tikzstyle{base} =
    [
        draw,
        on chain,
        on grid,
        align=center,
        minimum height=4ex,
        minimum width = 10ex,
        node distance = 6mm and 60mm,
        text badly centered,
        text width=5cm
    ]
\tikzstyle{coord} =
    [
        coordinate,
        on chain,
        on grid
    ]
\tikzstyle{cloud} =
    [
        base,
        ellipse,
        node distance = 3cm,
        minimum height = 2em,
        text width=2cm
    ]
\tikzstyle{decision} =
    [
        base,
        diamond,
        aspect=2,
        node distance = 2cm,
        inner sep = 0pt
    ]
\tikzstyle{block} =
    [
        rectangle,
        base,
        rounded corners,
        minimum height = 2em
    ]
\tikzstyle{print_block} =
    [
        base,
        tape,
        tape bend top=none,
    ]
\tikzstyle{io} =
    [
        base,
        trapezium,
        trapezium left angle = 70,
        trapezium right angle = 110,
    ]
\tikzstyle{prompt} =
    [
        base,
        trapezium,
        trapezium left angle = 90,
        trapezium right angle = 80,
        shape border rotate = 90
    ]
\tikzstyle{disk file} =
    [
        base,
        cylinder,
        aspect=0.2,
    ]
\tikzstyle{process} =
    [
        rectangle,
        base,
    ]
\makeatletter
\pgfkeys{/pgf/.cd,
    subrtshape w/.initial=2mm,
    cycleshape w/.initial=2mm
}
\pgfdeclareshape{parallelshape}{
    \inheritsavedanchors[from=rectangle]
    \inheritanchorborder[from=rectangle]
    \inheritanchor[from=rectangle]{north}
    \inheritanchor[from=rectangle]{center}
    \inheritanchor[from=rectangle]{west}
    \inheritanchor[from=rectangle]{east}
    \inheritanchor[from=rectangle]{mid}
    \inheritanchor[from=rectangle]{base}
    \inheritanchor[from=rectangle]{south}
    \backgroundpath{
        \southwest \pgf@xa=\pgf@x \pgf@ya=\pgf@y
        \northeast \pgf@xb=\pgf@x \pgf@yb=\pgf@y
        \pgfmathsetlength\pgfutil@tempdima{\pgfkeysvalueof{/pgf/subrtshape w}}
        \def\ppd@offset{\pgfpoint{\pgfutil@tempdima}{0ex}}
        \def\ppd@offsetm{\pgfpoint{-\pgfutil@tempdima}{0ex}}
        \pgfpathmoveto{\pgfqpoint{\pgf@xa}{\pgf@ya}}
            \pgfpathlineto{\pgfqpoint{\pgf@xb}{\pgf@ya}}
        \pgfpathclose
        \pgfpathmoveto{\pgfqpoint{\pgf@xb}{\pgf@yb}}
            \pgfpathlineto{\pgfqpoint{\pgf@xa}{\pgf@yb}}
        \pgfpathclose
    }
}
\pgfdeclareshape{subrtshape}{
    \inheritsavedanchors[from=rectangle]
    \inheritanchorborder[from=rectangle]
    \inheritanchor[from=rectangle]{north}
    \inheritanchor[from=rectangle]{center}
    \inheritanchor[from=rectangle]{west}
    \inheritanchor[from=rectangle]{east}
    \inheritanchor[from=rectangle]{mid}
    \inheritanchor[from=rectangle]{base}
    \inheritanchor[from=rectangle]{south}
    \backgroundpath{
        \southwest \pgf@xa=\pgf@x \pgf@ya=\pgf@y
        \northeast \pgf@xb=\pgf@x \pgf@yb=\pgf@y
        \pgfmathsetlength\pgfutil@tempdima{\pgfkeysvalueof{/pgf/subrtshape w}}
        \def\ppd@offset{\pgfpoint{\pgfutil@tempdima}{0ex}}
        \def\ppd@offsetm{\pgfpoint{-\pgfutil@tempdima}{0ex}}
        \pgfpathmoveto{\pgfqpoint{\pgf@xa}{\pgf@ya}}
        \pgfpathlineto{\pgfqpoint{\pgf@xb}{\pgf@ya}}
        \pgfpathlineto{\pgfqpoint{\pgf@xb}{\pgf@yb}}
        \pgfpathlineto{\pgfqpoint{\pgf@xa}{\pgf@yb}}
        \pgfpathclose
        \pgfpathmoveto{\pgfpointadd{\pgfpoint{\pgf@xa}{\pgf@yb}}{\ppd@offsetm}}
        \pgfpathlineto{\pgfpointadd{\pgfpoint{\pgf@xa}{\pgf@ya}}{\ppd@offsetm}}
        \pgfpathlineto{\pgfpointadd{\pgfpoint{\pgf@xb}{\pgf@ya}}{\ppd@offset}}
        \pgfpathlineto{\pgfpointadd{\pgfpoint{\pgf@xb}{\pgf@yb}}{\ppd@offset}}
        \pgfpathclose
    }
}
\pgfdeclareshape{cyclebegshape}{
    \inheritsavedanchors[from=rectangle]
    \inheritanchorborder[from=rectangle]
    \inheritanchor[from=rectangle]{north}
    \inheritanchor[from=rectangle]{center}
    \inheritanchor[from=rectangle]{west}
    \inheritanchor[from=rectangle]{east}
    \inheritanchor[from=rectangle]{mid}
    \inheritanchor[from=rectangle]{base}
    \inheritanchor[from=rectangle]{south}
    \backgroundpath{
        \southwest \pgf@xa=\pgf@x \pgf@ya=\pgf@y
        \northeast \pgf@xb=\pgf@x \pgf@yb=\pgf@y
        \pgfmathsetlength\pgfutil@tempdima{\pgfkeysvalueof{/pgf/cycleshape w}}
        \pgfpathmoveto{\pgfqpoint{\pgf@xa}{\pgf@ya}}
\pgfpathlineto{\pgfpointadd{\pgfpoint{\pgf@xa}{\pgf@yb}}{\pgfpoint{0ex}{-\pgfutil@tempdima}}}
\pgfpathlineto{\pgfpointadd{\pgfpoint{\pgf@xa}{\pgf@yb}}{\pgfpoint{\pgfutil@tempdima}{0ex}}}
\pgfpathlineto{\pgfpointadd{\pgfpoint{\pgf@xb}{\pgf@yb}}{\pgfpoint{-\pgfutil@tempdima}{0ex}}}
\pgfpathlineto{\pgfpointadd{\pgfpoint{\pgf@xb}{\pgf@yb}}{\pgfpoint{0ex}{-\pgfutil@tempdima}}}
\pgfpathlineto{\pgfqpoint{\pgf@xb}{\pgf@ya}}
        \pgfpathclose
    }
}
\pgfdeclareshape{cycleendshape}{
    \inheritsavedanchors[from=rectangle]
    \inheritanchorborder[from=rectangle]
    \inheritanchor[from=rectangle]{north}
    \inheritanchor[from=rectangle]{center}
    \inheritanchor[from=rectangle]{west}
    \inheritanchor[from=rectangle]{east}
    \inheritanchor[from=rectangle]{mid}
    \inheritanchor[from=rectangle]{base}
    \inheritanchor[from=rectangle]{south}
    \backgroundpath{
        \southwest \pgf@xa=\pgf@x \pgf@ya=\pgf@y
        \northeast \pgf@xb=\pgf@x \pgf@yb=\pgf@y
        \pgfmathsetlength\pgfutil@tempdima{\pgfkeysvalueof{/pgf/cycleshape w}}
        \pgfpathmoveto{\pgfqpoint{\pgf@xb}{\pgf@yb}}
\pgfpathlineto{\pgfpointadd{\pgfpoint{\pgf@xb}{\pgf@ya}}{\pgfpoint{0ex}{\pgfutil@tempdima}}}
\pgfpathlineto{\pgfpointadd{\pgfpoint{\pgf@xb}{\pgf@ya}}{\pgfpoint{-\pgfutil@tempdima}{0ex}}}
\pgfpathlineto{\pgfpointadd{\pgfpoint{\pgf@xa}{\pgf@ya}}{\pgfpoint{\pgfutil@tempdima}{0ex}}}
\pgfpathlineto{\pgfpointadd{\pgfpoint{\pgf@xa}{\pgf@ya}}{\pgfpoint{0ex}{\pgfutil@tempdima}}}
\pgfpathlineto{\pgfqpoint{\pgf@xa}{\pgf@yb}}
        \pgfpathclose
    }
}
\makeatother
\tikzstyle{subroutine} =
    [
        base,
        subrtshape,
    ]
\tikzstyle{cyclebegin} =
    [
        base,
        cyclebegshape,
    ]
\tikzstyle{cycleend} =
    [
        base,
        cycleendshape,
    ]
\tikzstyle{connector} =
    [
        base,
        circle,
    ]

\tikzstyle{parallel} =
    [
        base,
        parallelshape,
    ]
\begin{tikzpicture}[%
    start chain=going below,    % General flow is top-to-bottom
    node distance=6mm and 30mm, % Global setup of box spacing
    scale=0.7, 
    every node/.style={scale=0.72}
    ] 
        \node [cloud    ] (makefile)        [left  = 2cm ]                   {\small Начало};
        \node [process  ] (builder)         [below = 1cm of makefile]        {\small Сборщик};
        \node [parallel ] (parallel)        [below of = builder]             {};
        \node [process  ] (breakpointer)    [below right = 3cm of builder]   {\small Бинарный анализ};
        \node [process  ] (static analyzer) [below left = 3cm of builder]    {\small Статический анализ};
        \node [process  ] (stat parser)     [below = 2cm of static analyzer] {\small Преобразование результатов статического анализа};
        \node [process  ] (aggregator)      [below = 2cm of stat parser]     {\small Агрегирование результатов линковки и статического анализа};
        \node [process  ] (gdb manager)     [below = 2cm of breakpointer]    {\small Динамический анализ};
        \node [process  ] (dyn parser)      [below = 2cm of gdb manager]     {\small Преобразование результатов динамического анализа};
        \node [cloud    ] (end)             [below = 11cm of makefile]     {\small Конец};
        \node [process  ] (comparer)        [above = 1cm of end]             {\small Сравнительный анализ};
        \node [parallel ] (parallel aggr)   [above = 1cm of comparer]             {};

        \draw [->] (makefile)        -- (builder);
        \draw [-] (builder)          -- (parallel);

        \draw [-] (parallel)         -- (static analyzer);
        \draw [-] (parallel)         -- (breakpointer);

        \draw [->] (static analyzer) -- (stat parser);
        \draw [->] (stat parser)     -- (aggregator);

        \draw [->] (breakpointer)    -- (gdb manager);
        \draw [->] (gdb manager)     -- (dyn parser);

        \draw [-] (aggregator)       -- (parallel aggr);
        \draw [-] (dyn parser)       -- (parallel aggr);
        \draw [-] (parallel aggr)    -- (comparer);
        \draw [->] (comparer)        -- (end);

\end{tikzpicture}

    }
    \caption{Алгоритм работы {\ProgModule}\label{fig:algorithm}}
\end{figure}

