\chapter{Конструкторский раздел}\label{ch:ch2}
\section{Обоснование выбора языка программирования и среды разработки}\label{sec:ch2/sec1}

Для разработки {\ProgModule} понадобится сверхвысокоуровневый язык с кросс-платформенной
стандартной библиотекой, который позволит точно и лаконично описать этапы анализа,
а так же имеющий высокую скорость исполнения, для анализа больших объемов исходного кода и
исполняемых файлов.

{\small
    \setlength{\tabcolsep}{2pt}
    \begin{longtable}{*{5}{| c}|}
        \hline
        \diagbox[width=8cm]{Свойства}{Язык программирования} &
            \makecell{Nim \autocite{nim}} &
            \makecell{Python \autocite{python}} &
            \makecell{Perl \autocite{perl}} &
            \makecell{C/C++} \\
        \hline
            \makecell{Сверхвысокоуровневость} & 
            \greencell{Да} & 
            \greencell{Да} &
            \greencell{Да} &
            \redcell{Нет} \\
        \hline
            \makecell{Компилируется в\\машинный код} & 
            \greencell{Да} & 
            \redcell{Нет} &
            \redcell{Нет} &
            \greencell{Да} \\
        \hline
            \makecell{Количество функции в\\стандартной библиотеке} & 
            5585 & 
            638 &
            1338 &
            1224 \\
        \hline
            \makecell{Портируемость} & 
            \greencell{Есть} & 
            \greencell{Есть} &
            \greencell{Есть} &
            \yellowcell{\makecell{Есть,\\но неудобная}}\\
        \hline
            \makecell{Встроенная\\генерация документации} & 
            \greencell{Есть} & 
            \greencell{Есть} &
            \greencell{Есть} &
            \redcell{Нет}\\
        \hline
            \makecell{Статическая типизация} & 
            \greencell{Есть} & 
            \redcell{Нет} &
            \redcell{Нет} &
            \greencell{Есть}\\
        \hline
            \makecell{Автоматическое\\управление памятью} & 
            \greencell{Есть} & 
            \greencell{Есть} &
            \greencell{Есть} &
            \greencell{Есть} \\
        \hline
            \makecell{Опыт использования} & 
            \greencell{Есть} & 
            \greencell{Есть} &
            \redcell{Нет} &
            \greencell{Есть} \\
        \hline
    \caption{\label{table:languages-comparsion}
           Сравнительная таблица языков программирования}
    \end{longtable}
}

\section{Cхема данных}\label{sec:ch2/sec2}

\begin{figure}[!htbp]
    \centerfloat{
        % !TEX encoding = UTF-8 Unicode
% Úτƒ-8 encoded
% http://www.linux.org.ru/forum/general/10357036
\tikzset{
    line/.style={draw, -latex'},
    every join/.style={line},
    u/.style={anchor=south},
    r/.style={anchor=west},
    fxd/.style={text width = 6em},
    it/.style={font={\small\itshape}},
    bf/.style={font={\small\bfseries}}

}
\tikzstyle{base} =
    [
        draw,
        on chain,
        on grid,
        align=center,
        minimum height=4ex,
        minimum width = 10ex,
        node distance = 6mm and 60mm,
        text badly centered,
        text width=5cm
    ]
\tikzstyle{coord} =
    [
        coordinate,
        on chain,
        on grid
    ]
\tikzstyle{cloud} =
    [
        base,
        ellipse,
        node distance = 3cm,
        minimum height = 2em
    ]
\tikzstyle{decision} =
    [
        base,
        diamond,
        aspect=2,
        node distance = 2cm,
        inner sep = 0pt
    ]
\tikzstyle{block} =
    [
        rectangle,
        base,
        rounded corners,
        minimum height = 2em
    ]
\tikzstyle{print_block} =
    [
        base,
        tape,
        tape bend top=none,
    ]
\tikzstyle{io} =
    [
        base,
        trapezium,
        trapezium left angle = 70,
        trapezium right angle = 110,
    ]
\tikzstyle{prompt} =
    [
        base,
        trapezium,
        trapezium left angle = 90,
        trapezium right angle = 80,
        shape border rotate = 90
    ]
\tikzstyle{disk file} =
    [
        base,
        cylinder,
        aspect=0.2,
    ]
\tikzstyle{process} =
    [
        rectangle,
        base,
    ]
\makeatletter
\pgfkeys{/pgf/.cd,
    subrtshape w/.initial=2mm,
    cycleshape w/.initial=2mm
}
\pgfdeclareshape{subrtshape}{
    \inheritsavedanchors[from=rectangle]
    \inheritanchorborder[from=rectangle]
    \inheritanchor[from=rectangle]{north}
    \inheritanchor[from=rectangle]{center}
    \inheritanchor[from=rectangle]{west}
    \inheritanchor[from=rectangle]{east}
    \inheritanchor[from=rectangle]{mid}
    \inheritanchor[from=rectangle]{base}
    \inheritanchor[from=rectangle]{south}
    \backgroundpath{
        \southwest \pgf@xa=\pgf@x \pgf@ya=\pgf@y
        \northeast \pgf@xb=\pgf@x \pgf@yb=\pgf@y
        \pgfmathsetlength\pgfutil@tempdima{\pgfkeysvalueof{/pgf/subrtshape w}}
        \def\ppd@offset{\pgfpoint{\pgfutil@tempdima}{0ex}}
        \def\ppd@offsetm{\pgfpoint{-\pgfutil@tempdima}{0ex}}
        \pgfpathmoveto{\pgfqpoint{\pgf@xa}{\pgf@ya}}
        \pgfpathlineto{\pgfqpoint{\pgf@xb}{\pgf@ya}}
        \pgfpathlineto{\pgfqpoint{\pgf@xb}{\pgf@yb}}
        \pgfpathlineto{\pgfqpoint{\pgf@xa}{\pgf@yb}}
        \pgfpathclose
        \pgfpathmoveto{\pgfpointadd{\pgfpoint{\pgf@xa}{\pgf@yb}}{\ppd@offsetm}}
        \pgfpathlineto{\pgfpointadd{\pgfpoint{\pgf@xa}{\pgf@ya}}{\ppd@offsetm}}
        \pgfpathlineto{\pgfpointadd{\pgfpoint{\pgf@xb}{\pgf@ya}}{\ppd@offset}}
        \pgfpathlineto{\pgfpointadd{\pgfpoint{\pgf@xb}{\pgf@yb}}{\ppd@offset}}
        \pgfpathclose
    }
}
\pgfdeclareshape{cyclebegshape}{
    \inheritsavedanchors[from=rectangle]
    \inheritanchorborder[from=rectangle]
    \inheritanchor[from=rectangle]{north}
    \inheritanchor[from=rectangle]{center}
    \inheritanchor[from=rectangle]{west}
    \inheritanchor[from=rectangle]{east}
    \inheritanchor[from=rectangle]{mid}
    \inheritanchor[from=rectangle]{base}
    \inheritanchor[from=rectangle]{south}
    \backgroundpath{
        \southwest \pgf@xa=\pgf@x \pgf@ya=\pgf@y
        \northeast \pgf@xb=\pgf@x \pgf@yb=\pgf@y
        \pgfmathsetlength\pgfutil@tempdima{\pgfkeysvalueof{/pgf/cycleshape w}}
        \pgfpathmoveto{\pgfqpoint{\pgf@xa}{\pgf@ya}}
\pgfpathlineto{\pgfpointadd{\pgfpoint{\pgf@xa}{\pgf@yb}}{\pgfpoint{0ex}{-\pgfutil@tempdima}}}
\pgfpathlineto{\pgfpointadd{\pgfpoint{\pgf@xa}{\pgf@yb}}{\pgfpoint{\pgfutil@tempdima}{0ex}}}
\pgfpathlineto{\pgfpointadd{\pgfpoint{\pgf@xb}{\pgf@yb}}{\pgfpoint{-\pgfutil@tempdima}{0ex}}}
\pgfpathlineto{\pgfpointadd{\pgfpoint{\pgf@xb}{\pgf@yb}}{\pgfpoint{0ex}{-\pgfutil@tempdima}}}
\pgfpathlineto{\pgfqpoint{\pgf@xb}{\pgf@ya}}
        \pgfpathclose
    }
}
\pgfdeclareshape{cycleendshape}{
    \inheritsavedanchors[from=rectangle]
    \inheritanchorborder[from=rectangle]
    \inheritanchor[from=rectangle]{north}
    \inheritanchor[from=rectangle]{center}
    \inheritanchor[from=rectangle]{west}
    \inheritanchor[from=rectangle]{east}
    \inheritanchor[from=rectangle]{mid}
    \inheritanchor[from=rectangle]{base}
    \inheritanchor[from=rectangle]{south}
    \backgroundpath{
        \southwest \pgf@xa=\pgf@x \pgf@ya=\pgf@y
        \northeast \pgf@xb=\pgf@x \pgf@yb=\pgf@y
        \pgfmathsetlength\pgfutil@tempdima{\pgfkeysvalueof{/pgf/cycleshape w}}
        \pgfpathmoveto{\pgfqpoint{\pgf@xb}{\pgf@yb}}
\pgfpathlineto{\pgfpointadd{\pgfpoint{\pgf@xb}{\pgf@ya}}{\pgfpoint{0ex}{\pgfutil@tempdima}}}
\pgfpathlineto{\pgfpointadd{\pgfpoint{\pgf@xb}{\pgf@ya}}{\pgfpoint{-\pgfutil@tempdima}{0ex}}}
\pgfpathlineto{\pgfpointadd{\pgfpoint{\pgf@xa}{\pgf@ya}}{\pgfpoint{\pgfutil@tempdima}{0ex}}}
\pgfpathlineto{\pgfpointadd{\pgfpoint{\pgf@xa}{\pgf@ya}}{\pgfpoint{0ex}{\pgfutil@tempdima}}}
\pgfpathlineto{\pgfqpoint{\pgf@xa}{\pgf@yb}}
        \pgfpathclose
    }
}
\makeatother
\tikzstyle{subroutine} =
    [
        base,
        subrtshape,
    ]
\tikzstyle{cyclebegin} =
    [
        base,
        cyclebegshape,
    ]
\tikzstyle{cycleend} =
    [
        base,
        cycleendshape,
    ]
\tikzstyle{connector} =
    [
        base,
        circle,
    ]
\begin{tikzpicture}[%
    start chain=going below,    % General flow is top-to-bottom
    node distance=6mm and 30mm, % Global setup of box spacing
    scale=0.7, 
    every node/.style={scale=0.72}
    ] 
        \node [prompt   ] (makefile)        [left  = 2cm ]                      {\small Путь до папки с Makefile};
        \node [process  ] (builder)         [below = 2cm of makefile]           {\small Сборщик};
        \node [prompt   ] (executable)      [right = 10cm of makefile]          {\small Путь до исполняемого файла};
        \node [process  ] (breakpointer)                                        {\small Модуль бинарного анализа};
        \node [disk file] (modified exe)    [below right = 3cm of breakpointer] {\small Модифицированный исполняемый файл};
        \node [disk file] (build log)       [below right = 3cm of builder]      {\small Файл с информацией о сборке};
        \node [process  ] (static analyzer) [below = 3cm of build log]          {\small Модуль статического анализа};
        \node [disk file] (call map)        [below left = 3cm of builder]       {\small Файл с информацией о линковке};
        \node [disk file] (gdb script)      [below left = 3cm of breakpointer]  {\small Скрипт для GDB};
        \node [process  ] (gdb manager)     [below = 4cm of breakpointer]       {\small Модуль управления отладчиком};
        \node [disk file] (dyn result)      [below = 2cm of gdb manager]        {\small Результаты динамического анализа};
        \node [process  ] (dyn parser)      [below = 2cm of dyn result]         {\small Модуль преобразования результатов динамического анализа};
        \node [disk file] (dyn json)        [below = 2cm of dyn parser]         {\small Преобразованные результаты динамического анализа};
        \node [disk file] (stat result)     [below = 2cm of static analyzer]    {\small Результаты статического анализа};
        \node [process  ] (stat parser)     [below = 2cm of stat result]        {\small Модуль преобразования результатов статического анализа};
        \node [disk file] (stat json)       [below = 2cm of stat parser]        {\small Преобразованные результаты статического анализа};
        \node [process  ] (aggregator)      [below = 2cm of stat json]          {\small Модуль агрегирования результатов линковки и статического анализа};
        \node [disk file] (aggregator file) [below = 2cm of aggregator]         {\small Агрегированные результаты линковки и статического анализа};
        \node [process  ] (comparer)                                            {\small Модуль сравнительного анализа};
        \node [disk file] (summary)                                             {\small Результаты сравнительного анализа};

        \draw [line] (makefile)        -- (builder);
        \draw [line] (builder)         -| (call map);
        \draw [line] (builder)         -| (build log);

        \draw [line] (executable)      -- (breakpointer);
        \draw [line] (breakpointer)    -| (modified exe);
        \draw [line] (breakpointer)    -| (gdb script);

        \draw [line] (modified exe)    |- (gdb manager);
        \draw [line] (gdb script)      |- (gdb manager);
        \draw [line] (gdb manager)     -- (dyn result);
        \draw [line] (dyn result)      -- (dyn parser);
        \draw [line] (dyn parser)      -- (dyn json);


        \draw [line] (build log)       -- (static analyzer);
        \draw [line] (static analyzer) -- (stat result);
        \draw [line] (stat result)     -- (stat parser);
        \draw [line] (stat parser)     -- (stat json);
        \draw [line] (call map)        |- (aggregator);
        \draw [line] (stat json)       -- (aggregator);
        \draw [line] (aggregator)      -- (aggregator file);

        \draw [line] (aggregator file) -- (comparer);
        \draw [line] (dyn json)        |- (comparer);
        \draw [line] (comparer)        -- (summary);

\end{tikzpicture}

    }
    \caption{Схема данных {\ProgModule}\label{fig:dataflow}}
\end{figure}

Работу {\ProgModule} можно разделить на функциональные этапы:
\begin{enumerate}
    \item Сборка анализиуемой программы 
    \item Статический анализ результатов сборки\label{statical-analysis-stage}
    \item Динамический анализ собранной программы\label{dynamical-analysis-stage}
    \item Сравнительный анализ результатов предыдущих шагов
\end{enumerate}

Причем \autoref{statical-analysis-stage} и \autoref{dynamical-analysis-stage} могут выполняться
одновременно, так как не имеют зависимости по данным.
Что отражено в алгоритме решения задачи.

\section{Алгоритм}\label{sec:ch2/sec3}

\begin{figure}[!htbp]
    \centerfloat{
        % !TEX encoding = UTF-8 Unicode
% Úτƒ-8 encoded
% http://www.linux.org.ru/forum/general/10357036
\tikzset{
    line/.style={draw, -latex'},
    every join/.style={line},
    u/.style={anchor=south},
    r/.style={anchor=west},
    fxd/.style={text width = 6em},
    it/.style={font={\small\itshape}},
    bf/.style={font={\small\bfseries}},
}
\tikzstyle{base_long} =
    [
        draw,
        on chain,
        on grid,
        align=center,
        minimum height=4ex,
        minimum width = 10ex,
        node distance = 6mm and 60mm,
        text badly centered,
    ]
\tikzstyle{base} =
    [
        draw,
        on chain,
        on grid,
        align=center,
        minimum height=4ex,
        minimum width = 10ex,
        node distance = 6mm and 60mm,
        text badly centered,
        text width=5cm
    ]
\tikzstyle{coord} =
    [
        coordinate,
        on chain,
        on grid
    ]
\tikzstyle{cloud} =
    [
        base,
        ellipse,
        node distance = 3cm,
        minimum height = 2em,
        text width=2cm
    ]
\tikzstyle{decision} =
    [
        base,
        diamond,
        aspect=2,
        node distance = 2cm,
        inner sep = 0pt
    ]
\tikzstyle{block} =
    [
        rectangle,
        base,
        rounded corners,
        minimum height = 2em
    ]
\tikzstyle{print_block} =
    [
        base,
        tape,
        tape bend top=none,
    ]
\tikzstyle{io} =
    [
        base,
        trapezium,
        trapezium left angle = 70,
        trapezium right angle = 110,
    ]
\tikzstyle{prompt} =
    [
        base,
        trapezium,
        trapezium left angle = 90,
        trapezium right angle = 80,
        shape border rotate = 90
    ]
\tikzstyle{disk file} =
    [
        base,
        cylinder,
        aspect=0.2,
    ]
\tikzstyle{process} =
    [
        rectangle,
        base,
    ]
\makeatletter
\pgfkeys{/pgf/.cd,
    subrtshape w/.initial=2mm,
    cycleshape w/.initial=2mm
}
\pgfdeclareshape{parallelshape}{
    \inheritsavedanchors[from=rectangle]
    \inheritanchorborder[from=rectangle]
    \inheritanchor[from=rectangle]{north}
    \inheritanchor[from=rectangle]{center}
    \inheritanchor[from=rectangle]{west}
    \inheritanchor[from=rectangle]{east}
    \inheritanchor[from=rectangle]{mid}
    \inheritanchor[from=rectangle]{base}
    \inheritanchor[from=rectangle]{south}
    \backgroundpath{
        \southwest \pgf@xa=\pgf@x \pgf@ya=\pgf@y
        \northeast \pgf@xb=\pgf@x \pgf@yb=\pgf@y
        \def\ppd@offset{\pgfpoint{\pgfutil@tempdima}{0ex}}
        \def\ppd@offsetm{\pgfpoint{-\pgfutil@tempdima}{0ex}}
        \pgfpathmoveto{\pgfqpoint{\pgf@xa}{\pgf@ya}}
            \pgfpathlineto{\pgfqpoint{\pgf@xb}{\pgf@ya}}
        \pgfpathclose
        \pgfpathmoveto{\pgfqpoint{\pgf@xb}{\pgf@yb}}
            \pgfpathlineto{\pgfqpoint{\pgf@xa}{\pgf@yb}}
        \pgfpathclose
    }
}
\pgfdeclareshape{subrtshape}{
    \inheritsavedanchors[from=rectangle]
    \inheritanchorborder[from=rectangle]
    \inheritanchor[from=rectangle]{north}
    \inheritanchor[from=rectangle]{center}
    \inheritanchor[from=rectangle]{west}
    \inheritanchor[from=rectangle]{east}
    \inheritanchor[from=rectangle]{mid}
    \inheritanchor[from=rectangle]{base}
    \inheritanchor[from=rectangle]{south}
    \backgroundpath{
        \southwest \pgf@xa=\pgf@x \pgf@ya=\pgf@y
        \northeast \pgf@xb=\pgf@x \pgf@yb=\pgf@y
        \pgfmathsetlength\pgfutil@tempdima{\pgfkeysvalueof{/pgf/subrtshape w}}
        \def\ppd@offset{\pgfpoint{\pgfutil@tempdima}{0ex}}
        \def\ppd@offsetm{\pgfpoint{-\pgfutil@tempdima}{0ex}}
        \pgfpathmoveto{\pgfqpoint{\pgf@xa}{\pgf@ya}}
        \pgfpathlineto{\pgfqpoint{\pgf@xb}{\pgf@ya}}
        \pgfpathlineto{\pgfqpoint{\pgf@xb}{\pgf@yb}}
        \pgfpathlineto{\pgfqpoint{\pgf@xa}{\pgf@yb}}
        \pgfpathclose
        \pgfpathmoveto{\pgfpointadd{\pgfpoint{\pgf@xa}{\pgf@yb}}{\ppd@offsetm}}
        \pgfpathlineto{\pgfpointadd{\pgfpoint{\pgf@xa}{\pgf@ya}}{\ppd@offsetm}}
        \pgfpathlineto{\pgfpointadd{\pgfpoint{\pgf@xb}{\pgf@ya}}{\ppd@offset}}
        \pgfpathlineto{\pgfpointadd{\pgfpoint{\pgf@xb}{\pgf@yb}}{\ppd@offset}}
        \pgfpathclose
    }
}
\pgfdeclareshape{cyclebegshape}{
    \inheritsavedanchors[from=rectangle]
    \inheritanchorborder[from=rectangle]
    \inheritanchor[from=rectangle]{north}
    \inheritanchor[from=rectangle]{center}
    \inheritanchor[from=rectangle]{west}
    \inheritanchor[from=rectangle]{east}
    \inheritanchor[from=rectangle]{mid}
    \inheritanchor[from=rectangle]{base}
    \inheritanchor[from=rectangle]{south}
    \backgroundpath{
        \southwest \pgf@xa=\pgf@x \pgf@ya=\pgf@y
        \northeast \pgf@xb=\pgf@x \pgf@yb=\pgf@y
        \pgfmathsetlength\pgfutil@tempdima{\pgfkeysvalueof{/pgf/cycleshape w}}
        \pgfpathmoveto{\pgfqpoint{\pgf@xa}{\pgf@ya}}
\pgfpathlineto{\pgfpointadd{\pgfpoint{\pgf@xa}{\pgf@yb}}{\pgfpoint{0ex}{-\pgfutil@tempdima}}}
\pgfpathlineto{\pgfpointadd{\pgfpoint{\pgf@xa}{\pgf@yb}}{\pgfpoint{\pgfutil@tempdima}{0ex}}}
\pgfpathlineto{\pgfpointadd{\pgfpoint{\pgf@xb}{\pgf@yb}}{\pgfpoint{-\pgfutil@tempdima}{0ex}}}
\pgfpathlineto{\pgfpointadd{\pgfpoint{\pgf@xb}{\pgf@yb}}{\pgfpoint{0ex}{-\pgfutil@tempdima}}}
\pgfpathlineto{\pgfqpoint{\pgf@xb}{\pgf@ya}}
        \pgfpathclose
    }
}
\pgfdeclareshape{cycleendshape}{
    \inheritsavedanchors[from=rectangle]
    \inheritanchorborder[from=rectangle]
    \inheritanchor[from=rectangle]{north}
    \inheritanchor[from=rectangle]{center}
    \inheritanchor[from=rectangle]{west}
    \inheritanchor[from=rectangle]{east}
    \inheritanchor[from=rectangle]{mid}
    \inheritanchor[from=rectangle]{base}
    \inheritanchor[from=rectangle]{south}
    \backgroundpath{
        \southwest \pgf@xa=\pgf@x \pgf@ya=\pgf@y
        \northeast \pgf@xb=\pgf@x \pgf@yb=\pgf@y
        \pgfmathsetlength\pgfutil@tempdima{\pgfkeysvalueof{/pgf/cycleshape w}}
        \pgfpathmoveto{\pgfqpoint{\pgf@xb}{\pgf@yb}}
\pgfpathlineto{\pgfpointadd{\pgfpoint{\pgf@xb}{\pgf@ya}}{\pgfpoint{0ex}{\pgfutil@tempdima}}}
\pgfpathlineto{\pgfpointadd{\pgfpoint{\pgf@xb}{\pgf@ya}}{\pgfpoint{-\pgfutil@tempdima}{0ex}}}
\pgfpathlineto{\pgfpointadd{\pgfpoint{\pgf@xa}{\pgf@ya}}{\pgfpoint{\pgfutil@tempdima}{0ex}}}
\pgfpathlineto{\pgfpointadd{\pgfpoint{\pgf@xa}{\pgf@ya}}{\pgfpoint{0ex}{\pgfutil@tempdima}}}
\pgfpathlineto{\pgfqpoint{\pgf@xa}{\pgf@yb}}
        \pgfpathclose
    }
}
\makeatother
\tikzstyle{subroutine} =
    [
        base,
        subrtshape,
    ]
\tikzstyle{cyclebegin} =
    [
        base,
        cyclebegshape,
    ]
\tikzstyle{cycleend} =
    [
        base,
        cycleendshape,
    ]
\tikzstyle{connector} =
    [
        base,
        circle,
    ]

\tikzstyle{parallel} =
    [
        base_long,
        parallelshape,
    ]
\begin{tikzpicture}[%
    start chain=going below,    % General flow is top-to-bottom
    node distance=6mm and 30mm, % Global setup of box spacing
    ] 
        \node [cloud    ] (makefile)        [left  = 4cm ]                   {\small Начало};
        \node [process  ] (builder)         [below = 3cm of makefile]        {\small Сборщик};
        \node [parallel ] (parallel)        [below of = builder, yscale=0.3] {\rptf[75]{\ }};
        \node [process  ] (breakpointer)    [below right = 5cm of builder]   {\small Бинарный анализ};
        \node [process  ] (static analyzer) [below left = 5cm of builder]    {\small Статический анализ};
        \node [process  ] (stat parser)     [below = 4cm of static analyzer] {\small Преобразование результатов статического анализа};
        \node [process  ] (aggregator)      [below = 4cm of stat parser]     {\small Агрегирование результатов линковки и статического анализа};
        \node [process  ] (gdb manager)     [below = 4cm of breakpointer]    {\small Динамический анализ};
        \node [process  ] (dyn parser)      [below = 4cm of gdb manager]     {\small Преобразование результатов динамического анализа};
        \node [cloud    ] (end)             [below = 23cm of makefile]       {\small Конец};
        \node [process  ] (comparer)        [above = 3cm of end]             {\small Сравнительный анализ};
        \node [parallel ] (parallel aggr)   [above = 3cm of comparer, yscale=0.3] {\rpts[75]{\ }};

        \draw [line] (makefile) -- (builder);
        \draw [-] (builder)     -- (parallel);

        \draw [-] (parallel)  -- +(-3.55,-0.140) -- +(-3.55,-1.9) -- (static analyzer);
        \draw [-] (parallel)  -- +(+3.55,-0.140) -- +(+3.55,-1.9) -- (breakpointer);

        \draw [line] (static analyzer) -- (stat parser);
        \draw [line] (stat parser)     -- (aggregator);

        \draw [line] (breakpointer)    -- (gdb manager);
        \draw [line] (gdb manager)     -- (dyn parser);

        \draw [-] (aggregator) -- +(+0,-1.05) -- +(+0,-2.33) -- (parallel aggr);
        \draw [-] (dyn parser) -- +(-0,-1.09) -- +(-0,-2.33) -- (parallel aggr);
        \draw [-] (parallel aggr)    -- (comparer);
        \draw [line] (comparer)        -- (end);

\end{tikzpicture}

    }
    \caption{Алгоритм {\ProgModule}\label{fig:algorithm}}
\end{figure}

\newpage
Для выравнивания изображения по-центру используется команда \verb+\centerfloat+, которая является во
многом улучшенной версией встроенной команды \verb+\centering+.

\section{Длинное название параграфа, в котором мы узнаём как сделать две картинки с~общим номером и названием}\label{sec:ch2/sect2}

А это две картинки под общим номером и названием:
\begin{figure}[ht]
  \begin{minipage}[b][][b]{0.49\linewidth}\centering
    \includegraphics[width=0.5\linewidth]{knuth1} \\ а)
  \end{minipage}
  \hfill
  \begin{minipage}[b][][b]{0.49\linewidth}\centering
    \includegraphics[width=0.5\linewidth]{knuth2} \\ б)
  \end{minipage}
  \caption{Очень длинная подпись к изображению,
      на котором представлены две фотографии Дональда Кнута}
  \label{fig:knuth}
\end{figure}

Те~же~две картинки под~общим номером и~названием,
но с автоматизированной нумерацией подрисунков:
\begin{figure}[ht]
    \centerfloat{
        \hfill
        \subbottom[List-of-Figures entry][Первый подрисунок\label{fig:knuth_2-1}]{%
            \includegraphics[width=0.25\linewidth]{knuth1}}
        \hfill
        \subbottom[\label{fig:knuth_2-2}]{%
            \includegraphics[width=0.25\linewidth]{knuth2}}
        \hfill
        \subbottom[Третий подрисунок]{%
            \includegraphics[width=0.3\linewidth]{example-image-c}}
        \hfill
    }
    \legend{Подрисуночный текст, описывающий обозначения, например. Согласно
    ГОСТ 2.105, пункт 4.3.1, располагается перед наименованием рисунка.}
    \caption[Этот текст попадает в названия рисунков в списке рисунков]{Очень
    длинная подпись к второму изображению, на~котором представлены две
    фотографии Дональда Кнута}\label{fig:knuth_2}
\end{figure}

На рисунке~\ref{fig:knuth_2-1} показан Дональд Кнут без головного убора.
На рисунке~\ref{fig:knuth_2}\subcaptionref*{fig:knuth_2-2}
показан Дональд Кнут в головном уборе.

Возможно вставлять векторные картинки, рассчитываемые \LaTeX\ <<на~лету>>
с~их~предварительной компиляцией. Надписи в таких рисунках будут выполнены
тем же~шрифтом, который указан для документа в целом.
На~рисунке~\ref{fig:tikz_example} на~странице~\pageref{fig:tikz_example}
представлен пример схемы, рассчитываемой пакетом \verb|tikz| <<на~лету>>.
Для ускорения компиляции, подобные рисунки могут быть <<кешированы>>, что
определяется настройками в~\verb|common/setup.tex|.
Причём имя предкомпилированного
файла и~папка расположения таких файлов могут быть отдельно заданы,
что удобно, если не~для подготовки диссертации,
то~для подготовки научных публикаций.

Множество программ имеют либо встроенную возможность экспортировать векторную
графику кодом \verb|tikz|, либо соответствующий пакет расширения.
Например, в GeoGebra есть встроенный экспорт,
для Inkscape есть пакет svg2tikz,
для Python есть пакет matplotlib2tikz,
для R есть пакет tikzdevice.

\section{Пример вёрстки списков}\label{sec:ch2/sec3}

\noindent Нумерованный список:
\begin{enumerate}
  \item Первый пункт.
  \item Второй пункт.
  \item Третий пункт.
\end{enumerate}

\noindent Маркированный список:
\begin{itemize}
  \item Первый пункт.
  \item Второй пункт.
  \item Третий пункт.
\end{itemize}

\noindent Вложенные списки:
\begin{itemize}
  \item Имеется маркированный список.
  \begin{enumerate}
    \item В нём лежит нумерованный список,
    \item в котором
    \begin{itemize}
      \item лежит ещё один маркированный список.
    \end{itemize}
  \end{enumerate}
\end{itemize}

\noindent Нумерованные вложенные списки:
\begin{enumerate}
  \item Первый пункт.
  \item Второй пункт.
  \item Вообще, по ГОСТ 2.105 первый уровень нумерации
  (при необходимости ссылки в тексте документа на одно из перечислений)
  идёт буквами русского или латинского алфавитов,
  а второй "--- цифрами со~скобками.
  Здесь отходим от ГОСТ.
    \begin{enumerate}
      \item в нём лежит нумерованный список,
      \item в котором
        \begin{enumerate}
          \item ещё один нумерованный список,
          \item третий уровень нумерации не нормирован ГОСТ 2.105;
          \item обращаем внимание на строчность букв,
          \item в этом списке
          \begin{itemize}
            \item лежит ещё один маркированный список.
          \end{itemize}
        \end{enumerate}

    \end{enumerate}

  \item Четвёртый пункт.
\end{enumerate}

\section{Традиции русского набора}

Много полезных советов приведено в материале
<<\href{http://www.dropbox.com/s/x4hajy4pkw3wdql/wholesome-typesetting.pdf?dl=1\&pv=1}{Краткий курс благородного набора}>> (автор А.\:В.~Костырка).
Далее мы коснёмся лишь некоторых наиболее распространённых особенностей.

\subsection{Пробелы}

В~русском наборе принято:
\begin{itemize}
    \item единицы измерения, знак процента отделять пробелами от~числа:
        10~кВт, 15~\% (согласно ГОСТ 8.417, раздел 8);
    \item \(\tg 20\text{\textdegree}\), но: 20~{\textdegree}C
        (согласно ГОСТ 8.417, раздел 8);
    \item знак номера, параграфа отделять от~числа: №~5, \S~8;
    \item стандартные сокращения: т.\:е., и~т.\:д., и~т.\:п.;
    \item неразрывные пробелы в~предложениях.
\end{itemize}

\subsection{Математические знаки и символы}

Русская традиция начертания греческих букв и некоторых математических
функций отличается от~западной. Это исправляется серией
\verb|\renewcommand|.
\begin{itemize}
%Все \original... команды заранее, ради этого примера, определены в Dissertation\userstyles.tex
    \item[До:] \( \originalepsilon \originalge \originalphi\),
    \(\originalphi \originalleq \originalepsilon\),
    \(\originalkappa \in \originalemptyset\),
    \(\originaltan\),
    \(\originalcot\),
    \(\originalcsc\).
    \item[После:] \( \epsilon \ge \phi\),
    \(\phi \leq \epsilon\),
    \(\kappa \in \emptyset\),
    \(\tan\),
    \(\cot\),
    \(\csc\).
\end{itemize}

Кроме того, принято набирать греческие буквы вертикальными, что
решается подключением пакета \verb|upgreek| (см. закомментированный
блок в~\verb|userpackages.tex|) и~аналогичным переопределением в
преамбуле (см.~закомментированный блок в~\verb|userstyles.tex|). В
этом шаблоне такие переопределения уже включены.

Знаки математических операций принято переносить. Пример переноса
в~формуле~\eqref{eq:equation3}.

\subsection{Кавычки}
В английском языке приняты одинарные и двойные кавычки в~виде ‘...’ и~“...”.
В России приняты французские («...») и~немецкие („...“) кавычки (они называются
«ёлочки» и~«лапки», соответственно). ,,Лапки`` обычно используются внутри
<<ёлочек>>, например, <<... наш гордый ,,Варяг``...>>.

Французкие левые и правые кавычки набираются
как лигатуры \verb|<<| и~\verb|>>|, а~немецкие левые
и правые кавычки набираются как лигатуры \verb|,,| и~\verb|‘‘| (\verb|``|).

Вместо лигатур или команд с~активным символом "\ можно использовать команды
\verb|\glqq| и \verb|\grqq| для набора немецких кавычек и команды \verb|\flqq|
и~\verb|\frqq| для набора французских кавычек. Они определены в пакете
\verb|babel|.

\subsection{Тире}
%  babel+pdflatex по умолчанию, в polyglossia надо включать опцией (и перекомпилировать с удалением временных файлов)
Команда \verb|"---| используется для печати тире в тексте. Оно несколько короче
английского длинного тире. Кроме того, команда задаёт небольшую жёсткую отбивку
от слова, стоящего перед тире. При этом, само тире не~отрывается от~слова.
После тире следует такая же отбивка от текста, как и~перед тире. При наборе
текста между словом и командой, за которым она следует, должен стоять пробел.

В составных словах, таких, как <<Закон Менделеева"--~Клапейрона>>, для печати
тире надо использовать команду \verb|"--~|. Она ставит более короткое,
по~сравнению с~английским, тире и позволяет делать переносы во втором слове.
При~наборе текста команда \verb|"--~| не отделяется пробелом от слова,
за~которым она следует (\verb|Менделеева"--~|). Следующее за командой слово
может быть  отделено от~неё пробелом или перенесено на другую строку.

Если прямая речь начинается с~абзаца, то перед началом её печатается тире
командой \verb|"--*|. Она печатает русское тире и жёсткую отбивку нужной
величины перед текстом.

\subsection{Дефисы и переносы слов}
%  babel+pdflatex по умолчанию, в polyglossia надо включать опцией (и перекомпилировать с удалением временных файлов)
Для печати дефиса в~составных словах введены две команды. Команда~\verb|"~|
печатает дефис и~запрещает делать переносы в~самих словах, а~команда \verb|"=|
печатает дефис, оставляя \TeX ’у право делать переносы в~самих словах.

В отличие от команды \verb|\-|, команда \verb|"-| задаёт место в~слове, где
можно делать перенос, не~запрещая переносы и~в~других местах слова.

Команда \verb|""| задаёт место в~слове, где можно делать перенос, причём дефис
при~переносе в~этом месте не~ставится.

Команда \verb|",| вставляет небольшой пробел после инициалов с~правом переноса
в~фамилии.

\section{Текст из панграмм и формул}

Любя, съешь щипцы, "--- вздохнёт мэр, "--- кайф жгуч. Шеф взъярён тчк щипцы
с~эхом гудбай Жюль. Эй, жлоб! Где туз? Прячь юных съёмщиц в~шкаф. Экс-граф?
Плюш изъят. Бьём чуждый цен хвощ! Эх, чужак! Общий съём цен шляп (юфть) "---
вдрызг! Любя, съешь щипцы, "--- вздохнёт мэр, "--- кайф жгуч. Шеф взъярён тчк
щипцы с~эхом гудбай Жюль. Эй, жлоб! Где туз? Прячь юных съёмщиц в~шкаф.
Экс-граф? Плюш изъят. Бьём чуждый цен хвощ! Эх, чужак! Общий съём цен шляп
(юфть) "--- вдрызг! Любя, съешь щипцы, "--- вздохнёт мэр, "--- кайф жгуч. Шеф
взъярён тчк щипцы с~эхом гудбай Жюль. Эй, жлоб! Где туз? Прячь юных съёмщиц
в~шкаф. Экс-граф? Плюш изъят. Бьём чуждый цен хвощ! Эх, чужак! Общий съём цен
шляп (юфть) "--- вдрызг! Любя, съешь щипцы, "--- вздохнёт мэр, "--- кайф жгуч.
Шеф взъярён тчк щипцы с~эхом гудбай Жюль. Эй, жлоб! Где туз? Прячь юных съёмщиц
в~шкаф. Экс-граф? Плюш изъят. Бьём чуждый цен хвощ! Эх, чужак! Общий съём цен
шляп (юфть) "--- вдрызг! Любя, съешь щипцы, "--- вздохнёт мэр, "--- кайф жгуч.
Шеф взъярён тчк щипцы с~эхом гудбай Жюль. Эй, жлоб! Где туз? Прячь юных съёмщиц
в~шкаф. Экс-граф? Плюш изъят. Бьём чуждый цен хвощ! Эх, чужак! Общий съём цен
шляп (юфть) "--- вдрызг! Любя, съешь щипцы, "--- вздохнёт мэр, "--- кайф жгуч.
Шеф взъярён тчк щипцы с~эхом гудбай Жюль. Эй, жлоб! Где туз? Прячь юных съёмщиц
в~шкаф. Экс-граф? Плюш изъят. Бьём чуждый цен хвощ! Эх, чужак! Общий съём цен
шляп (юфть) "--- вдрызг! Любя, съешь щипцы, "--- вздохнёт мэр, "--- кайф жгуч.
Шеф взъярён тчк щипцы с~эхом гудбай Жюль. Эй, жлоб! Где туз? Прячь юных съёмщиц
в~шкаф. Экс-граф? Плюш изъят. Бьём чуждый цен хвощ! Эх, чужак! Общий съём цен
шляп (юфть) "--- вдрызг! Любя, съешь щипцы, "--- вздохнёт мэр, "--- кайф жгуч.
Шеф взъярён тчк щипцы с~эхом гудбай Жюль. Эй, жлоб! Где туз? Прячь юных съёмщиц
в~шкаф. Экс-граф? Плюш изъят. Бьём чуждый цен хвощ! Эх, чужак! Общий съём цен
шляп (юфть) "--- вдрызг! Любя, съешь щипцы, "--- вздохнёт мэр, "--- кайф жгуч.
Шеф взъярён тчк щипцы с~эхом гудбай Жюль. Эй, жлоб! Где туз? Прячь юных съёмщиц
в~шкаф. Экс-граф? Плюш изъят. Бьём чуждый цен хвощ! Эх, чужак! Общий съём цен
шляп (юфть) "--- вдрызг! Любя, съешь щипцы, "--- вздохнёт мэр, "--- кайф жгуч.
Шеф взъярён тчк щипцы с~эхом гудбай Жюль. Эй, жлоб! Где туз? Прячь юных съёмщиц
в~шкаф. Экс-граф? Плюш изъят. Бьём чуждый цен хвощ! Эх, чужак! Общий съём цен
шляп (юфть) "--- вдрызг! Любя, съешь щипцы, "--- вздохнёт мэр, "--- кайф жгуч.
Шеф взъярён тчк щипцы с~эхом гудбай Жюль. Эй, жлоб! Где туз? Прячь юных съёмщиц
в~шкаф. Экс-граф? Плюш изъят. Бьём чуждый цен хвощ! Эх, чужак! Общий съём цен
шляп (юфть) "--- вдрызг!Любя, съешь щипцы, "--- вздохнёт мэр, "--- кайф жгуч.
Шеф взъярён тчк щипцы с~эхом гудбай Жюль. Эй, жлоб! Где туз? Прячь юных съёмщиц
в~шкаф. Экс-граф? Плюш изъят. Бьём чуждый цен хвощ! Эх, чужак! Общий съём цен

Ку кхоро адолэжкэнс волуптариа хаж, вим граэко ыкчпэтында ты. Граэкы жэмпэр
льюкяльиюч квуй ку, аэквюы продыжщэт хаж нэ. Вим ку магна пырикульа, но квюандо
пожйдонёюм про. Квуй ат рыквюы ёнэрмйщ. Выро аккузата вим нэ.
\begin{multline*}
\mathsf{Pr}(\digamma(\tau))\propto\sum_{i=4}^{12}\left( \prod_{j=1}^i\left(
\int_0^5\digamma(\tau)e^{-\digamma(\tau)t_j}dt_j
\right)\prod_{k=i+1}^{12}\left(
\int_5^\infty\digamma(\tau)e^{-\digamma(\tau)t_k}dt_k\right)C_{12}^i
\right)\propto\\
\propto\sum_{i=4}^{12}\left( -e^{-1/2}+1\right)^i\left(
e^{-1/2}\right)^{12-i}C_{12}^i \approx 0.7605,\quad
\forall\tau\neq\overline{\tau}
\end{multline*}
Квуй ыёюз омниюм йн. Экз алёквюам кончюлату квуй, ты альяквюам ёнвидюнт пэр.
Зыд нэ коммодо пробатуж. Жят доктюж дйжпютандо ут, ку зальутанде юрбанйтаж
дёзсэнтёаш жят, вим жюмо долорэж ратионебюж эа.

Ад ентэгры корпора жплэндидэ хаж. Эжт ат факэтэ дычэрунт пэржыкюти. Нэ нам
доминг пэрчёус. Ку квюо ёужто эррэм зючкёпит. Про хабэо альбюкиюс нэ.
\[
        \begin{pmatrix}
                a_{11} & a_{12} & a_{13} \\
                a_{21} & a_{22} & a_{23}
        \end{pmatrix}
\]

\[
        \begin{vmatrix}
                a_{11} & a_{12} & a_{13} \\
                a_{21} & a_{22} & a_{23}
        \end{vmatrix}
\]

\[
        \begin{bmatrix}
                a_{11} & a_{12} & a_{13} \\
                a_{21} & a_{22} & a_{23}
        \end{bmatrix}
\]
Про эа граэки квюаыквуэ дйжпютандо. Ыт вэл тебиквюэ дэфянятйоныс, нам жолюм
квюандо мандамюч эа. Эож пауло лаудым инкедыринт нэ, пэрпэтюа форынчйбюж пэр
эю. Модыратиюз дытыррюизщэт дуо ад, вирйз фэугяат дытракжйт нык ед, дуо алиё
каючаэ лыгэндоч но. Эа мольлиз юрбанйтаж зигнёфэрумквюы эжт.

Про мандамюч кончэтытюр ед. Трётанё прёнкипыз зигнёфэрумквюы вяш ан. Ат хёз
эквюедым щуавятатэ. Алёэнюм зэнтынтиаэ ад про, эа ючю мюнырэ граэки дэмокритум,
ку про чент волуптариа. Ыльит дыкоры аляквюид еюж ыт. Ку рыбюм мюндй ютенам
дуо.
\begin{align*}
        2\times 2       & = 4      & 6\times 8 & = 48 \\
        3\times 3       & = 9      & a+b       & = c  \\
        10 \times 65464 & = 654640 & 3/2       & =1,5
\end{align*}

\begin{equation}
        \begin{aligned}
                2\times 2       & = 4      & 6\times 8 & = 48 \\
                3\times 3       & = 9      & a+b       & = c  \\
                10 \times 65464 & = 654640 & 3/2       & =1,5
        \end{aligned}
\end{equation}

Пэр йн тальэ пожтэа, мыа ед попюльо дэбетиз жкрибэнтур. Йн квуй аппэтырэ
мэнандря, зыд аляквюид хабымуч корпора йн. Омниюм пэркёпитюр шэа эю, шэа
аппэтырэ аккузата рэформйданч ыт, ты ыррор вёртюты нюмквуам \(10 \times 65464 =
654640\quad  3/2=1,5\) мэя. Ипзум эуежмод \(a+b = c\) мальюизчыт ад дуо. Ад
фэюгаят пытынтёюм адвыржаряюм вяш. Модо эрепюят дэтракто ты нык, еюж мэнтётюм
пырикульа аппэльлььантюр эа.

Мэль ты дэлььынётё такематыш. Зэнтынтиаэ конклььюжионэмквуэ ан мэя. Вёжи лебыр
квюаыквуэ квуй нэ, дуо зймюл дэлььиката ку. Ыам ку алиё путынт.

%Большая фигурная скобка только справа
\[\left. %ВАЖНО: точка после слова left делает скобку неотображаемой
\begin{aligned}
	2 \times x      & = 4 \\
	3 \times y      & = 9 \\
	10 \times 65464 & = z
\end{aligned}\right\}
\]


Конвынёры витюпырата но нам, тебиквюэ мэнтётюм позтюлант ед про. Дуо эа лаудым
копиожаы, нык мовэт вэниам льебэравичсы эю, нам эпикюре дэтракто рыкючабо ыт.
Вэрйтюж аккюжамюз ты шэа, дэбетиз форынчйбюж жкряпшэрит ыт прё. Ан еюж тымпор
рыфэррэнтур, ючю дольор котёдиэквюэ йн. Зыд ипзум дытракжйт ныглэгэнтур нэ,
партым ыкжплььикари дёжжэнтиюнт ад пэр. Мэль ты кытэрож молыжтйаы, нам но ыррор
жкрипта аппарэат.

\[ \frac{m_{t\vphantom{y}}^2}{L_t^2} = \frac{m_{x\vphantom{y}}^2}{L_x^2} +
\frac{m_y^2}{L_y^2} + \frac{m_{z\vphantom{y}}^2}{L_z^2} \]

Вэре льаборэж тебиквюэ хаж ут. Ан пауло торквюатоз хаж, нэ пробо фэугяат
такематыш шэа. Мэльёуз пэртинакёа юлламкорпэр прё ад, но мыа рыквюы конкыптам.
Хёз квюот пэртинакёа эи, ельлюд трактатоз пэр ад. Зыд ед анёмал льаборэж
номинави, жят ад конгуы льабятюр. Льаборэ тамквюам векж йн, пэр нэ дёко диам
шапэрэт, экз вяш тебиквюэ элььэефэнд мэдиокретатым.

Нэ про натюм фюйзчыт квюальизквюэ, аэквюы жкаывола мэль ку. Ад граэкйж
плььатонэм адвыржаряюм квуй, вим емпыдит коммюны ат, ат шэа одео квюаырэндум.
Вёртюты ажжынтиор эффикеэнди эож нэ, доминг лаборамюз эи ыам. Чэнзэрет
мныжаркхюм экз эож, ыльит тамквюам факильизиж нык эи. Квуй ан элыктрам
тинкидюнт ентырпрытаряш. Йн янвыняры трактатоз зэнтынтиаэ зыд. Дюиж зальютатуж
ыам но, про ыт анёмал мныжаркхюм, эи ыюм пондэрюм майыжтатйж.
