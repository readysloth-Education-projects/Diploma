\chapter{Конструкторский раздел}\label{ch:ch2}
\section{Обоснование выбора языка программирования и среды разработки}\label{sec:ch2/sec1}
Для удобной, быстрой и эффективной, как по срокам выполнения, так и по качеству конечного продукта,
разработки {\ProgModule} потребуются правильные инструменты -- язык программирования, на котором 
легче всего описать решение данной задачи и среда разработки, не только поддерживающая данный язык,
но и позволяющая эффективно с ним работать.

\subsection{Сравнение языков программирования}\label{sec:ch2/sec1/sub1}
Для разработки {\ProgModule} понадобится сверхвысокоуровневый язык с кросс-платформенной
стандартной библиотекой, который позволит точно и лаконично описать этапы анализа,
а так же имеющий высокую скорость исполнения, для анализа больших объемов исходного кода и
исполняемых файлов.

\begin{table}
    {\small
        \setlength{\tabcolsep}{2pt}
        \caption{\label{table:languages-comparsion}
               Сравнительная таблица языков программирования}
        \begin{longtable}{*{5}{| c}|}
            \hline
            \diagbox[width=8cm]{Свойства}{Язык программирования} &
                \makecell{Nim \autocite{nim}} &
                \makecell{Python \autocite{python}} &
                \makecell{Perl \autocite{perl}} &
                \makecell{C/C++} \\
            \hline
                \makecell{Сверхвысокоуровневость} & 
                \greencell{Да} & 
                \greencell{Да} &
                \greencell{Да} &
                \redcell{Нет} \\
            \hline
                \makecell{Компилируется в\\машинный код} & 
                \greencell{Да} & 
                \redcell{Нет} &
                \redcell{Нет} &
                \greencell{Да} \\
            \hline
                \makecell{Количество функции в\\стандартной библиотеке} & 
                5585 & 
                638 &
                1338 &
                1224 \\
            \hline
                \makecell{Портируемость} & 
                \greencell{Есть} & 
                \greencell{Есть} &
                \greencell{Есть} &
                \yellowcell{\makecell{Есть,\\но неудобная}}\\
            \hline
                \makecell{Встроенная\\генерация документации} & 
                \greencell{Есть} & 
                \greencell{Есть} &
                \greencell{Есть} &
                \redcell{Нет}\\
            \hline
                \makecell{Статическая типизация} & 
                \greencell{Есть} & 
                \redcell{Нет} &
                \redcell{Нет} &
                \greencell{Есть}\\
            \hline
                \makecell{Автоматическое\\управление памятью} & 
                \greencell{Есть} & 
                \greencell{Есть} &
                \greencell{Есть} &
                \greencell{Есть} \\
            \hline
                \makecell{Обобщенное программирование} & 
                \greencell{Есть} & 
                \greencell{Есть} &
                \greencell{Есть} &
                \greencell{Есть} \\
            \hline
                \makecell{Мета-программирование} & 
                \greencell{Есть} & 
                \greencell{Есть} &
                \greencell{Есть} &
                \greencell{Есть} \\
            \hline
                \makecell{Опыт использования} & 
                \greencell{Есть} & 
                \greencell{Есть} &
                \redcell{Нет} &
                \greencell{Есть} \\
            \hline
        \end{longtable}
    }
\end{table}

Рассмотрим подробно каждый из представленных в таблице языков.

\subsubsection{C++}\label{sec:ch2/sec1/sub1/sub1}
Мультипарадигменный высокоуровневый язык программирования,
разработанный в 1983 году Бьёрном Страуструпом. Является практически
полным надмножеством языка C. Статически типизирован.\\
Отличается высокой производительностью и неплохой гибкостью при написании кода.
К минусам языка можно отнести сложность освоения и перегруженность 
<<наследием>> 80-х годов прошлого века, а так же низкую скорость компиляции,
по сравнению с предшественником -- C.\\
Портируемость языка на различные платформы обеспечивается пере- или
кросс-компиляцией исходного кода под нужную платформу.


\subsubsection{Python}\label{sec:ch2/sec1/sub1/sub2}
Мультипарадигменный сверхвысокоуровневый язык программирования,
разработанный в 1991 году Гвидо Ван Россумом.
Является интерпретируемым языком, имеет слабую динамическую типизацию,
что позволяет легко писать обобщенный код и использовать мета-программирование,
но так же ведет к трудноулавливаемым ошибкам. Негативное влияние можно сгладить
с помощью указания типов при объявлении перемнных и аргументов функций, а так же 
программы, проверяющей эти типы -- линтера. Например pylint \autocite{pylint} или
pyflakes \autocite{pyflakes}.\\
Благодаря своей популярности, python так же портирован на большое количество платформ.
Большим плюсом языка является его обширная стандартная библиотека, позволяющая легко
писать комплексные приложения, не прибегая к установке дополнительных библиотек --
такие программы, как и сам python, следуют философии <<в комплекте с батарейками>>
(<<batteries included>> \autocite{batteries-included}), суть которой заключается в 
самодостаточности программ. Помимо этого вместе с python поставляется менеджер
пакетов pip \autocite{pip}, позволяющий удобно устанавливать требуемые библиотечные модули вместе
с зависимостями.\\
К минусам языка можно отнести медлительность эталонного интерпретатора языка -- cpython \autocite{cpython}.
Код, исполняемый им, в определенных задачах медленнее кода на C в сотни раз. Не смотря на то, что
есть более быстрые интерпретаторы: PyPy \autocite{pypy}, Jython \autocite{jython}, Iron Python \autocite{iron-python},
они не смогут достичь скорости исполнения программ, компилируемых в машинный код.\\
На данный момент существует две, между собой несовместимые, версии языка: 
python 2, поддержка которого закончилась \DTMdate{2020-01-01} и python 3.

\subsubsection{Perl}\label{sec:ch2/sec1/sub1/sub2}
Мультипарадигменный сверхвысокоуровневый 
язык программирования, разработанный в 1987 году Ларри Уоллом.
Является интерпретируемым языком, имеет слабую динамическую типизацию.\\
Полное название языка -- <<Practical Extraction and Report Language>> 
(<<Практический Язык для Извлечения Данных и Составления Отчётов>>), отражает его суть:
в языке реализованы обширные возможности для работы с текстом, в синтаксис интегрированы 
регулярные выражения, как и в языках, которые оказали на него наибольшее влияние --
AWK \autocite{awk} и sed \autocite{sed}. Но это же и я является его слабой частью, так как
Perl скорее предназначен для однострочных команд в терминале, как AWK и
sed.
        
\subsubsection{Nim}\label{sec:ch2/sec1/sub1/sub3}
Мультипарадигменный сверхвысокоуровневый 
язык программирования, разработанный в 2004 году Андреасом Румпфом.
Является компилируемым языком, имеет строгую статическую типизацию.\\
Заметно, что на синтаксис языка повлиял Python, что сделало его
выразительным и понятным. Язык использует промежуточную компиляцию, которая несколько
замедляет процесс компиляции программ, но позволяет запускать nim-программы на различных
платформах. На данный момент поддерживается компиляция в JavaScript \autocite{javascript}
и оптимизированный C-код с несколькими моделями управления памятью:\\
\begin{itemize}
    \item Сборщики мусора, основанные на:
        \begin{enumerate}[label={\arabic*)}]
            \item подсчете ссылок;
            \item подсчете ссылок с оптимизацией move-семантикой \autocite{nim-gc-move};
            \item boehm \autocite{boehm-gc};
            \item go \autocite{go-gc};
        \end{enumerate}
    \item ручном освобождении памяти;
    \item модель, в которой вся выделенная память высвобождается только по завершению программы
        (не рекомендуется к использованию).
\end{itemize}
Компиляции Nim в C означает не только высокую скорость работы, но и прозрачный программный интерфейс при взаимодействии с
C библиотеками. Это значит, что можно писать Nim-код, взаимодействующий с С библиотекой так же, как
если бы это была Nim-библиотека, в отличие от, например, Python.\\
Так же вместе с компилятором языка поставляется пакетный менеджер nimble \autocite{nimble} и генератор
документации из комментариев, написанных на reStructuredText \autocite{restructuredtext}.

\subsubsection{Вывод}\label{sec:ch2/sec1/sub1/sub4}
Из всего вышесказанного следует, что для {\ProgModule} лучше всего подойдет язык Nim
благодаря его скорости, выразительности и портируемости на различные платформы.
Кроме того, для подготовки динамического анализа программы будут использованы утилиты, умеющие разбирать
заголовки исполняемого файла, а именно objdump и readelf. Форматирование входных данных для данных утилит
будет осуществляться с помощью Bash-скриптов. Не смотря на то, что данные программы имеются только на UNIX системах,
есть возможность использовать их и в операционной системе Windows, через Cygwin \autocite{cygwin}.

\subsection{Сравнение сред разработки}\label{sec:ch2/sec1/sub2}

Для разработки на Nim существует несколько IDE и огромное количество
текстовых редакторов, часть которых рассмотрим ниже:

\begin{table}[!htbp]
    {\small
        \setlength{\tabcolsep}{2pt}
        \caption{\label{table:ide-comparsion}
               Сравнительная таблица IDE и редакторов кода}
        \begin{longtable}{*{6}{| c}|}
            \hline
            \diagbox[width=8cm]{Свойства}{IDE/Редактор} &
                \makecell{Aporia \autocite{aporia-ide}} &
                \makecell{Atom \autocite{atom-ide}} &
                \makecell{Sublime\\Text \autocite{sublime-ide}} &
                \makecell{Visual\\Studio\\Code \autocite{vs-code-ide}} &
                \makecell{Vim \autocite{vim-ide}} \\
            \hline
                \makecell{Поддержка плагинов} & 
                \redcell{Нет} &
                \greencell{Да} & 
                \greencell{Да} &
                \greencell{Да} &
                \greencell{Да} \\
            \hline
                \makecell{Требователен к ресурсам} & 
                \greencell{Нет} & 
                \redcell{Да} & 
                \greencell{Нет} & 
                \redcell{Да} & 
                \greencell{Нет} \\ 
            \hline
                \makecell{Имеет продвинутую систему\\редактирования текста} & 
                \redcell{Нет} &
                \redcell{Нет} &
                \redcell{Нет} &
                \redcell{Нет} &
                \greencell{Да} \\
            \hline
                \makecell{Кросс-платформенность} & 
                \greencell{Есть} & 
                \greencell{Есть} &
                \greencell{Есть} &
                \greencell{Есть} &
                \greencell{Есть} \\
            \hline
                \makecell{Может работать\\без GUI} & 
                \redcell{Нет} &
                \redcell{Нет} &
                \redcell{Нет} &
                \redcell{Нет} &
                \greencell{Да} \\
            \hline
                \makecell{Восстановление после сбоев} & 
                \redcell{Нет} & 
                \greencell{Есть} &
                \greencell{Есть} &
                \greencell{Есть} &
                \greencell{Есть} \\
            \hline
                \makecell{Возможность выделять\\ключевые слова с помощью\\регулярных выражений} & 
                \redcell{Нет} & 
                \greencell{Есть} &
                \greencell{Есть} &
                \greencell{Есть} &
                \greencell{Есть} \\
            \hline
                \makecell{Опыт использования} & 
                \redcell{Нет} &
                \redcell{Нет} &
                \greencell{Есть} &
                \greencell{Есть} &
                \greencell{Есть} \\
            \hline
        \end{longtable}
    }
\end{table}

Рассмотрим подробно каждый из представленных в таблице редакторов.

\subsubsection{Aporia}\label{sec:ch2/sec1/sub2/sub1}
Простая IDE, написанная на nim, с использованием GTK2.
В настоящее время не поддерживается, так как большая часть Nim-программистов
перешла на Visual Studio Code.

\subsubsection{Atom}\label{sec:ch2/sec1/sub2/sub2}
Редактор с открытым исходным кодом от GitHub Inc., написан с использованием Electron \autocite{electron} -- фреймворка
для разработки кросс-платформенных приложений с помощью HTML, JavaScript и CSS. Из-за архитектурных
и технологических решений, все программы, написанные на данном фреймворке, будут очень требовательны
к ресурсам.

\subsubsection{Sublime Text}\label{sec:ch2/sec1/sub2/sub3}
Проприетарный текстовый редактор написан на C++ и python, возможности которого могут быть расширены
с помощью плагинов на python.

\subsubsection{Visual Studio Code}\label{sec:ch2/sec1/sub2/sub4}
Редактор с открытым исходным кодом от Microsoft. Так же, как и Atom, написан с использованием Electron.
Имеет встроенный <<магазин>> плагинов.

\subsubsection{Vim}\label{sec:ch2/sec1/sub2/sub5}
Текстовый редактор с открытым исходным кодом и большими возможностями к
быстрому редактированию текстов. Является наследником редактора vi, который, в свою
очередь, создавался с оглядкой на редактор ed. Управление делится на
режим ввода и режим команд, благодаря чему есть возможность управлять 
редактором только с помощью клавиатуры, что, при должном умении, повышает скорость
не только из-за отсутствия необходимости в использовании компьютерной мыши, но и
более коротким сочетаниям  <<горячих клавиш>>. Легко поддается модифицированию с помощью плагинов.
Есть под множество платформ.


\subsubsection{Вывод}\label{sec:ch2/sec1/sub1/sub4}
Из всего вышесказанного и личного опыта следует, 
что для разработки {\ProgModule} лучше всего подойдет текстовый редактор Vim,
так как он поддерживает добавление плагинов, не требователен к ресурсам и позволяет
очень быстро редактировать текст. В качестве расширения его функциональности использованы
плагины:
\begin{enumerate}[label={\arabic*)}]
    \item NERDTree \autocite{nerdtree} -- улучшает просмотр каталогов;
    \item Tabular \autocite{tabular} -- позволяет быстро выравнивать текст
        для улучшения читаемости;
    \item vim-polyglot \autocite{vim-polyglot} -- подсветка синтаксиса большого
        числа языков;
    \item undotree \autocite{undotree} -- просмотр истории изменений в виде дерева;
    \item rainbow \autocite{rainbow} -- подсветка вложенных скобок разными цветами,
        для улучшения читаемости.
\end{enumerate}


\begin{figure}[!htbp]
    \centerfloat{
        \includegraphics[width=\linewidth]{images/Vim.png}
    }
    \caption{Интерфейс Vim {\ProgModule}\label{fig:vim-tui}}
\end{figure}

\section{Архитектура {\ProgModule}}\label{sec:ch2/sec2}
Архитектура программного обеспечения это система, объединяющая внутрение
компоненты, их связи между собой и с окружением, а так же принципы,
использующиеся при проектировании и эволюции программы \autocite{software-architecture}.

При проектировании {\ProgModule} была выбрана UNIX-философия \autocite{unix-philosophy},
заключающаяся в следующих основопологающих принципах:
\begin{itemize}
    \item создавать маленькие программы;
    \item программы делают одно дело, но делают его хорошо;
    \item хранить данные в текстовом, читаемом для людей формате.
\end{itemize}

Поэтому было принято решение разрабатывать под каждую подзадачу проведения сертификации ПО
самостоятельную программу, которая была бы маленькой и хорошо бы справлялась со своим назначением.

\subsection{Организация передачи информации между компонентами {\ProgModule}}\label{sec:ch2/sec2/sub1}
Передача информации между компонентами {\ProgModule} осуществляется посредством
сериализации внутренних структур (\autoref{fig:dynamic-json} и \autoref{fig:static-json})
конкретного модуля в формате JSON. JSON удобен тем, что является простым для 
чтения как человеком, так и компьютером, что позволяет оператору 
анализировать так же и промежуточные результаты работы, для
вынесения вердикта.

\subsubsection{Виды сериализуемых данных}\label{sec:ch2/sec2/sub1/sub1}
В {\ProgModule} сериализуются данные после прохождения этапа:
\begin{itemize}
    \item статического анализа исходных кодов;
    \item динамического анализа сертифицируемой программы.
\end{itemize}

\begin{figure}[!htbp]
    \centerfloat{
        % !TEX encoding = UTF-8 Unicode
% Úτƒ-8 encoded
% http://www.linux.org.ru/forum/general/10357036
\tikzset{
    line/.style={draw, -latex'},
    every join/.style={line},
    u/.style={anchor=south},
    r/.style={anchor=west},
    fxd/.style={text width = 6em},
    it/.style={font={\small\itshape}},
    bf/.style={font={\small\bfseries}},
}
\tikzstyle{base} =
    [
        draw,
        on chain,
        on grid,
        align=center,
        minimum height=4ex,
        minimum width = 10ex,
        node distance = 6mm and 60mm,
        text badly centered,
    ]
\tikzstyle{coord} =
    [
        coordinate,
        on chain,
        on grid
    ]
\tikzstyle{cloud} =
    [
        base,
        ellipse,
        node distance = 3cm,
        minimum height = 2em,
        text width=2cm
    ]
\tikzstyle{decision} =
    [
        base,
        diamond,
        aspect=2,
        node distance = 2cm,
        inner sep = 0pt
    ]
\tikzstyle{block} =
    [
        rectangle,
        base,
        rounded corners,
        minimum height = 2em
    ]
\tikzstyle{print_block} =
    [
        base,
        tape,
        tape bend top=none,
    ]
\tikzstyle{io} =
    [
        base,
        trapezium,
        trapezium left angle = 70,
        trapezium right angle = 110,
    ]
\tikzstyle{prompt} =
    [
        base,
        trapezium,
        trapezium left angle = 90,
        trapezium right angle = 80,
        shape border rotate = 90
    ]
\tikzstyle{disk file} =
    [
        base,
        cylinder,
        aspect=0.2,
    ]
\tikzstyle{process} =
    [
        rectangle,
        base,
    ]
\makeatletter
\pgfkeys{/pgf/.cd,
    subrtshape w/.initial=2mm,
    cycleshape w/.initial=2mm
}
\pgfdeclareshape{parallelshape}{
    \inheritsavedanchors[from=rectangle]
    \inheritanchorborder[from=rectangle]
    \inheritanchor[from=rectangle]{north}
    \inheritanchor[from=rectangle]{center}
    \inheritanchor[from=rectangle]{west}
    \inheritanchor[from=rectangle]{east}
    \inheritanchor[from=rectangle]{mid}
    \inheritanchor[from=rectangle]{base}
    \inheritanchor[from=rectangle]{south}
    \backgroundpath{
        \southwest \pgf@xa=\pgf@x \pgf@ya=\pgf@y
        \northeast \pgf@xb=\pgf@x \pgf@yb=\pgf@y
        \def\ppd@offset{\pgfpoint{\pgfutil@tempdima}{0ex}}
        \def\ppd@offsetm{\pgfpoint{-\pgfutil@tempdima}{0ex}}
        \pgfpathmoveto{\pgfqpoint{\pgf@xa}{\pgf@ya}}
            \pgfpathlineto{\pgfqpoint{\pgf@xb}{\pgf@ya}}
        \pgfpathclose
        \pgfpathmoveto{\pgfqpoint{\pgf@xb}{\pgf@yb}}
            \pgfpathlineto{\pgfqpoint{\pgf@xa}{\pgf@yb}}
        \pgfpathclose
    }
}
\pgfdeclareshape{subrtshape}{
    \inheritsavedanchors[from=rectangle]
    \inheritanchorborder[from=rectangle]
    \inheritanchor[from=rectangle]{north}
    \inheritanchor[from=rectangle]{center}
    \inheritanchor[from=rectangle]{west}
    \inheritanchor[from=rectangle]{east}
    \inheritanchor[from=rectangle]{mid}
    \inheritanchor[from=rectangle]{base}
    \inheritanchor[from=rectangle]{south}
    \backgroundpath{
        \southwest \pgf@xa=\pgf@x \pgf@ya=\pgf@y
        \northeast \pgf@xb=\pgf@x \pgf@yb=\pgf@y
        \pgfmathsetlength\pgfutil@tempdima{\pgfkeysvalueof{/pgf/subrtshape w}}
        \def\ppd@offset{\pgfpoint{\pgfutil@tempdima}{0ex}}
        \def\ppd@offsetm{\pgfpoint{-\pgfutil@tempdima}{0ex}}
        \pgfpathmoveto{\pgfqpoint{\pgf@xa}{\pgf@ya}}
        \pgfpathlineto{\pgfqpoint{\pgf@xb}{\pgf@ya}}
        \pgfpathlineto{\pgfqpoint{\pgf@xb}{\pgf@yb}}
        \pgfpathlineto{\pgfqpoint{\pgf@xa}{\pgf@yb}}
        \pgfpathclose
        \pgfpathmoveto{\pgfpointadd{\pgfpoint{\pgf@xa}{\pgf@yb}}{\ppd@offsetm}}
        \pgfpathlineto{\pgfpointadd{\pgfpoint{\pgf@xa}{\pgf@ya}}{\ppd@offsetm}}
        \pgfpathlineto{\pgfpointadd{\pgfpoint{\pgf@xb}{\pgf@ya}}{\ppd@offset}}
        \pgfpathlineto{\pgfpointadd{\pgfpoint{\pgf@xb}{\pgf@yb}}{\ppd@offset}}
        \pgfpathclose
    }
}
\pgfdeclareshape{cyclebegshape}{
    \inheritsavedanchors[from=rectangle]
    \inheritanchorborder[from=rectangle]
    \inheritanchor[from=rectangle]{north}
    \inheritanchor[from=rectangle]{center}
    \inheritanchor[from=rectangle]{west}
    \inheritanchor[from=rectangle]{east}
    \inheritanchor[from=rectangle]{mid}
    \inheritanchor[from=rectangle]{base}
    \inheritanchor[from=rectangle]{south}
    \backgroundpath{
        \southwest \pgf@xa=\pgf@x \pgf@ya=\pgf@y
        \northeast \pgf@xb=\pgf@x \pgf@yb=\pgf@y
        \pgfmathsetlength\pgfutil@tempdima{\pgfkeysvalueof{/pgf/cycleshape w}}
        \pgfpathmoveto{\pgfqpoint{\pgf@xa}{\pgf@ya}}
\pgfpathlineto{\pgfpointadd{\pgfpoint{\pgf@xa}{\pgf@yb}}{\pgfpoint{0ex}{-\pgfutil@tempdima}}}
\pgfpathlineto{\pgfpointadd{\pgfpoint{\pgf@xa}{\pgf@yb}}{\pgfpoint{\pgfutil@tempdima}{0ex}}}
\pgfpathlineto{\pgfpointadd{\pgfpoint{\pgf@xb}{\pgf@yb}}{\pgfpoint{-\pgfutil@tempdima}{0ex}}}
\pgfpathlineto{\pgfpointadd{\pgfpoint{\pgf@xb}{\pgf@yb}}{\pgfpoint{0ex}{-\pgfutil@tempdima}}}
\pgfpathlineto{\pgfqpoint{\pgf@xb}{\pgf@ya}}
        \pgfpathclose
    }
}
\pgfdeclareshape{cycleendshape}{
    \inheritsavedanchors[from=rectangle]
    \inheritanchorborder[from=rectangle]
    \inheritanchor[from=rectangle]{north}
    \inheritanchor[from=rectangle]{center}
    \inheritanchor[from=rectangle]{west}
    \inheritanchor[from=rectangle]{east}
    \inheritanchor[from=rectangle]{mid}
    \inheritanchor[from=rectangle]{base}
    \inheritanchor[from=rectangle]{south}
    \backgroundpath{
        \southwest \pgf@xa=\pgf@x \pgf@ya=\pgf@y
        \northeast \pgf@xb=\pgf@x \pgf@yb=\pgf@y
        \pgfmathsetlength\pgfutil@tempdima{\pgfkeysvalueof{/pgf/cycleshape w}}
        \pgfpathmoveto{\pgfqpoint{\pgf@xb}{\pgf@yb}}
\pgfpathlineto{\pgfpointadd{\pgfpoint{\pgf@xb}{\pgf@ya}}{\pgfpoint{0ex}{\pgfutil@tempdima}}}
\pgfpathlineto{\pgfpointadd{\pgfpoint{\pgf@xb}{\pgf@ya}}{\pgfpoint{-\pgfutil@tempdima}{0ex}}}
\pgfpathlineto{\pgfpointadd{\pgfpoint{\pgf@xa}{\pgf@ya}}{\pgfpoint{\pgfutil@tempdima}{0ex}}}
\pgfpathlineto{\pgfpointadd{\pgfpoint{\pgf@xa}{\pgf@ya}}{\pgfpoint{0ex}{\pgfutil@tempdima}}}
\pgfpathlineto{\pgfqpoint{\pgf@xa}{\pgf@yb}}
        \pgfpathclose
    }
}
\makeatother
\tikzstyle{subroutine} =
    [
        base,
        subrtshape,
    ]
\tikzstyle{cyclebegin} =
    [
        base,
        cyclebegshape,
    ]
\tikzstyle{cycleend} =
    [
        base,
        cycleendshape,
    ]
\tikzstyle{connector} =
    [
        base,
        circle,
    ]

\tikzstyle{parallel} =
    [
        base,
        parallelshape,
    ]

\newcommand{\BreakpointInfo}{
\begin{tabular}{*{2}{| l}|}
BreakpointInfo     & Информация о точке останова\\
\hline
    address        &  адрес, на котором находится точка останова\\
    call\_address   &  адрес, по которому будет совершен вызов\\
    scope\_address  &  адрес функции, внутри которой происходит вызов\\
    registers      &  информация о регистрах процессора в момент останова\\
    instructions   &  восемь инструкций, следующих после команды вызова \\
    backtrace      &  стек вызовов\\
    stack          &  hex-dump стека программы\\
\end{tabular}
}
\newcommand{\SegmentInfo}{
\begin{tabular}{*{2}{| l}|}
SegmentInfo            & Информация о сегментах программы\\
\hline
    name               &  название сегмента\\
    occupied\_mem\_begin &  адрес памяти, с которого начинается сегмент\\
    occupied\_mem\_end   &  адрес памяти, которым заканчивается сегмент\\
\end{tabular}
}

\newcommand{\FunctionInfo}{
\begin{tabular}{*{2}{| l}|}
FunctionInfo & Информация об определенных функциях\\
\hline
    name     &  название функции\\
    address  &  адрес \\
\end{tabular}
}

\newcommand{\ProcessStartInfo}{
\begin{tabular}{*{2}{| l}|}
ProcessStartInfo & Информация о запуске исследуемой программы\\
\hline
    cmdline      &  переданные аргументы\\
    cwd          &  текущая рабочая директория\\
    exe          &  путь к исполняемому файлу\\
\end{tabular}
}

\newcommand{\ProcessSegmentsInfo}{
\begin{tabular}{*{2}{| l}|}
ProcessSegmentsInfo & Информация о всех сегментах программы\\
\hline
    entrypoint      & точка входа в программу\\
    segments        & список информации о сегментах\\
\end{tabular}
}
\newcommand{\Process}{
\begin{tabular}{*{2}{| l}|}
Process              & Информация о исследуемой программе\\
\hline
    pid              &  PID процесса\\
    start\_info       &  информация о запуске программы\\
    segments\_info    &  информация о сегментах программы\\
    breakpoints\_info &  список информации о точках останова\\
    functions\_info   &  список информации об определенных функциях\\
\end{tabular}
}

\begin{tikzpicture}[%
    start chain=going below,    % General flow is top-to-bottom
    node distance=6mm and 30mm, % Global setup of box spacing
    ] 
        \node [process] (breakpointinfo)                                          {\small \BreakpointInfo};
        \node [process] (segmentinfo)        [below = 5cm of breakpointinfo]      {\small \SegmentInfo};
        \node [process] (functioninfo)       [below = 3cm of segmentinfo]         {\small \FunctionInfo};
        \node [process] (processstartinfo)   [below = 3cm of functioninfo]        {\small \ProcessStartInfo};
        \node [process] (processsegmentsinfo)[below = 3cm of processstartinfo]    {\small \ProcessSegmentsInfo};
        \node [process] (process)            [below = 4cm of processsegmentsinfo] {\small \Process};

        \draw [line] (process) -- +(-9,-0) |- (breakpointinfo);
        \draw [line] (process) -- +(-9,-0) |- (segmentinfo);
        \draw [line] (process) -- +(-9,-0) |- (functioninfo);
        \draw [line] (process) -- +(-9,-0) |- (processstartinfo);
        \draw [line] (process) -- +(-9,-0) |- (processsegmentsinfo);

\end{tikzpicture}

    }
    \caption{Сохраняемые структуры динамического анализа \label{fig:dynamic-json}}
\end{figure}

Структура данных помогает иерархически организовать доступ к собранной, во время
динамического анализа, информации.

Данные с расставленных точек останова, содержатся в структуре \texttt{BreakpointInfo}, 
которая заполняется непосредственно во время выполнения машинных инструкций программы, а значит важно
в них получить максимальное количество информации текущем мгновенном состоянии программы.
В структуре содержится:
\begin{itemize}
    \item адреса:
            \begin{itemize}
                \item \texttt{call}-инструкции, на которой находится точка останова;
                \item по которому собирается сделать вызов \texttt{call}-инструкция;
                \item функции, в котором находится данная \texttt{call}-инструкция;
            \end{itemize}
            Которые необходимы для последующего сравнительного анализа;
        \item регистры, в которых могут содержаться передаваемые параметры (fastcall convention \autocite{fastcall});
        \item следующие за \texttt{call} 8 инструкций, в которых может содержаться код, обрабатывающий
            возвращенное значение;
        \item стек вызовов, позволяет посмотреть ветку исполнения исследуемой программы.
\end{itemize}

Информация о сегментах в \texttt{SegmentInfo} позволяет определить, к какому сегменту относится
вызываемая, или текущая функция. Например, это может быть сегмент динамически загружаемой библиотеки.

\texttt{FunctionInfo} содержит информацию, которую предоставляет GDB при загрузке программы:
список известных функций и их адреса. 

\texttt{ProcessStartInfo} сохраняет параметры запуска, \texttt{ProcessSegmentsInfo} -- 
агрегирует информацию по всем сегментам программы.
Структура \texttt{Process} же агрегирует в себе всё вышеперечисленное.

\begin{figure}[!htbp]
    \centerfloat{
        % !TEX encoding = UTF-8 Unicode
% Úτƒ-8 encoded
% http://www.linux.org.ru/forum/general/10357036
\tikzset{
    line/.style={draw, -latex'},
    every join/.style={line},
    u/.style={anchor=south},
    r/.style={anchor=west},
    fxd/.style={text width = 6em},
    it/.style={font={\small\itshape}},
    bf/.style={font={\small\bfseries}},
}
\tikzstyle{base} =
    [
        draw,
        on chain,
        on grid,
        align=center,
        minimum height=4ex,
        minimum width = 10ex,
        node distance = 6mm and 60mm,
        text badly centered,
    ]
\tikzstyle{coord} =
    [
        coordinate,
        on chain,
        on grid
    ]
\tikzstyle{cloud} =
    [
        base,
        ellipse,
        node distance = 3cm,
        minimum height = 2em,
        text width=2cm
    ]
\tikzstyle{decision} =
    [
        base,
        diamond,
        aspect=2,
        node distance = 2cm,
        inner sep = 0pt
    ]
\tikzstyle{block} =
    [
        rectangle,
        base,
        rounded corners,
        minimum height = 2em
    ]
\tikzstyle{print_block} =
    [
        base,
        tape,
        tape bend top=none,
    ]
\tikzstyle{io} =
    [
        base,
        trapezium,
        trapezium left angle = 70,
        trapezium right angle = 110,
    ]
\tikzstyle{prompt} =
    [
        base,
        trapezium,
        trapezium left angle = 90,
        trapezium right angle = 80,
        shape border rotate = 90
    ]
\tikzstyle{disk file} =
    [
        base,
        cylinder,
        aspect=0.2,
    ]
\tikzstyle{process} =
    [
        rectangle,
        base,
    ]
\makeatletter
\pgfkeys{/pgf/.cd,
    subrtshape w/.initial=2mm,
    cycleshape w/.initial=2mm
}
\pgfdeclareshape{parallelshape}{
    \inheritsavedanchors[from=rectangle]
    \inheritanchorborder[from=rectangle]
    \inheritanchor[from=rectangle]{north}
    \inheritanchor[from=rectangle]{center}
    \inheritanchor[from=rectangle]{west}
    \inheritanchor[from=rectangle]{east}
    \inheritanchor[from=rectangle]{mid}
    \inheritanchor[from=rectangle]{base}
    \inheritanchor[from=rectangle]{south}
    \backgroundpath{
        \southwest \pgf@xa=\pgf@x \pgf@ya=\pgf@y
        \northeast \pgf@xb=\pgf@x \pgf@yb=\pgf@y
        \def\ppd@offset{\pgfpoint{\pgfutil@tempdima}{0ex}}
        \def\ppd@offsetm{\pgfpoint{-\pgfutil@tempdima}{0ex}}
        \pgfpathmoveto{\pgfqpoint{\pgf@xa}{\pgf@ya}}
            \pgfpathlineto{\pgfqpoint{\pgf@xb}{\pgf@ya}}
        \pgfpathclose
        \pgfpathmoveto{\pgfqpoint{\pgf@xb}{\pgf@yb}}
            \pgfpathlineto{\pgfqpoint{\pgf@xa}{\pgf@yb}}
        \pgfpathclose
    }
}
\pgfdeclareshape{subrtshape}{
    \inheritsavedanchors[from=rectangle]
    \inheritanchorborder[from=rectangle]
    \inheritanchor[from=rectangle]{north}
    \inheritanchor[from=rectangle]{center}
    \inheritanchor[from=rectangle]{west}
    \inheritanchor[from=rectangle]{east}
    \inheritanchor[from=rectangle]{mid}
    \inheritanchor[from=rectangle]{base}
    \inheritanchor[from=rectangle]{south}
    \backgroundpath{
        \southwest \pgf@xa=\pgf@x \pgf@ya=\pgf@y
        \northeast \pgf@xb=\pgf@x \pgf@yb=\pgf@y
        \pgfmathsetlength\pgfutil@tempdima{\pgfkeysvalueof{/pgf/subrtshape w}}
        \def\ppd@offset{\pgfpoint{\pgfutil@tempdima}{0ex}}
        \def\ppd@offsetm{\pgfpoint{-\pgfutil@tempdima}{0ex}}
        \pgfpathmoveto{\pgfqpoint{\pgf@xa}{\pgf@ya}}
        \pgfpathlineto{\pgfqpoint{\pgf@xb}{\pgf@ya}}
        \pgfpathlineto{\pgfqpoint{\pgf@xb}{\pgf@yb}}
        \pgfpathlineto{\pgfqpoint{\pgf@xa}{\pgf@yb}}
        \pgfpathclose
        \pgfpathmoveto{\pgfpointadd{\pgfpoint{\pgf@xa}{\pgf@yb}}{\ppd@offsetm}}
        \pgfpathlineto{\pgfpointadd{\pgfpoint{\pgf@xa}{\pgf@ya}}{\ppd@offsetm}}
        \pgfpathlineto{\pgfpointadd{\pgfpoint{\pgf@xb}{\pgf@ya}}{\ppd@offset}}
        \pgfpathlineto{\pgfpointadd{\pgfpoint{\pgf@xb}{\pgf@yb}}{\ppd@offset}}
        \pgfpathclose
    }
}
\pgfdeclareshape{cyclebegshape}{
    \inheritsavedanchors[from=rectangle]
    \inheritanchorborder[from=rectangle]
    \inheritanchor[from=rectangle]{north}
    \inheritanchor[from=rectangle]{center}
    \inheritanchor[from=rectangle]{west}
    \inheritanchor[from=rectangle]{east}
    \inheritanchor[from=rectangle]{mid}
    \inheritanchor[from=rectangle]{base}
    \inheritanchor[from=rectangle]{south}
    \backgroundpath{
        \southwest \pgf@xa=\pgf@x \pgf@ya=\pgf@y
        \northeast \pgf@xb=\pgf@x \pgf@yb=\pgf@y
        \pgfmathsetlength\pgfutil@tempdima{\pgfkeysvalueof{/pgf/cycleshape w}}
        \pgfpathmoveto{\pgfqpoint{\pgf@xa}{\pgf@ya}}
\pgfpathlineto{\pgfpointadd{\pgfpoint{\pgf@xa}{\pgf@yb}}{\pgfpoint{0ex}{-\pgfutil@tempdima}}}
\pgfpathlineto{\pgfpointadd{\pgfpoint{\pgf@xa}{\pgf@yb}}{\pgfpoint{\pgfutil@tempdima}{0ex}}}
\pgfpathlineto{\pgfpointadd{\pgfpoint{\pgf@xb}{\pgf@yb}}{\pgfpoint{-\pgfutil@tempdima}{0ex}}}
\pgfpathlineto{\pgfpointadd{\pgfpoint{\pgf@xb}{\pgf@yb}}{\pgfpoint{0ex}{-\pgfutil@tempdima}}}
\pgfpathlineto{\pgfqpoint{\pgf@xb}{\pgf@ya}}
        \pgfpathclose
    }
}
\pgfdeclareshape{cycleendshape}{
    \inheritsavedanchors[from=rectangle]
    \inheritanchorborder[from=rectangle]
    \inheritanchor[from=rectangle]{north}
    \inheritanchor[from=rectangle]{center}
    \inheritanchor[from=rectangle]{west}
    \inheritanchor[from=rectangle]{east}
    \inheritanchor[from=rectangle]{mid}
    \inheritanchor[from=rectangle]{base}
    \inheritanchor[from=rectangle]{south}
    \backgroundpath{
        \southwest \pgf@xa=\pgf@x \pgf@ya=\pgf@y
        \northeast \pgf@xb=\pgf@x \pgf@yb=\pgf@y
        \pgfmathsetlength\pgfutil@tempdima{\pgfkeysvalueof{/pgf/cycleshape w}}
        \pgfpathmoveto{\pgfqpoint{\pgf@xb}{\pgf@yb}}
\pgfpathlineto{\pgfpointadd{\pgfpoint{\pgf@xb}{\pgf@ya}}{\pgfpoint{0ex}{\pgfutil@tempdima}}}
\pgfpathlineto{\pgfpointadd{\pgfpoint{\pgf@xb}{\pgf@ya}}{\pgfpoint{-\pgfutil@tempdima}{0ex}}}
\pgfpathlineto{\pgfpointadd{\pgfpoint{\pgf@xa}{\pgf@ya}}{\pgfpoint{\pgfutil@tempdima}{0ex}}}
\pgfpathlineto{\pgfpointadd{\pgfpoint{\pgf@xa}{\pgf@ya}}{\pgfpoint{0ex}{\pgfutil@tempdima}}}
\pgfpathlineto{\pgfqpoint{\pgf@xa}{\pgf@yb}}
        \pgfpathclose
    }
}
\makeatother
\tikzstyle{subroutine} =
    [
        base,
        subrtshape,
    ]
\tikzstyle{cyclebegin} =
    [
        base,
        cyclebegshape,
    ]
\tikzstyle{cycleend} =
    [
        base,
        cycleendshape,
    ]
\tikzstyle{connector} =
    [
        base,
        circle,
    ]

\tikzstyle{parallel} =
    [
        base,
        parallelshape,
    ]

\newcommand{\UnitInfo}{
\begin{tabular}{*{2}{| l}|}
UnitInfo & Информация об одном файле исходного кода \\
\hline
    arguments           & список аргументов компиляции \\
    directory           & папка с файлом исходного кода \\
    file                & имя файла \\
\end{tabular}
}
\newcommand{\BuildInfo}{
\begin{tabular}{*{2}{| l}|}
BuildInfo  & Информация о сборке программы \\
\hline
        units\_info          & список файлов исходного кода \\
\end{tabular}
}

\newcommand{\FunctionInfo}{
\begin{tabular}{*{2}{| l}|}
    CflowConstruct & Описание функции в статическом анализе \\
\hline
        name         & имя функции \\
        nesting      & уровень вложенности \\
        signature    & сигнатура функции \\
        path         & имя файла, в котором используется функция \\
        line         & номер строки \\
        recursive    & рекурсивность функции \\
        text\_offset & отступ в сегменте .text \\
\end{tabular}
}

\begin{tikzpicture}[%
    start chain=going below,    % General flow is top-to-bottom
    node distance=6mm and 30mm, % Global setup of box spacing
    ] 
        \node [process] (unitinfo)                                {\small \UnitInfo};
        \node [process] (buildinfo)    [below = 4cm of unitinfo]  {\small \BuildInfo};
        \node [process] (functioninfo) [below = 4cm of buildinfo] {\small \FunctionInfo};


\end{tikzpicture}

    }
    \caption{Сохраняемые структуры статического анализа \label{fig:static-json}}
\end{figure}

Структуры данных, относящиеся к статическому анализу косвенно связаны друг с
другом. 
Их можно разделить на структуры времени компиляции программы и структуры времени статического анализа.
К структурам времени компиляции относятся:
\begin{itemize}
    \item \texttt{UnitInfo} содержит информацию о сборке одного файла исходного кода;
        В нее входит:
        \begin{itemize}
            \item аргументы компилятору -- указание заголовочных файлов, параметры генерации машинного кода,
                указание макросов и т.д.;
            \item папка, в котрой находится файл исходного кода;
            \item название файла.
        \end{itemize}
    \item \texttt{BuildInfo} агрегирует все \texttt{UnitInfo}, полученные при компиляции проекта и записанные в
        compilation database \autocite{compile-db};
\end{itemize}
К структурам времени анализа относится \texttt{CflowConstruct}, которая содержит в себе уже разобранную
и типизированную информацию, предоставляемую Cflow -- программой статического анализа:
\begin{itemize}
    \item имя функции;
    \item уровень вложенности вызова -- уровень дерева, на котором располагается конкретная функция, относительно
        точки входа -- функции с нулевым уровнем вложенности;
    \item сигнатура функции, в данном случае вместе с возвращаемым типом;
    \item путь до файла, в котором функция была использована;
    \item номер строки, где функция была использована;
    \item рекурсивность функции -- значение принимающее либо <<ложь>>, либо <<истина>>, в зависимости, есть ли в
        определении функции вызов самой себя;
    \item отступ в области .text -- количество в байтах от начала .text-сегмента уже скомпилированной программы до
        начала функции.
\end{itemize}

Все значения, кроме \texttt{text\_offset}, заполняются непосредственно во время проведения статического анализа.

\texttt{text\_offset} заполняется на стадии агрегации результатов линковки и результатов статического анлиза.
Это необходимо, чтобы на стадии сравнительного анализа можно было сопоставить адреса вызываемых функций в динамическом
и статическом анализе, полагаясь на разность между началом сегмента .text и адресом функции. Как на стадии линковки, так
и в динамическом анализе для конкретной фунции он будет одинаков.


\subsection{Cхема данных}\label{sec:ch2/sec2/sub2}

\begin{figure}[!htbp]
    \centerfloat{
        \tikzset{
    line/.style={draw, -latex'},
%     every join/.style={line},
    u/.style={anchor=south},
    r/.style={anchor=west},
    fxd/.style={text width = 6em},
    it/.style={font={\itshape}},
    bf/.style={font={\bfseries}}

}
\tikzstyle{base} =
    [
        draw,
%         on chain,
%         on grid, именно из-за этой опции у вас node distance было расстоянием между центрами, а не между блоками
%         align=center,
%         minimum width = 5ex,
%         node distance = 6mm and 60mm,
        text badly centered,
        text width=12em,
        minimum height=3ex,
        inner xsep = 1pt,
        inner ysep = 3pt,
    ]
\tikzstyle{coord} =
    [
        coordinate,
%         on chain,
%         on grid
    ]
\tikzstyle{cloud} =
    [
        base,
        ellipse,
%         node distance = 3cm,
%         minimum height = 2em
    ]
\tikzstyle{decision} =
    [
        base,
        diamond,
        aspect=2,
%         node distance = 2cm,
        inner sep = 0pt
    ]
\tikzstyle{block} =
    [
        rectangle,
        base,
        rounded corners,
%         minimum height = 2em
    ]
\tikzstyle{print_block} =
    [
        base,
        tape,
        tape bend top=none,
    ]
\tikzstyle{io} =
    [
        base,
        trapezium,
        trapezium left angle = 70,
        trapezium right angle = 110,
    ]
\tikzstyle{prompt} =
    [
        base,
        trapezium,
        trapezium left angle = 90,
        trapezium right angle = 80,
        shape border rotate = 90
    ]
\tikzstyle{disk file} =
    [
        base,
        cylinder,
        aspect=0.2,
        minimum width=4ex, % то, что у~лежачего цилиндра по вертикали — это ширина
    ]
\tikzstyle{process} =
    [
        rectangle,
        base,
    ]
\makeatletter
\pgfkeys{/pgf/.cd,
    subrtshape w/.initial=2mm,
    cycleshape w/.initial=2mm
}
\pgfdeclareshape{subrtshape}{
    \inheritsavedanchors[from=rectangle]
    \inheritanchorborder[from=rectangle]
    \inheritanchor[from=rectangle]{north}
    \inheritanchor[from=rectangle]{center}
    \inheritanchor[from=rectangle]{west}
    \inheritanchor[from=rectangle]{east}
    \inheritanchor[from=rectangle]{mid}
    \inheritanchor[from=rectangle]{base}
    \inheritanchor[from=rectangle]{south}
    \backgroundpath{
        \southwest \pgf@xa=\pgf@x \pgf@ya=\pgf@y
        \northeast \pgf@xb=\pgf@x \pgf@yb=\pgf@y
        \pgfmathsetlength\pgfutil@tempdima{\pgfkeysvalueof{/pgf/subrtshape w}}
        \def\ppd@offset{\pgfpoint{\pgfutil@tempdima}{0ex}}
        \def\ppd@offsetm{\pgfpoint{-\pgfutil@tempdima}{0ex}}
        \pgfpathmoveto{\pgfqpoint{\pgf@xa}{\pgf@ya}}
        \pgfpathlineto{\pgfqpoint{\pgf@xb}{\pgf@ya}}
        \pgfpathlineto{\pgfqpoint{\pgf@xb}{\pgf@yb}}
        \pgfpathlineto{\pgfqpoint{\pgf@xa}{\pgf@yb}}
        \pgfpathclose
        \pgfpathmoveto{\pgfpointadd{\pgfpoint{\pgf@xa}{\pgf@yb}}{\ppd@offsetm}}
        \pgfpathlineto{\pgfpointadd{\pgfpoint{\pgf@xa}{\pgf@ya}}{\ppd@offsetm}}
        \pgfpathlineto{\pgfpointadd{\pgfpoint{\pgf@xb}{\pgf@ya}}{\ppd@offset}}
        \pgfpathlineto{\pgfpointadd{\pgfpoint{\pgf@xb}{\pgf@yb}}{\ppd@offset}}
        \pgfpathclose
    }
}
\pgfdeclareshape{cyclebegshape}{
    \inheritsavedanchors[from=rectangle]
    \inheritanchorborder[from=rectangle]
    \inheritanchor[from=rectangle]{north}
    \inheritanchor[from=rectangle]{center}
    \inheritanchor[from=rectangle]{west}
    \inheritanchor[from=rectangle]{east}
    \inheritanchor[from=rectangle]{mid}
    \inheritanchor[from=rectangle]{base}
    \inheritanchor[from=rectangle]{south}
    \backgroundpath{
        \southwest \pgf@xa=\pgf@x \pgf@ya=\pgf@y
        \northeast \pgf@xb=\pgf@x \pgf@yb=\pgf@y
        \pgfmathsetlength\pgfutil@tempdima{\pgfkeysvalueof{/pgf/cycleshape w}}
        \pgfpathmoveto{\pgfqpoint{\pgf@xa}{\pgf@ya}}
\pgfpathlineto{\pgfpointadd{\pgfpoint{\pgf@xa}{\pgf@yb}}{\pgfpoint{0ex}{-\pgfutil@tempdima}}}
\pgfpathlineto{\pgfpointadd{\pgfpoint{\pgf@xa}{\pgf@yb}}{\pgfpoint{\pgfutil@tempdima}{0ex}}}
\pgfpathlineto{\pgfpointadd{\pgfpoint{\pgf@xb}{\pgf@yb}}{\pgfpoint{-\pgfutil@tempdima}{0ex}}}
\pgfpathlineto{\pgfpointadd{\pgfpoint{\pgf@xb}{\pgf@yb}}{\pgfpoint{0ex}{-\pgfutil@tempdima}}}
\pgfpathlineto{\pgfqpoint{\pgf@xb}{\pgf@ya}}
        \pgfpathclose
    }
}
\pgfdeclareshape{cycleendshape}{
    \inheritsavedanchors[from=rectangle]
    \inheritanchorborder[from=rectangle]
    \inheritanchor[from=rectangle]{north}
    \inheritanchor[from=rectangle]{center}
    \inheritanchor[from=rectangle]{west}
    \inheritanchor[from=rectangle]{east}
    \inheritanchor[from=rectangle]{mid}
    \inheritanchor[from=rectangle]{base}
    \inheritanchor[from=rectangle]{south}
    \backgroundpath{
        \southwest \pgf@xa=\pgf@x \pgf@ya=\pgf@y
        \northeast \pgf@xb=\pgf@x \pgf@yb=\pgf@y
        \pgfmathsetlength\pgfutil@tempdima{\pgfkeysvalueof{/pgf/cycleshape w}}
        \pgfpathmoveto{\pgfqpoint{\pgf@xb}{\pgf@yb}}
\pgfpathlineto{\pgfpointadd{\pgfpoint{\pgf@xb}{\pgf@ya}}{\pgfpoint{0ex}{\pgfutil@tempdima}}}
\pgfpathlineto{\pgfpointadd{\pgfpoint{\pgf@xb}{\pgf@ya}}{\pgfpoint{-\pgfutil@tempdima}{0ex}}}
\pgfpathlineto{\pgfpointadd{\pgfpoint{\pgf@xa}{\pgf@ya}}{\pgfpoint{\pgfutil@tempdima}{0ex}}}
\pgfpathlineto{\pgfpointadd{\pgfpoint{\pgf@xa}{\pgf@ya}}{\pgfpoint{0ex}{\pgfutil@tempdima}}}
\pgfpathlineto{\pgfqpoint{\pgf@xa}{\pgf@yb}}
        \pgfpathclose
    }
}
\makeatother
\tikzstyle{subroutine} =
    [
        base,
        subrtshape,
    ]
\tikzstyle{cyclebegin} =
    [
        base,
        cyclebegshape,
    ]
\tikzstyle{cycleend} =
    [
        base,
        cycleendshape,
    ]
\tikzstyle{connector} =
    [
        base,
        circle,
    ]

% \small
% \footnotesize
\scriptsize
\renewcommand{\baselinestretch}{0.8}
\sf

\noindent
\resizebox{\linewidth}{!}{
% !TEX encoding = UTF-8 Unicode
% Úτƒ-8 encoded
% http://www.linux.org.ru/forum/general/10357036
% \begin{figure}
% \hspace{-4cm}
% \small
\begin{tikzpicture}[%
    start chain=main_vert going below,    % General flow is top-to-bottom
    start chain=main_horz going right,  
    start chain=rev_vert going above,    
    node distance=1.ex and 1em, % Global setup of box spacing
%     scale=0.7, 
%     every node/.style={scale=0.72}
every on chain/.style=join,
    ] 

        \tikzstyle{fitblock}=[inner sep = 0ex]
        \tikzstyle{shortline}=[draw, thin]
        \tikzstyle{longline}=[shortline,-latex']
        \tikzstyle{revline}=[shortline,latex'-]
        \tikzstyle{nodraw}=[draw=none]

        \tikzset{every join/.style=shortline}
        \node [disk file ] (sources)         [on chain=main_vert                   ] {  Файлы с~исходным кодом};
        \node [disk file ] (makefile)        [right  = of sources ] {  Makefile};
%         \node [prompt    ] (makefile path)   [right  = of makefile                             ] {  Путь до папки с~Makefile};
        \coordinate                          [on chain=main_vert] (main_from_makefile);
%         \coordinate[on chain=main_vert] (main_from_makefile path);
%         \coordinate                          [on chain=main_vert] (no_used);
        \tikzset{every join/.style=shortline} %longline
        \node [process   ] (builder)         [on chain=main_vert              ] {  Сборка};
        \tikzset{every join/.style=shortline}
                
        \node [disk file ] (build log)       [on chain=main_vert        ] {  Файл с~информацией о~сборке};
        \node [process   ] (static analyzer) [on chain=main_vert        ] {  Модуль статического анализа};
        \node [disk file ] (stat result)     [on chain=main_vert        ] {  Результаты статического анализа};
        \node [process   ] (stat parser)     [on chain=main_vert        ] {  Модуль преобразования результатов статического анализа};
        \node [disk file ] (stat json)       [on chain=main_vert                ] {  Преобразованные результаты статического анализа};
        \node [process   ] (aggregator)      [on chain=main_vert        ] {  Модуль агрегирования результатов линковки и~статического анализа};
        \node [disk file ] (aggregator file) [on chain=main_vert           ] {  Агрегированные результаты линковки и~статического анализа};
        \node [process   ] (comparer)        [on chain=main_vert           ] {  Модуль сравнительного анализа};
        \node [disk file ] (summary)         [on chain=main_vert           ] {  Результаты сравнительного анализа};
        \tikzset{every join/.style=nodraw}

        
        \draw [longline] (makefile)   |- (main_from_makefile);
%         \draw [longline] (makefile path)         |- (main_from_makefile path);

%         \draw [longline] (sources)         |- (static analyzer);
         \coordinate                          [left = of static analyzer] (static_from_sources);
        \draw [longline] (sources)     -| (static_from_sources)    -- (static analyzer);
       
        \node [disk file ] (call map)        [right =  of build log                ] {  Файл с~информацией о~линковке};
        \draw [longline] (builder)         -| (call map);
        \draw [longline] (call map)        |- (aggregator);
       
        \node[fit=(call map.north) (sources) (summary), fitblock] (left_vert_base) {};
        
        
        
        % самая широкая часть правой вертикали
        \tikzset{every join/.style=nodraw}
%         \coordinate                          [right = of stat parser.north east-|left_vert_base.east, on chain=main_horz] (right_vert_anchorpoint); % right = of stat result?
        \coordinate                          [right = of stat result.south-|left_vert_base.east, on chain=main_horz] (right_vert_anchorpoint); % right = of stat result?
        \node [disk file ] (gdb script)      [on chain=main_horz] {  Скрипт для GDB};
        \coordinate                          [on chain=main_horz] (center_from_breakpointer);
        \node [disk file ] (modified exe)    [on chain=main_horz] {  Модифицированный исполняемый файл};
        \node[fit=(gdb script) (modified exe), fitblock] (right_vert_cross) {};
       
        % вверх
        \node [process   ] (breakpointer)    [on chain=rev_vert, above = of right_vert_cross] {  Модуль бинарного анализа};
        \draw [longline] (breakpointer)    -| (modified exe);
        \draw [longline] (breakpointer)    -| (gdb script);
%         \tikzset{every join/.style=revline}
%         \tikzset{every join/.style=shortline} %revline
%         \coordinate                          [on chain=rev_vert] (executable_to_breakpointer);
        \tikzset{every join/.style=shortline}
        \node [disk file ] (file executable) [on chain=rev_vert   ] {  Исполняемый\\файл};
        \draw [longline] (builder)         -| (file executable);
        
% %         \node [prompt    ] (executable)      [right = of  file executable          ] {  Путь до исполняемого файла};
%         \node [prompt    ] (executable)      [above = of  file executable.north-|modified exe          ] {  Путь до исполняемого файла};
%         \draw [revline] (executable_to_breakpointer)    -| (executable);            
        
        % вниз
        \tikzset{every join/.style=nodraw}
        \node [process   ] (gdb manager)     [on chain=main_vert, below = of right_vert_cross   ] {  Модуль управления отладчиком};
        \draw [longline] (modified exe)    |- (gdb manager);
        \draw [longline] (gdb script)      |- (gdb manager);

        \tikzset{every join/.style=shortline}
        \node [disk file ] (dyn result)      [on chain=main_vert                ] {  Результаты динамического анализа};
        \node [process   ] (dyn parser)      [on chain=main_vert                ] {  Модуль преобразования результатов динамического анализа};
        \node [disk file ] (dyn json)        [on chain=main_vert                ] {  Преобразованные результаты динамического анализа};
   
        \draw [longline] (dyn json)        |- (comparer);
     
        
       
% другой вариант, по раскладке ближе к исходному       
%         \coordinate[right = of sources, on chain=main_horz] (right_vert_anchorpoint);
%         \node [disk file ] (file executable) [on chain=main_vert, right = of right_vert_anchorpoint             ] {  Исполняемый\\файл};
%         \tikzset{every join/.style=shortline}
%         \coordinate[on chain=main_vert] (executable_to_breakpointer);
%         \node [prompt    ] (executable)      [right = of  file executable          ] {  Путь до исполняемого файла};
%         \draw [revline] (executable_to_breakpointer)    -| (executable);
% 
%         
%         \tikzset{every join/.style=longline}
%         \node [process   ] (breakpointer)    [on chain=main_vert   ] {  Модуль бинарного анализа};% решает последнее
%         \tikzset{every join/.style=nodraw}
%         \coordinate[on chain=main_vert] (center_from_breakpointer);
% %         \node[circle, fill=red] at  (center_from_breakpointer) {};
%       
%         
% %         \node [disk file ] (modified exe)    [right = of center_from_breakpointer, anchor = north west] {  Модифицированный исполняемый файл}; у~цилиндра north west почти на north
%         \node [disk file ] (modified exe)    [right = of center_from_breakpointer, anchor = after bottom] {  Модифицированный исполняемый файл};
%         \node [disk file ] (gdb script)      [left = of center_from_breakpointer|-modified exe] {  Скрипт для GDB};
%         \node [process   ] (gdb manager)     [on chain=main_vert, below = of breakpointer|-modified exe.south              ] {  Модуль управления отладчиком};
%         \tikzset{every join/.style=shortline}
%         \node [disk file ] (dyn result)      [on chain=main_vert                ] {  Результаты динамического анализа};
%         \node [process   ] (dyn parser)      [on chain=main_vert                ] {  Модуль преобразования результатов динамического анализа};
%         \node [disk file ] (dyn json)        [on chain=main_vert                ] {  Преобразованные результаты динамического анализа};
%         
%         
% %         \draw [line] (builder)         -| (build log);
% % 
% % %         \draw [line] (executable)      -| (breakpointer);
% % %         \draw [line] (file executable) -| (breakpointer);
%         \draw [longline] (breakpointer)    -| (modified exe);
%         \draw [longline] (breakpointer)    -| (gdb script);
% % 
%         \draw [longline] (modified exe)    |- (gdb manager);
%         \draw [longline] (gdb script)      |- (gdb manager);
% % %         \draw [-] (gdb manager)        -- (dyn result);
% % %         \draw [-] (dyn result)         -- (dyn parser);
% % %         \draw [-] (dyn parser)         -- (dyn json);
% % 
% % 
% % %         \draw [-] (build log)          -- (static analyzer);
% % %         \draw [-] (static analyzer)    -- (stat result);
% % %         \draw [-] (stat result)        -- (stat parser);
% % %         \draw [-] (stat parser)        -- (stat json);
% % %         \draw [-] (stat json)          -- (aggregator);
% % %         \draw [-] (aggregator)         -- (aggregator file);
% % 
% % %         \draw [-] (aggregator file)    -- (comparer);
%         \draw [longline] (dyn json)        |- (comparer);
% % %         \draw [-] (comparer)           -- (summary);
        
        
% контроль полей
%         \draw [red] (current bounding box.south east) rectangle (current bounding box.north west);
\end{tikzpicture}
}

    }
    \caption{Схема данных {\ProgModule}\label{fig:dataflow}}
\end{figure}
Из схемы данных на \autoref{fig:dataflow} видно, что работу {\ProgModule} можно разбить на параллельные 
задачи.


\subsection{Алгоритм работы программы}\label{sec:ch2/sec2/sub3}
Работу {\ProgModule} можно разделить на функциональные этапы:
\begin{enumerate}[label={\arabic*)}]
    \item сборка анализируемой программы;
    \item статический анализ результатов сборки\label{statical-analysis-stage};
    \item динамический анализ собранной программы\label{dynamical-analysis-stage};
    \item сравнительный анализ результатов статического и динамического анализа.
\end{enumerate}

Причем \autoref{statical-analysis-stage} и \autoref{dynamical-analysis-stage} могут выполняться
одновременно, так как не имеют зависимости по данным.

Рассмотрим подробнее каждый из этапов.
\subsubsection{Сборка анализируемой программы}\label{sec:ch2/sec2/sub3/sub1}

\paragraph{Утилита make}\label{sec:ch2/sec2/sub3/sub1/par1}\mbox{}

Make -- утилита для автоматической сборки программ и библиотек из исходного кода.
Работает через чтение специальных файлов -- <<мейкфайлов>> (англ. Makefile), в которых 
описаны <<рецепты>> сборки. В мейкфайле может находиться любое количество рецептов, они могут
быть как зависимы друг от друга, так и быть совершенно непересекающимися.

Отдельный рецепт имеет название, компоненты, от которых он зависит (могут остать пустыми, это будет означать,
что рецепт независим) и правила сборки, они тоже могут оставаться пустыми.

Стоит заметить, что использование программы make в UNIX системах не обязательно ограничевается
компиляцией программ и библиотек. В мейкфайлах с помощью рецептов так же можно описать различные сценарии,
требующие последовательного выполнения команд. В большинстве программ, использующих схему распространения
через компиляцию исходного кода, имеются мейкфайлы, в которых определены рецепты clean -- очистить и help --
помощь. Которые реализуют, соответственно, очистку директорий проекта от временных файлов, полученных в результате
выполнения других рецептов мейкфайла и получения информации о доступных рецептах.

По-умолчанию, make выполняет рецепты один за другим, не начиная выполнение нового рецепта, пока
не закончится старый. Но при указании определенного аргумента, make может выполнять несвязанные рецепты
параллельно, что значительно ускоряет процесс сборки.

\paragraph{Утилита BEAR}\label{sec:ch2/sec2/sub3/sub1/par2}\mbox{}

Build EAR \autocite{bear}, или сокращенно BEAR позволяет генерировать compilation database, указывая ей
команду сборки.

Сборка анализируемой программы происходит посредством программы-обертки, повторяющей интерфейс программы make
и запускающая её в контексте программы BEAR, для генерации compilation database. Помимо этого, для make
указывается генерация map-файла, файла содержащего информацию о сегментах программы,
относительных отступах функций внутри сегментов и др. После окончания компиляции дополнительно
происходит разбор сегмента .text map-файла на предмет функций и их относительных адресов внутри сегмента.
Полученные данные сохраняются на диск в JSON формате.


\subsubsection{Статический анализ результатов сборки}\label{sec:ch2/sec2/sub3/sub2}
Статический анализ результатов сборки производится с помощью программы Cflow, которой на вход
подаются аргументы компиляции, взятые из compilation database, полученной на предыдущем шаге,
а так же сами файлы с исходными кодами.

Отчет Cflow состоит из списка функций, определяемых следующим правилом, описанным в \autoref{lst:cflow-rule},
где описания полей обрамлены косыми чертами:

\begin{ListingEnv}[!h]
    \captiondelim{ }
    \caption{Формат записи в отчете Cflow}\label{lst:cflow-rule}
    \begin{lstlisting}[]
{/уровень вложенности/} /имя функции/() </сигнатура функции вместе с возвращаемым значением/ at /абсолютный путь до файла/:/номер строки в файле/>:
{/уровень вложенности вызываемой функции/} /имя вызываемой функции/() </сигнатура вызываемой функции вместе с возвращаемым значением/ at /абсолютный путь до файла/:/номер строки в файле/>:

    ... 
    \end{lstlisting}
\end{ListingEnv}

Данный формат файла легко поддается разбору с помощью регулярных выражений. В {\ProgModule} использовалась
библиотека регулярных выражений PCRE \autocite{pcre}.
Не смотря на то, что Cflow умеет генерировать отчет, в которых представлен не граф вызываемых функций,
а список функций, вызывавших данную, этот формат, не смотря на удобство, страдает большим количеством
повторений, что в свою очередь вызывает слишком большой объем отчета и замедляет его разбор, из-за
чего в {\ProgModule} решено было использовать стандартную версию отчета.

\begin{ListingEnv}[!h]
    \captiondelim{ }
    \caption{Пример генерации отчета Cflow}\label{lst:cflow-example}
    \begin{Verb}[]
{   0} printsel() <void printsel (const arg *arg) at /st/st.c:1988>:
{   1}  tdumpsel() <void tdumpsel (void) at /st/st.c:1994>:
{   2}    getsel() <char *getsel (void) at /st/st.c:590>:
{   3}      xmalloc() <void *xmalloc (size_t len) at /st/st.c:253>:
{   4}        malloc()
{   4}        die() <void die (const char *errstr, ...) at /st/st.c:654>:
    \end{Verb}
\end{ListingEnv}

\subsubsection{Динамический анализ собранной программы}\label{sec:ch2/sec2/sub3/sub3}
Подготовка к динамическому анализу собранной программы начинается сразу после завершения 
этапа сборки \autoref{sec:ch2/sec2/sub3/sub1}. Путь до исполняемого файла передается в модуль расстановки
точек останова для первичного модифицирования. Модифирование заключается в том, что с помощью
программ objdump и readelf, о которых говорилось в \autoref{sec:ch2/sec1/sub1/sub4} и
небольших скриптов, написанных на bash, происходит следующее: 
\begin{enumerate}[label={\arabic*)}]
    \item находятся все \texttt{call}-инструкции, сохраняя их относительные адреса от начала сегмента .text;
    \item узнается отступ сегмента .text в байтах от начала файла;
    \item сохраняется байт по адресу, полученным на предыдущем шаге;
    \item заменяется байт по адресу, полученным на предыдущем шаге, на 0xCC в шестнадцатиричной системе счисления.
        Это машинный код инструкции \texttt{int 3} -- программного прерывания, которое используется в отладчиках
        для установки точек останова;
    \item генерируется скрипт для отладчика GDB, по расстановке точек останова на все \texttt{call}-инструкции,
        восстановлению изменений в файле и снятию состояний программы.

\end{enumerate}
Процесс исполнения данного скрипта:
\begin{enumerate}[label={\arabic*)}]
    \item \verb|file абсолютный-путь-до-файла| -- загружается исполняемый файл по абсолютному пути;
    \item выставляется формат выводимых данных:
            \begin{enumerate}[label={\arabic*)}]
                \item \verb|set disassembly-flavor intel| -- выставляется отображение синтаксиса
                    ассемблерных мнемоник в стиль intel;
                \item \verb|set input-radix 10| -- выставляется десятичная система для ввода;
                \item \verb|set args аргументы-программе| -- анализируемой программе передаются аргументы;
                \item 
                    \begin{Verbatim}
define xxd
    dump binary memory dump.bin $arg0 $arg0+$arg1
    shell xxd dump.bin >> gdb.log
end 
                    \end{Verbatim} 
                    -- определеяется команда \verb|xxd|, которая будет добавлять в лог
                            динамического анализа дамп заданного места памяти;
            \end{enumerate}
    \item \verb|run| -- запускается исследуемая программа;
    \item
        \begin{Verbatim}
info proc 
info files
info functions
        \end{Verbatim} 
        -- выводится информация о процессе, сегментах и обнаруженных функциях;
    \item программа останавливается на первом байте сегмента .text, 0xCC, кодирующем программную точку останова;
    \item \verb|set $pc--| -- счетчик команд уменьшается на единицу;
    \item \verb|set *(char*)$pc=байт| -- по адресу, указанном в счетчике команд записывается ранее сохраненный первый байт сегмента .text;
    \item расставляются относительные точки останова;
    \item программа выходит из останова и продолжает работу, собирая информацию с точек останова.
\end{enumerate}

Информация с прошедших точек останова собирается с помощью следующих команд GDB:
\begin{Verbatim}
commands
    info registers
    x/8i $pc
    bt
    xxd $sp-256 256
    continue
end
\end{Verbatim}

Нужно отметить, что в отладчике GDB существует команда \texttt{starti}, которая запускает программу и 
останавливается на первой инструкции, что позволяет отлаживать программу прямо с точки входа.
Но проблема использования \texttt{starti} состоит в том, что первой инструкцией программы может оказаться
не .text-сегмент, а какой-нибудь другой, а значит относительная расстановка точек будет неверной.
Поэтому приходится на уровне исполняемого файла удостоверяться, что исполнение программы прервется
именно на первой инструкции .text-сегмента.

\subsubsection{Сравнительный анализ результатов статического и динамического анализа}\label{sec:ch2/sec2/sub3/sub3}
Модуль сравнительного анализа запускается после того, как становятся готовы результаты статического и
динамического анализа.
Он загружает результаты с диска в описанные ранее структуры \autoref{sec:ch2/sec2/sub1/sub1}, а так же
информацию о функциях из map-файла.
Это позволяет дать отчет по нескольким вариантам несовпадения:
\begin{itemize}
    \item несовпадение функций в map-файле и функций, объявленных в статическом анализе
        (каких имен из множества функций, полученных из map-файла нет среди функций, определенных в исходникых текстах);
    \item несовпадение распознанных отладчиком GDB функций и функций, полученных в динамическом анализе
        (каких имен из множества функций, определенных GDB нет среди функций, полученных из map-файла);
    \item несовпадение функций в статическом и динамическом анализе.
\end{itemize}

\begin{figure}[!htbp]
    \centerfloat{
        % !TEX encoding = UTF-8 Unicode
% Úτƒ-8 encoded
% http://www.linux.org.ru/forum/general/10357036
\tikzset{
    pics/.cd,
    line/.style={draw, -latex'},
    every join/.style={line},
    u/.style={anchor=south},
    r/.style={anchor=west},
    fxd/.style={text width = 6em},
    it/.style={font={\small\itshape}},
    bf/.style={font={\small\bfseries}},
}
\tikzstyle{base} =
    [
        draw,
        on chain,
        on grid,
        align=center,
        minimum height=4ex,
        minimum width = 10ex,
        node distance = 6mm and 60mm,
        text badly centered,
        text width=5cm
    ]
\tikzstyle{coord} =
    [
        coordinate,
        on chain,
        on grid
    ]
\tikzstyle{cloud} =
    [
        base,
        ellipse,
        node distance = 3cm,
        minimum height = 2em,
        text width=2cm
    ]
\tikzstyle{decision} =
    [
        base,
        diamond,
        aspect=2,
        node distance = 2cm,
        inner sep = 0pt
    ]
\tikzstyle{block} =
    [
        rectangle,
        base,
        rounded corners,
        minimum height = 2em
    ]
\tikzstyle{print_block} =
    [
        base,
        tape,
        tape bend top=none,
    ]
\tikzstyle{io} =
    [
        base,
        trapezium,
        trapezium left angle = 70,
        trapezium right angle = 110,
    ]
\tikzstyle{prompt} =
    [
        base,
        trapezium,
        trapezium left angle = 90,
        trapezium right angle = 80,
        shape border rotate = 90
    ]
\tikzstyle{disk file} =
    [
        base,
        cylinder,
        aspect=0.2,
    ]
\tikzstyle{process} =
    [
        rectangle,
        base,
    ]
\makeatletter
\pgfkeys{/pgf/.cd,
    subrtshape w/.initial=2mm,
    cycleshape w/.initial=2mm
}
\pgfdeclareshape{parallelshape}{
    \inheritsavedanchors[from=rectangle]
    \inheritanchorborder[from=rectangle]
    \inheritanchor[from=rectangle]{north}
    \inheritanchor[from=rectangle]{center}
    \inheritanchor[from=rectangle]{west}
    \inheritanchor[from=rectangle]{east}
    \inheritanchor[from=rectangle]{mid}
    \inheritanchor[from=rectangle]{base}
    \inheritanchor[from=rectangle]{south}
    \backgroundpath{
        \southwest \pgf@xa=\pgf@x \pgf@ya=\pgf@y
        \northeast \pgf@xb=\pgf@x \pgf@yb=\pgf@y
        \pgfmathsetlength\pgfutil@tempdima{\pgfkeysvalueof{/pgf/subrtshape w}}
        \def\ppd@offset{\pgfpoint{\pgfutil@tempdima}{0ex}}
        \def\ppd@offsetm{\pgfpoint{-\pgfutil@tempdima}{0ex}}
        \pgfpathmoveto{\pgfqpoint{\pgf@xa}{\pgf@ya}}
            \pgfpathlineto{\pgfqpoint{\pgf@xb}{\pgf@ya}}
        \pgfpathclose
        \pgfpathmoveto{\pgfqpoint{\pgf@xb}{\pgf@yb}}
            \pgfpathlineto{\pgfqpoint{\pgf@xa}{\pgf@yb}}
        \pgfpathclose
    }
}
\pgfdeclareshape{subrtshape}{
    \inheritsavedanchors[from=rectangle]
    \inheritanchorborder[from=rectangle]
    \inheritanchor[from=rectangle]{north}
    \inheritanchor[from=rectangle]{center}
    \inheritanchor[from=rectangle]{west}
    \inheritanchor[from=rectangle]{east}
    \inheritanchor[from=rectangle]{mid}
    \inheritanchor[from=rectangle]{base}
    \inheritanchor[from=rectangle]{south}
    \backgroundpath{
        \southwest \pgf@xa=\pgf@x \pgf@ya=\pgf@y
        \northeast \pgf@xb=\pgf@x \pgf@yb=\pgf@y
        \pgfmathsetlength\pgfutil@tempdima{\pgfkeysvalueof{/pgf/subrtshape w}}
        \def\ppd@offset{\pgfpoint{\pgfutil@tempdima}{0ex}}
        \def\ppd@offsetm{\pgfpoint{-\pgfutil@tempdima}{0ex}}
        \pgfpathmoveto{\pgfqpoint{\pgf@xa}{\pgf@ya}}
        \pgfpathlineto{\pgfqpoint{\pgf@xb}{\pgf@ya}}
        \pgfpathlineto{\pgfqpoint{\pgf@xb}{\pgf@yb}}
        \pgfpathlineto{\pgfqpoint{\pgf@xa}{\pgf@yb}}
        \pgfpathclose
        \pgfpathmoveto{\pgfpointadd{\pgfpoint{\pgf@xa}{\pgf@yb}}{\ppd@offsetm}}
        \pgfpathlineto{\pgfpointadd{\pgfpoint{\pgf@xa}{\pgf@ya}}{\ppd@offsetm}}
        \pgfpathlineto{\pgfpointadd{\pgfpoint{\pgf@xb}{\pgf@ya}}{\ppd@offset}}
        \pgfpathlineto{\pgfpointadd{\pgfpoint{\pgf@xb}{\pgf@yb}}{\ppd@offset}}
        \pgfpathclose
    }
}
\pgfdeclareshape{cyclebegshape}{
    \inheritsavedanchors[from=rectangle]
    \inheritanchorborder[from=rectangle]
    \inheritanchor[from=rectangle]{north}
    \inheritanchor[from=rectangle]{center}
    \inheritanchor[from=rectangle]{west}
    \inheritanchor[from=rectangle]{east}
    \inheritanchor[from=rectangle]{mid}
    \inheritanchor[from=rectangle]{base}
    \inheritanchor[from=rectangle]{south}
    \backgroundpath{
        \southwest \pgf@xa=\pgf@x \pgf@ya=\pgf@y
        \northeast \pgf@xb=\pgf@x \pgf@yb=\pgf@y
        \pgfmathsetlength\pgfutil@tempdima{\pgfkeysvalueof{/pgf/cycleshape w}}
        \pgfpathmoveto{\pgfqpoint{\pgf@xa}{\pgf@ya}}
\pgfpathlineto{\pgfpointadd{\pgfpoint{\pgf@xa}{\pgf@yb}}{\pgfpoint{0ex}{-\pgfutil@tempdima}}}
\pgfpathlineto{\pgfpointadd{\pgfpoint{\pgf@xa}{\pgf@yb}}{\pgfpoint{\pgfutil@tempdima}{0ex}}}
\pgfpathlineto{\pgfpointadd{\pgfpoint{\pgf@xb}{\pgf@yb}}{\pgfpoint{-\pgfutil@tempdima}{0ex}}}
\pgfpathlineto{\pgfpointadd{\pgfpoint{\pgf@xb}{\pgf@yb}}{\pgfpoint{0ex}{-\pgfutil@tempdima}}}
\pgfpathlineto{\pgfqpoint{\pgf@xb}{\pgf@ya}}
        \pgfpathclose
    }
}
\pgfdeclareshape{cycleendshape}{
    \inheritsavedanchors[from=rectangle]
    \inheritanchorborder[from=rectangle]
    \inheritanchor[from=rectangle]{north}
    \inheritanchor[from=rectangle]{center}
    \inheritanchor[from=rectangle]{west}
    \inheritanchor[from=rectangle]{east}
    \inheritanchor[from=rectangle]{mid}
    \inheritanchor[from=rectangle]{base}
    \inheritanchor[from=rectangle]{south}
    \backgroundpath{
        \southwest \pgf@xa=\pgf@x \pgf@ya=\pgf@y
        \northeast \pgf@xb=\pgf@x \pgf@yb=\pgf@y
        \pgfmathsetlength\pgfutil@tempdima{\pgfkeysvalueof{/pgf/cycleshape w}}
        \pgfpathmoveto{\pgfqpoint{\pgf@xb}{\pgf@yb}}
\pgfpathlineto{\pgfpointadd{\pgfpoint{\pgf@xb}{\pgf@ya}}{\pgfpoint{0ex}{\pgfutil@tempdima}}}
\pgfpathlineto{\pgfpointadd{\pgfpoint{\pgf@xb}{\pgf@ya}}{\pgfpoint{-\pgfutil@tempdima}{0ex}}}
\pgfpathlineto{\pgfpointadd{\pgfpoint{\pgf@xa}{\pgf@ya}}{\pgfpoint{\pgfutil@tempdima}{0ex}}}
\pgfpathlineto{\pgfpointadd{\pgfpoint{\pgf@xa}{\pgf@ya}}{\pgfpoint{0ex}{\pgfutil@tempdima}}}
\pgfpathlineto{\pgfqpoint{\pgf@xa}{\pgf@yb}}
        \pgfpathclose
    }
}
\makeatother
\tikzstyle{subroutine} =
    [
        base,
        subrtshape,
    ]
\tikzstyle{cyclebegin} =
    [
        base,
        cyclebegshape,
    ]
\tikzstyle{cycleend} =
    [
        base,
        cycleendshape,
    ]
\tikzstyle{connector} =
    [
        base,
        circle,
    ]

\tikzstyle{parallel} =
    [
        base,
        parallelshape,
    ]
\begin{tikzpicture}[%
    start chain=going below,    % General flow is top-to-bottom
    node distance=6mm and 30mm, % Global setup of box spacing
    scale=0.7, 
    every node/.style={scale=0.72}
    ] 
        \node [cloud    ] (makefile)        [left  = 2cm ]                   {\small Начало};
        \node [process  ] (builder)         [below = 1cm of makefile]        {\small Сборщик};
        \node [parallel ] (parallel)        [below of = builder]             {};
        \node [process  ] (breakpointer)    [below right = 3cm of builder]   {\small Бинарный анализ};
        \node [process  ] (static analyzer) [below left = 3cm of builder]    {\small Статический анализ};
        \node [process  ] (stat parser)     [below = 2cm of static analyzer] {\small Преобразование результатов статического анализа};
        \node [process  ] (aggregator)      [below = 2cm of stat parser]     {\small Агрегирование результатов линковки и статического анализа};
        \node [process  ] (gdb manager)     [below = 2cm of breakpointer]    {\small Динамический анализ};
        \node [process  ] (dyn parser)      [below = 2cm of gdb manager]     {\small Преобразование результатов динамического анализа};
        \node [cloud    ] (end)             [below = 11cm of makefile]     {\small Конец};
        \node [process  ] (comparer)        [above = 1cm of end]             {\small Сравнительный анализ};
        \node [parallel ] (parallel aggr)   [above = 1cm of comparer]             {};

        \draw [->] (makefile)        -- (builder);
        \draw [-] (builder)          -- (parallel);

        \draw [-] (parallel)         -- (static analyzer);
        \draw [-] (parallel)         -- (breakpointer);

        \draw [->] (static analyzer) -- (stat parser);
        \draw [->] (stat parser)     -- (aggregator);

        \draw [->] (breakpointer)    -- (gdb manager);
        \draw [->] (gdb manager)     -- (dyn parser);

        \draw [-] (aggregator)       -- (parallel aggr);
        \draw [-] (dyn parser)       -- (parallel aggr);
        \draw [-] (parallel aggr)    -- (comparer);
        \draw [->] (comparer)        -- (end);

\end{tikzpicture}

    }
    \caption{Алгоритм работы {\ProgModule}\label{fig:algorithm}}
\end{figure}

\subsection{Разработка консольного интерфейса {\ProgModule}}\label{sec:ch2/sec2/sub4}
Консольный интерфейс программы, или программа, поддерживающая интерфейс командной строки --
компьютерная программа, обрабатывающая аргументы, переданные ей в определенном формате. 
Консольный интерфейс не может существовать без командного интерпретатора -- другой компьютерной программы, которая
обрабатывает команды компьютеру, заданные в виде текста.
Один из самых старых видов взаимодействия человека и компьютера.
Появившись в середине 1960-х, он используется и по сей день.


На текущий момент, для запуска {\ProgModule} нужно перейти в папку с собранным {\ProgModule},
после чего использовать bash-скрипт \autoref{lst:run.sh}, который
принимает следующие аргументы: 
\begin{enumerate}[label={\arabic*)}]
    \item \texttt{\$1} -- путь до папки, в которой хранится мейкфайл проекта;
    \item \texttt{\$2} -- путь до исследуемого исполняемого файла.
\end{enumerate}
Результаты сравнительного анализа выводятся на экран.

\begin{ListingEnv}[!h]
    \captiondelim{ }
    \caption{run.sh}\label{lst:run.sh}
    \small
    \begin{Verb}[]
    pushd $1
        make clean
    popd
    pushd build
        ./build -C=$1
        (./set_breakpoints -e=$2 &&
         ./gdb ;
         ./parse_log &&
         ./dynamic_analysis;) &
        (./static_analysis &&
         ./aggregation) &
         wait $(jobs -p)
        ./comparative_analysis
    popd
    \end{Verb}
\end{ListingEnv}

Если же {\ProgModule} еще не собран, то нужно воспользоваться скриптом \autoref{lst:build.sh}:
\begin{ListingEnv}[!h]
    \captiondelim{ }
    \caption{build.sh}\label{lst:build.sh}
    \small
    \begin{Verb}[]
    rm -rf build
    mkdir build 
    pushd build
       for source_file in $(find ../breakpoints ../analysis -name "*.nim"); do
           echo $source_file
           nim --parallelBuild:$(nproc) \
               --outDir=. \
               -p=.. \
               --threads:on \
               c $(readlink -f $source_file) &
       done
       wait $(jobs -p)
    popd
    \end{Verb}
\end{ListingEnv}

\section{Выводы по разделу}\label{sec:ch1/sec5}
В конструкторском разделе было проведено сравнение и обоснование выбора языка программирования
и среды разработки для {\ProgModule}.
Разработана архитектура {\ProgModule}.
Также были описаны:
\begin{enumerate}[label={\arabic*)}]
    \item алгоритм передачи данных между модулями {\ProgModule};
    \item формат данных, передающихся между модулями {\ProgModule};
    \item используемые сторонние программы и форматы данных, обрабатываемые ими.
\end{enumerate}
Составлена схема данных, алгоритм работы {\ProgModule}.
Подробно рассмотрены шаги выполнения процесса сертификации с помощью {\ProgModule}
