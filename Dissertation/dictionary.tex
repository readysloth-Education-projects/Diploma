\chapter*{Словарь терминов}             % Заголовок
\addcontentsline{toc}{chapter}{Словарь терминов}  % Добавляем его в оглавление
\textbf{Кроссплатформенный}: программа, которая может запускаться на различных операционных системах и/или архитектурах процессоров           \\
\textbf{Отладчик}: программа, в контексте которой запускается другая программа для локализации и устранения ошибок
в контролируемых условиях                                                                                                                     \\
\textbf{Отладка}: процесс локализации и устранения ошибок программы в контролируемых условиях                                                 \\
\textbf{Препроцессор}: программа-макропроцессор, обрабатывающая специальные директивы в исходном коде и запускающаяся до компилятора          \\
\textbf{Препроцессирование}: процесс обработки исходного кода препроцессором                                                                  \\
\textbf{Открытое ПО}: ПО с открытым исходным кодом, который доступен для просмотра, изучения и изменения                                      \\
\textbf{Сериализация}: процесс перевода определеного типа данных программы в некоторый формат                                                 \\
\textbf{Десериализация}: процесс перевода данных, находящихся в некотором формате, во внутренний тип данных программы                         \\
\textbf{Скрипт}: программа, обычно на интерпретируемом языке программирования, выполняющая конкретное действие                                \\
\textbf{Сигнатура функции}: объявление функции, в которое входит имя функции, количество входных параметров и их тип                          \\
\textbf{Сборка}: процесс компиляции, линковки и публикации программного обеспечения из исходных кодов                                         \\
\textbf{Сверхвысокоуровневый ЯП}: классификация языков программирования, к данной категории относятся языки программирования,
позволяющие описать задачу не на уровне <<как нужно сделать>>, а на уровне <<что нужно сделать>>                                              \\
\textbf{Система контроля версий}: система, сохраняющая изменения одного или нескольких файлов в течение времени и позволяющая вернуться к их определенной версии \\
\textbf{Коммит}: снимок системы контроля версий во времени \\
\textbf{Source-to-source}: компиляция исходного кода некоторого языка в исходный код другого языка. Во время компиляции языка данным способом
 может происходить несколько итераций преобразования,
 пока последний язык в цепочке преобразований не будет скомпилирован в машинный код или интерпретирован \\
