\chapter*{Словарь терминов}             % Заголовок
\addcontentsline{toc}{chapter}{Словарь терминов}  % Добавляем его в оглавление

\textbf{Кросс-платформенный}     : Программа, которая может запускаться на различных операционных системах и/или архитектурах процессоров            \\
\textbf{Программная закладка}    : Подпрограмма, либо фрагмент исходного кода, скрытно внедренный в исполняемый файл                                 \\
\textbf{Динамическая трасса}     : Дерево вызванных программой функций во время конкретного ее исполнения                                            \\
\textbf{Статическая трасса}      : Дерево функций программы, которые объявлены для вызова                                                            \\
\textbf{Отладчик}                : Программа, в контексте которой запускается другая программа для локализации и устранения ошибок
в контролируемых условиях \\
\textbf{Отладка}                 : Процесс локализации и устранения ошибок программы в контролируемых условиях                                       \\
\textbf{Удаленная отладка}       : Процесс отладки программы, запущенной вне контекста отладчика                                                     \\
\textbf{Препроцессор}            : Программа-макропроцессор, обрабатывающая специальные директивы в исходном коде и запускающаяся до компилятора     \\
\textbf{Препроцессирование}      : Процесс обработки исходного кода препроцессором                                                                   \\
\textbf{Открытое ПО}             : ПО с открытым исходным кодом, который доступен для просмотра, изучения и изменения                                \\
\textbf{Сериализация}            : Процесс перевода определеного типа данных программы в некоторый формат                                            \\
\textbf{Десериализация}          : Процесс перевода данных, находящихся в некотором формате, во внутренний тип данных программы                      \\
\textbf{Скрипт}                  : Программа, обычно на интерпретируемом языке программирования, выполняющая конкретное действие                     \\
\textbf{Сигнатура функции}       : Объявление функции, в которое входит имя функции, количество входных параметров и их тип                          \\
\textbf{Сборка}                  : Процесс компиляции, линковки и публикации программного обеспечения из исходных кодов                              \\
\textbf{Рефакторинг}             : Процесс улучшения кода без введения новой функциональности. Результатом является чистый код с улучшенным дизайном \\
\textbf{Релизная сборка}         : Сборка программы происходит без отладочных символов, обычно с использованием техник оптимизации кода              \\
\textbf{Терминал}                : То же, что и консоль                                                                                              \\
\textbf{Антиотладка}             : Набор методов детектирования отлаживаемой программы окружения отладки и препятствование ей                        \\
\textbf{Мультитаскинг}           : Возможность программы или операционной системы обеспечивать возможность параллелльного исполнения задач           \\
\textbf{Сверхвысокоуровневый ЯП} : Классификация языков программирования, к данной категории относятся языки программирования,
позволяющие описать задачу не на уровне <<как нужно сделать>>, а на уровне <<что нужно сделать>>                                                     \\
\textbf{Source-to-source} : Компиляция исходного кода некоторого языка в исходный код другого языка. Во время компиляции языка данным способом
 может происходить несколько итераций преобразования, пока последний язык в цепочке преобразований не будет скомпилирован в машинный код или интерпретирован\\
