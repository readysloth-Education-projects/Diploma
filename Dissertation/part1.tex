\chapter{Создание аппаратного обеспечения}\label{ch:ch1}

В представлении обывателя компьютер (системный блок) состоит из материнской платы,
процессора, оперативной памяти, возможно видеокарты.
Давно прошли те времена, когда пользователям компьютеров приходилось отдельно покупать такие модули расширения,
как, например, математический сопроцессор.
С течением времени все больше ранее внешних модулей становится частью материнской платы или самого процессора.

Но производство аппаратного обеспечения не ограничивается потребительским рынком.
Специфические аппаратные решения требуются отдельным отраслям или организациям.

Создание аппаратного обеспечения трудоемкий процесс, непременно сязанный с созданием прикладного программного обеспечения.
В свою очередь, отсутствие возможности исполнения ПО с использованием АО серьезно усложняет отладку и тестирование конечного продукта.

\section{Эмуляция аппаратного обеспечения}\label{sec:ch1/sec1}

Тезис Чёрча-Тьюринга гласит: любая вычислимая (то есть та,
которая может быть реализована на машине Тьюринга) функция вычислима машиной Тьюринга.
Физический тезис Чёрча-Тьюринга гласит: любая функция, которая может быть вычислена физическим устройством, может быть вычислена машиной Тьюринга.

Существуют различное отношение к тезису Чёрча-Тьюринга.
Некоторые считают, что он может быть доказан, другие говорят, что он служит определением вычислений.
Не смотря на то, что тезис до сих пор не доказан, его верность исходит из того, что любая, открытая на текущий момент
реалистичная модель вычислений доказывала его правоту.

Тезис неразрывно связан с термином "эмуляция".

Эмуляция -- это процесс имитирование поведения одного оборудования и/или программного обеспечения
на другом оборудовании и/или программном обеспечении.

Эмулятор АО может использоваться как для имитирования совершенно другого оборудования, так и того, на котором она проводится.
Например, большое количество принтеров умеет эмулировать линейку LaserJet компании Hewlett-Packard,
так как большая часть ПО разработана как раз под LaserJet.
Эмуляторы аппаратного обеспечения бывают разного назначения:

\paragraph{Симуляторы логики}\label{logic-sim}

Данный вид эмуляторов используется для исследования и верификации аппаратного обеспечения на различных уровнях:
\begin{itemize}
    \item компонентном;
    \item логических вентилей;
    \item регистровых передач;
\end{itemize}

\paragraph{Высокоуровневые эмуляторы}\label{high-level-emu}

Высокоуровневая эмуляция -- это набор методов эмулирования некоторых компонентов целой системы.
Позволяет не исполнять инструкции или производить обработку данных один в один, как на целевом оборудовании,
заменяя "горячие" пути обработки аналогичными по результату.

Одними из первых данный подход был использован в эмуляторах игровых консолей, где команды отрисовки трёхмерной сцены
эмулировались не на процессоре, как все остальное оборудование консоли, а отдавались напрямую
графическому процессору машины, на которой происходил процесс эмуляции.


\paragraph{Эмуляторы процессора}\label{cpu-emu}

Эмулируют конкретную архитектуру процессора, позволяя запускать программы, не предназначенные для физического компьютера.
Зачастую умеют эмулировать не только процессор но и переферийные устройства.


\paragraph{Эмуляторы терминала}\label{term-emu}

Эмулируют терминал компьютера внутри некоторой архитектуры отображения данных на дисплее.


\paragraph{Сэмуляторы}\label{sim-emu}

Являются симбиозом симуляторов и эмуляторов, который берет лучшее от двух миров.
Аппаратное обеспечение описовается HDL языками, после чего данное описание оборудования симулируется на стендах.
Первоначальная функциональная верификация производится через симуляцию на уровне регистровых передач или логических вентилей.
В событийно-ориентированной симуляции инструкции последовательно исполняются процессором, потому что в превалирующем большинстве
сценариев не возможно провести данную симуляцию параллельно. Последовательных подход приводит к долгим симуляциям, особенно в
сложных системах на кристалле.
После симуляции описание регистровых передач должно быть зашито в аппаратное обеспечение (FPGA, ASIC).
Но идеализированное представление аппаратного обеспечения в симуляции отличается от реального аппаратного обеспечения.
Отличие между симуляцией и работой аппаратного обеспечения является серьезной причиной применений эмуляции при
проектировании.
Преимуществом данного метода является:
\begin{itemize}
    \item ускорение симуляции: часть сложной системы, не релевантная для симуляции переносится в эмулятор,
          что позволяет вынести из симуляции нерелевантные части;
    \item использование реального аппаратного обеспечения на ранних этапах проектирования;
\end{itemize}


\section{Аппаратное обеспечение -- физическое и виртуальное}\label{sec:ch1/sec2}

Прикладная польза от эмуляции аппаратного обеспечения может быть достигнута не только
при проектировании и разработке специализированного аппаратного обеспечения, но и, напрмер,
для совместного использования одного физического устройства несколькими компьютерами (в основном
виртуальными машинами).
Представляя аппаратное обеспечение гостевой системе как физическим устройство, можно
свести на нет затраты на написание специализированных драйверов и утилизировать
существующие, вынося обработку совместного использования в реализацию эмулируемого аппаратного обеспечения.

Например, можно запустить некоторое количество виртуальных машин на одной физической, и для каждой
виртуальной машины использование графического ускорителя будет прозрачным, тогда как на самом
деле управлением задачами отрисовки или обработки данных будет заниматься виртуальное устройство.

Также вынесение интерфейса реального устройства в виртуальное позволяет применить общение с программой-посредником,
которая будет отдавать команды реальному устройству или группе устройств.
Например, запускать в виртуальной машине вычисления на виртуальных графических ускорителях, которые
будут, в свою очередь, отдавать задачи обсчета реальным графическим ускорителям где-нибудь в облаке.
