\chapter{Аналитический обзор существующих методов нахождения НДВ}\label{ch:ch1}
\section{Анализ существующих подходов к нахождению НДВ}\label{sec:ch1/sec1}
Процедура нахождения НДВ состоит из следующих этапов:
\begin{enumerate}[label={\arabic*)}]
    \item готовность документации ПО, доступность исходных текстов;
    \item опредление объема исходных текстов, подлежащих анализу;
    \item обращение заявителя в испытательную лабораторию с собранной информацией;
    \item анализ документации;
    \item разработка <<Программы и методик проведения сертификационных испытаний>>;
    \item проведение испытаний;
    \item экспертиза результатов.
\end{enumerate}

Сертификация должна выявить присутствие в исполняемом файле недекларированных возможностей,
которые могут являться как злым умыслом разработчиков компилятора,
линкера и других вспомогательных программ, так и методами оптимизации ПО,
которые применяются для более рационального
потребления ресурсов программой.


\section{Анализ актуальности существующих нормативных документов для исследования ПО на НДВ}\label{sec:ch1/sec2}

\section{Анализ имеющегося программного обеспечения для исследования ПО на НДВ}\label{sec:ch1/sec3}

\section{Постановка задач диссертации}\label{sec:ch1/sec4}

\section{Выводы по главе 1}\label{sec:ch1/sec5}
