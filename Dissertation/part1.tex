\chapter{Исследовательский раздел}\label{ch:ch1}
\section{Процесс сертификации ПО на отсутствие НДВ}\label{sec:/sec1}
Сертификация программного обеспечения проводится, 
когда необходимо подтвердить соответствие разрабатываемой 
продукции требованиям защиты информации.

Сертификационная процедура состоит из следующих этапов:
\begin{enumerate}
    \item Готовность документации ПО, доступность исходных текстов
    \item Опредление объема исходных текстов, подлежащих анализу
    \item Обращение заявителя в испытательную лабораторию с собранной информацией
    \item Анализ документации
    \item Разработка <<Программы и методик проведения сертификационных испытаний>>
    \item Проведение испытаний
    \item Экспертиза результатов
\end{enumerate}

Сертификация должна выявить присутствие в исполняемом файле недекларированных возможностей,
которые могут являться как злым умыслом\autocite{compile-a-virus, ken-thompson-hack} 
разработчиков компилятора, линкера и других вспомогательных программ,
так и методами оптимизации ПО, которые применяются для более рационального 
потребления ресурсов программой.

Выявить данные расхождения между необработанными исходными кодами и 
поведением программы во время исполнения позволяет разработанный мной
программный модуль.

Дадим определение термину <<Недекларированные возможности>>:

\textbf{Недекларированные возможности}\autocite{undeclared-capabilities-antimalware} 
— намеренно измененная часть ПО, с помощью которой можно получить незаметный 
несанкционированный доступ к безопасной компьютерной среде.

\section{Классификация НДВ}\label{sec:ch1/sec2}
    \subsection{По применению}\label{sec:ch1/sec2/sub1}
    \begin{itemize}
        \item \textbf{Перехват данных}
        \item \textbf{Подмена данных}
        \item \textbf{Вывод компьютерной системы из строя}
        \item \textbf{Полный доступ к удаленной компьютерной системе}

              Такие программы могут быть использованы злоумышленниками для
              всех вышеперечисленных целей
    \end{itemize}
\subsection{По целям}\label{sec:ch1/sec2/sub2}
\begin{itemize}
    \item \textbf{Персональные компьютеры и рабочие станции}
          
          Целью могут быть как персональные компьютеры широкого числа пользователей,
          так и отдельные рабочие станции, которые могут являться точкой входа в
          защищенную компьютерную систему, так и использоваться для перехвата важной
          информации
    \item \textbf{Серверы}
          
          Серверы обслуживают большое количество клиентов, а значит проникновение на
          сервер может существенно повлиять на работу всех компьютеров, работающих с
          данным сервером
    \item \textbf{Встраиваемые системы}

          Благодаря постоянному удешелению микроконтроллеров и периферийных устройств,
          все больше и больше повседневных вещей обзаводятся <<умной>> функциональностью.
          Погоня производителей за прибылями отражается на безопасности прошивок умных устройств.
    \item \textbf{Промышленные компьютеры}

        Программные закладки в такие системы чреваты шпионажем или диверсией\autocite{stuxnet}
        \footnote{Хотя данная программа является вирусом, а не программой с НДВ, случившееся
        ярко показывает реальное применение подобных техник для деструктивных действий}.
\end{itemize}

\section{Степень опасности НДВ}\label{sec:ch1/sec3}
Для определения опасности НДВ будем пользоваться следующими нормативными документами:
\begin{itemize}
    \item Приказ ФСТЭК России от 18 февраля 2013 г. № 21
    \item Федеральный закон "О персональных данных" от 27.07.2006 N 152-ФЗ
\end{itemize}

Тип угроз безопасности персональных данных определяется 
в зависимости от комбинаций актуальности угроз в ИСПДн (\autoref{table:threats}):
\begin{itemize}
    \item Наличием НДВ в системном программном обеспечении (ПО), используемом в ИСПДн
    \item Наличием НДВ в прикладном ПО, используемом в ИСПДн
\end{itemize}


\newcommand{\vital}[1]{
    \cellcolor{red!30} 
    #1 }

\newcommand{\nonvital}[1]{
    \cellcolor{green!30} 
    #1 }

\begin{table}[!htbp]
    \centering
    %\captionsetup{justification=centering}


    \begin{center}
        \begin{tabular}{ | c | c | c | c | }
            \hline
            Угрозы & \multicolumn{3}{ c |}{Тип актуальных угроз} \\
            \cline{2-4}
                   & 1 Тип & 2 Тип & 3 Тип\\
            \hline
            \makecell{Наличие НДВ в системном ПО,\\ используемом в ИСПДн} & \vital{актуально} & \nonvital{неактуально} & \vital{неактуально} \\
            \hline
            \makecell{Наличие НДВ в прикладном ПО \\ используемом в ИСПДн} & \makecell{актуально \\ или \\ неактуально} & \vital{актуально} & \nonvital{неактуально} \\
            \hline
        \end{tabular}
    \end{center}

    \caption{\label{table:threats}Тип актуальных угроз}

\end{table}

Порядок определения актуальных угроз безопасности 
персональных данных в ИСПДн осуществляется 
в соответствии с Методикой определения 
актуальных угроз безопасности персональных данных 
при их обработке в информационных 
системах персональных данных, утвержденных 
ФСТЭК России, 2008 год.

Актуальной считается угроза, которая может быть 
реализована в ИСПДн и представляет опасность для 
персональных данных.  Подход к составлению перечня 
актуальных угроз состоит в следующем. 
Для оценки возможности реализации угрозы применяются два показателя: 

$Y_1$ - уровень исходной защищенности ИСПДн\\
$Y_2$ - частота (вероятность) реализации рассматриваемой угрозы.\\
Коэффициент реализуемости угрозы $Y$ определяется соотношением: 

\[Y = \frac{Y_1 + Y_2}{20}\]

По значению коэффициента реализуемости угрозы Y 
интерпретация реализуемости угрозы следующим образом: 
\begin{itemize}
    \item если $0 \leq Y \leq 0.3$, то возможность реализации угрозы признается низкой
    \item если $0.3 < Y \leq  0.6$, то возможность реализации угрозы признается средней 
    \item если $0.6 < Y \leq  0.8$, то возможность реализации угрозы признается высокой 
    \item если $Y > 0,8$, то возможность реализации угрозы признается очень высокой
\end{itemize}

Далее оценивается опасность каждой угрозы.
Этот показатель имеет три значения: 
\begin{itemize}
    \item низкая опасность – если реализация угрозы может привести к незначительным негативным последствиям для субъектов персональных данных
    \item средняя опасность – если реализация угрозы может привести к негативным последствиям для субъектов персональных данных
    \item высокая опасность – если реализация угрозы может привести к значительным негативным последствиям для субъектов персональных данных.
\end{itemize}

Затем осуществляется выбор из общего
перечня угроз безопасности тех, 
которые относятся к актуальным для данной ИСПДн,
в соответствии с правилами, приведенными в \autoref{table:actual-threats}
\begin{table}[!htbp]
    \centering
    %\captionsetup{justification=centering}

    \begin{center}
        \begin{tabular}{ | c | c | c | c | }
            \hline
            \makecell{Возможность \\ реализации угрозы} & \multicolumn{3}{ c |}{Показатель опасности угрозы} \\
            \cline{2-4}
                   & Низкая & Средняя & Высокая\\
            \hline
            \makecell{Низкая}        & \nonvital{неактуальная} & \nonvital{неактуальная} & \vital{актуальная} \\
            \hline
            \makecell{Средняя}       & \nonvital{неактуальная} & \vital{актуальная}      & \vital{актуальная} \\
            \hline
            \makecell{Высокая}       & \vital{актуальная}      & \vital{актуальная}      & \vital{актуальная} \\
            \hline
            \makecell{Очень высокая} & \vital{актуальная}      & \vital{актуальная}      & \vital{актуальная} \\
            \hline
        \end{tabular}
    \end{center}

    \caption{\label{table:actual-threats}Правила отнесения угрозы безопасности персональных данных к актуальной}

\end{table}

После чего выносится решение о проведении анализа ПО 
на НДВ в процесс сертификации или его игнорирование,
как неактуального.

\section{Обзор программных решений для сертификации ПО на отсутствие НДВ}\label{sec:ch1/sec3}
На сегодняшний день не существует в комплексных разработок по сертификации программного обеспечения
на предмет НДВ. Однако, существуют программы, специализирующиеся отдельно на анализе исходных кодов
и отдельно исполняемого файла. Рассмотрим их по отдельности

\subsection{Статические анализаторы}\label{sec:ch1/sec3/sub1}
\begin{table}[!htbp]
    \centering
    %\captionsetup{justification=centering}

    \begin{center}
        \begin{tabular*}{\textwidth}{| c | c | c | c | c | }
            \hline
            \makecell{Название \\ программы} & 
            \makecell{Описание} & 
             \makecell{Кросс-платформенность} &
             \makecell{Поддерживаемые\\языки} & 
             \makecell{Доступность} \\
            \hline
            \makecell{Microsoft\\Application\\Inspector} &
            \makecell[l]{
             Кросс-платформенный, open-source инструмент\\
             для анализа исходных кодов прогамм.\\
             Отличается от традиционных \\
             инструментов статического анализа \\
             тем, что он не пытается определить \\
             «хорошие» или «плохие» паттерны; \\
             он сообщит о том, что обнаружит,\\
             по исходному набору из \\
             более чем 500 шаблонов правил \\
             для обнаружения функций} &
             \nonvital{Да} &
             \makecell{34 языка\\и распознает\\их смешение} &
             \vital{\makecell{Бесплатна\\и с открытым\\исходным кодом}} \\
            \hline
        \end{tabular*}
    \end{center}

    \caption{\label{table:static-analysis-comparsion}Сравнение статических анализаторов}

\end{table}
\subsection{динамические анализаторы}\label{sec:ch1/sec3/sub2}
