\chapter*{Список сокращений и условных обозначений} % Заголовок
\addcontentsline{toc}{chapter}{Список сокращений и условных обозначений}  % Добавляем его в оглавление
\noindent
%\begin{longtabu} to \dimexpr \textwidth-5\tabcolsep {r X}
\begin{longtabu} to \textwidth {r X}
% Жирное начертание для математических символов может иметь
% дополнительный смысл, поэтому они приводятся как в тексте
% диссертации

\textbf{QOM}         & Qemu Object Model \\
\textbf{QMP}         & Qemu Machine Protocol \\
\textbf{TCG}         & Tiny Code Generator \\
\textbf{QAPI}        & Qemu Application Programming Interface \\
\textbf{ПО}          & Программное обеспечение \\
\textbf{АО}          & Аппаратное обеспечение \\
\textbf{GUI}         & Graphical User Interface \\
\textbf{АПМДЗ}       & Аппаратно-Программный Модуль Доверенной Загрузки \\
\textbf{IDE}         & Интегрированная среда разработки \\
\textbf{JSON}        & Формат описания структур данных в текстовом виде~ключ~$\rightarrow$~значение \\

\end{longtabu}
\addtocounter{table}{-1}% Нужно откатить на единицу счетчик номеров таблиц, так как предыдующая таблица сделана для удобства представления информации по ГОСТ
