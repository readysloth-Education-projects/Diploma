\chapter*{Список сокращений и условных обозначений} % Заголовок
\addcontentsline{toc}{chapter}{Список сокращений и условных обозначений}  % Добавляем его в оглавление
\noindent
%\begin{longtabu} to \dimexpr \textwidth-5\tabcolsep {r X}
\begin{longtabu} to \textwidth {r X}
% Жирное начертание для математических символов может иметь
% дополнительный смысл, поэтому они приводятся как в тексте
% диссертации

\textbf{QOM}         & Qemu Object Model (Объектная модель Qemu)\\
\textbf{ПМ}          & Программный модуль \\
\textbf{ПО}          & Программное обеспечение \\
\textbf{АО}          & Аппаратное обеспечение \\
\textbf{ЯП}          & Язык программирования\\
\textbf{GUI}         & Graphical User Interface \\
\textbf{IDE}         & Интегрированная среда разработки \\
\textbf{JSON}        & Формат описания структур данных в текстовом виде~ключ~$\rightarrow$~значение \\
\textbf{PID}         & Уникальный идентификатор процесса в ОС \\
\textbf{TDD}         & Test-driven development \\
\textbf{HDL}         & Hardware description language \\
\textbf{\ProgModule} & Программный модуль анализа на недекларированные возможности \\

\end{longtabu}
\addtocounter{table}{-1}% Нужно откатить на единицу счетчик номеров таблиц, так как предыдующая таблица сделана для удобства представления информации по ГОСТ
