\chapter*{Заключение}                       % Заголовок
\addcontentsline{toc}{chapter}{Заключение}  % Добавляем его в оглавление

%% Согласно ГОСТ Р 7.0.11-2011:
%% 5.3.3 В заключении диссертации излагают итоги выполненного исследования, рекомендации, перспективы дальнейшей разработки темы.
%% 9.2.3 В заключении автореферата диссертации излагают итоги данного исследования, рекомендации и перспективы дальнейшей разработки темы.
%% Поэтому имеет смысл сделать эту часть общей и загрузить из одного файла в автореферат и в диссертацию:

Результатом выпускной квалификационной работы стала рабочая версия
программного модуля анализа программ на языках C/C++ на недекларированные
возможности. {\ProgModule} позволил унифицировать и ускорил процесс исследования
программного обеспечения на НДВ.
Уменьшение фрагментации программ по анализу ПО на наличие НДВ позволяет
не тратить время программистов на написание анализатора под конкретный продукт,
что положительно сказывается на продуктивности всей команды разработчиков.

В рамках выпускной квалификационной работы были решены задачи:
\begin{enumerate}[label={\arabic*)}]
    \item исследование предметной области ;
    \item сравнительный анализ существующих программных решений;
    \item выбор языка и среды разработки;
    \item разработка схемы данных {\ProgModule};
    \item разработка схемы алгоритма {\ProgModule};
    \item программирование {\ProgModule};
    \item отладка и тестирование {\ProgModule};
    \item разработка документации к {\ProgModule}.
\end{enumerate}
 
В заключение автор
выражает благодарность и большую признательность научному руководителю
Кононовой Александре Игоревне за поддержку, помощь, обсуждение результатов и~научное
руководство.
