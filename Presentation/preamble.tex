%\begin{frame}[noframenumbering,plain]
%    \setcounter{framenumber}{1}
%    \maketitle
%\end{frame}

\begin{frame}
\frametitle{\textbf{Тема:~\thesisTitle}}
\textbf{Руководитель от кафедры:}~\supervisorRegaliaShort~\supervisorFioShort

\textbf{Исполнитель}~ст. гр. ПИН-43~\thesisAuthorShort

\textbf{Цель:} Ускорение проведения сравнительного анализа статических и динамических трасс программ, написанных на C/C++

\textbf{Задачи:}
\begin{enumerate}
    \item исследование предметной области;
    \item сравнительный анализ существующих программных решений;
    \item выбор языка и среды разработки;
    \item разработка схемы данных {\ProgModule};
    \item разработка схемы алгоритма {\ProgModule};
    \item программирование {\ProgModule};
    \item отладка и тестирование {\ProgModule};
    \item разработка документации к {\ProgModule};
\end{enumerate}
\end{frame}
%\begin{frame}
%    \frametitle{Положения, выносимые на защиту}
%    \begin{itemize}
%        \item Результаты расчёта этого путём таким-то.
%        \item Результаты разработки того.
%        \item И ещё \dots
%        \item \dots пару пунктов.
%    \end{itemize}
%\end{frame}
%\note{
%    Проговариваются вслух положения, выносимые на защиту
%}
%
%\begin{frame}
%    \frametitle{Содержание}
%    \tableofcontents
%\end{frame}
%\note{
%    Работа состоит из четырёх глав.
%
%    \medskip
%    В первой главе \dots
%
%    Во второй главе \dots
%
%    Третья глава посвящена \dots
%
%    В четвёртой главе \dots
%}
