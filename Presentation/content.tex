%\subsection{Исследование предметной области}
\begin{frame}[shrink=20]%[plain, noframenumbering, t, shrink=20]
\frametitle{Исследование предметной области}
    \vspace{-2ex} 
    \begin{figure}[!htbp]
        \includegraphics[width=\textwidth,height=\textheight,keepaspectratio]{images/uml_before_cropped.png}
    \end{figure}
    \begin{figure}[!htbp]
    \vspace{-2.6ex} 
        \includegraphics[width=\textwidth,height=\textheight,keepaspectratio]{images/uml_after_cropped.png}
    \end{figure}
%\begin{table}[!htbp]
%    \centering
%    \caption{\label{table:why-am-i-the-best}До и после разработки {\ProgModule}}
%
%    \begin{center}
%        \begin{tabular}{ | c | c | }
%            \hline
%            \makecell{До разработки {\ProgModule}} & \makecell{После разработки {\ProgModule}}\\
%            \hline
%            \makecell{Проведение статического,\\
%                      динамического и сравнительного\\
%                      анализа проходило вручную} & 
%            \makecell{Проведение статического,\\
%                      динамического и сравнительного\\
%                      анализа проходит автоматически}\\
%            \hline
%            \makecell{Для проведения анализов\\
%                      нужно было вручную выбирать\\
%                      исследуемые файлы} & 
%            \makecell{Для проведения анализов,\\
%                      {\ProgModule}\\
%                      делает это автоматически}\\
%            \hline
%            \makecell{Динамический анализ включал\\
%                      в себя только вызовы функций} & 
%            \makecell{Динамический анализ включает\\
%                      в себя информацию о состоянии\\
%                      стека и регистров программы\\
%                      во время конкретного вызова}\\
%            \hline
%        \end{tabular}
%    \end{center}
%
%\end{table}
\end{frame}

\begin{frame}%[plain, noframenumbering, t]
\frametitle{Обзор существующих решений. Статические анализаторы.}

\begin{table}[!htbp]
    {\footnotesize
        \setlength{\tabcolsep}{2pt}
        \begin{longtable}{*{4}{| c}|}
            \hline
            \diagbox[width=4cm]{Свойства}{Название\\программы}                                       &
            \makecell{Microsoft\\Application\\Inspector \autocite{microsoft-application-inspector}} &
            \makecell{SCI\\Tools\\Understand \autocite{sci-tools-understand}}                       &
            \makecell{GNU cflow \autocite{gnu-cflow}} \\
            \hline
            \makecell{Кроссплатформенность}             & \greencell{Да} & \greencell{Да} & \greencell{Да}\\
            \hline
            \makecell{Открытость\\исходного кода}        & \greencell{Да} & \redcell{Нет}   & \greencell{Да}\\
            \hline
            \makecell{Препроцессирование\\кода C/C++}    & \redcell{Нет}   & \greencell{Да} & \greencell{Да}\\
            \hline
            \makecell{Представление\\
                      препроцессорных директив как\\
                      вызов функций}                     & \redcell{Нет}   & \redcell{Нет}   & \greencell{Да}\\
            \hline
            \makecell{Создание графа вызовов}            & \redcell{Нет}   & \greencell{Да} & \greencell{Да} \\
            \hline
            \makecell{Создание обратного\\графа вызовов} & \redcell{Нет}   & \greencell{Да} & \greencell{Да}\\
            \hline
            \makecell{Бесплатность}                      & \greencell{Да} & \redcell{Нет}   & \greencell{Да}\\
            \hline
            \makecell{Графический интерфейс}             & Нет & Есть & Нет \\
            \hline
        \end{longtable}
    }
\end{table}

\vspace{-5ex}
{\scriptsize
\begin{itemize}
    \item \cite{microsoft-application-inspector} \fullcite{microsoft-application-inspector}
    \item \cite{sci-tools-understand} \fullcite{sci-tools-understand}
    \item \cite{gnu-cflow} \fullcite{gnu-cflow}
\end{itemize}
}

\end{frame}

\begin{frame}%[plain, noframenumbering, t]
\frametitle{Обзор существующих решений. Динамические анализаторы.}

\begin{table}[!htbp]
    {\small
        \setlength{\tabcolsep}{2pt}
        \begin{longtable}{*{3}{| c}|}
            \hline
            \diagbox[width=8cm]{Свойства}{Название программы} &
            \makecell{GDB \autocite{gdb}}                     &
            \makecell{QEMU \autocite{qemu}}                   \\
            \hline
            \makecell{Кроссплатформенность}             & \greencell{Да} & \greencell{Да} \\
            \hline
            \makecell{Открытость исходного кода}        & \greencell{Да} & \greencell{Да} \\
            \hline
            \makecell{Возможность анализировать память} & \greencell{Да} & \greencell{Да} \\
            \hline
            \makecell{Возможность программно управлять} & \greencell{Да} & \greencell{Да} \\
            \hline
            \makecell{Возможность создавать 
                      собственные команды}               & \greencell{Да} & \redcell{Нет}   \\
            \hline
            \makecell{Возможность удаленной 
                      отладки}                           & \greencell{Да} & \redcell{Нет}   \\
            \hline
            \makecell{Бесплатность}                      & \greencell{Да} & \greencell{Да} \\
            \hline
            \makecell{Графический интерфейс}             & Есть & Есть \\
            \hline
        \end{longtable}
    }
\end{table}
\vspace{-5ex}
{\scriptsize
\begin{itemize}
    \item \cite{gdb} \fullcite{gdb}
    \item \cite{qemu} \fullcite{qemu}
\end{itemize}
}
\end{frame}

\begin{frame}%[plain, noframenumbering, t]
\frametitle{Выбор языка программирования}

\begin{table}
    {\small
        \setlength{\tabcolsep}{2pt}
        \begin{longtable}{*{5}{| c}|}
            \hline
            \diagbox[width=5cm]{Свойства}{Язык} &
                \makecell{Nim \autocite{nim}} &
                \makecell{Python \autocite{python}} &
                \makecell{Perl \autocite{perl}} &
                \makecell{C/C++ \autocite{cpp}} \\
            \hline
                \makecell{Сверхвысокоуровневость} & 
                \greencell{Да} & 
                \greencell{Да} &
                \greencell{Да} &
                \redcell{Нет} \\
            \hline
                \makecell{Компилируется в\\машинный код} & 
                \greencell{Да} & 
                \redcell{Нет} &
                \redcell{Нет} &
                \greencell{Да} \\
            \hline
                \makecell{Количество функции в\\стандартной библиотеке} & 
                5585 & 
                638 &
                1338 &
                1224 \\
            \hline
                \makecell{Портируемость} & 
                \greencell{Есть} & 
                \greencell{Есть} &
                \greencell{Есть} &
                \yellowcell{\makecell{Есть,\\но неудобная}}\\
            \hline
                \makecell{Встроенная\\генерация документации} & 
                \greencell{Есть} & 
                \greencell{Есть} &
                \greencell{Есть} &
                \redcell{Нет}\\
            \hline
                \makecell{Статическая типизация} & 
                \greencell{Есть} & 
                \redcell{Нет} &
                \redcell{Нет} &
                \greencell{Есть}\\
            \hline
                \makecell{Автоматическое\\управление памятью} & 
                \greencell{Есть} & 
                \greencell{Есть} &
                \greencell{Есть} &
                \greencell{Есть} \\
            \hline
                \makecell{Обобщенное программирование} & 
                \greencell{Есть} & 
                \greencell{Есть} &
                \greencell{Есть} &
                \greencell{Есть} \\
            \hline
                \makecell{Метапрограммирование} & 
                \greencell{Есть} & 
                \greencell{Есть} &
                \greencell{Есть} &
                \greencell{Есть} \\
            \hline
                \makecell{Опыт использования} & 
                \greencell{Есть} & 
                \greencell{Есть} &
                \redcell{Нет} &
                \greencell{Есть} \\
            \hline
        \end{longtable}
    }
\end{table}
\vspace{-5ex}
{\scriptsize
\begin{itemize}
    \item \cite{nim} \fullcite{nim}
    \item \cite{python} \fullcite{python}
    \item \cite{perl} \fullcite{perl}
\end{itemize}
}

\end{frame}

\begin{frame}%[plain, noframenumbering, t]
\frametitle{Выбор среды разработки}

\begin{table}[!htbp]
    {\small
        \setlength{\tabcolsep}{2pt}
        \begin{longtable}{*{6}{| c}|}
            \hline
            \diagbox[width=4cm]{Свойства}{IDE/Редактор} &
                \makecell{Aporia \autocite{aporia-ide}} &
                \makecell{Atom \autocite{atom-ide}} &
                \makecell{Sublime\\Text \autocite{sublime-ide}} &
                \makecell{Visual\\Studio\\Code \autocite{vs-code-ide}} &
                \makecell{Vim \autocite{vim-ide}} \\
            \hline
                \makecell{Поддержка плагинов} & 
                \redcell{Нет} &
                \greencell{Да} & 
                \greencell{Да} &
                \greencell{Да} &
                \greencell{Да} \\
            \hline
                \makecell{Требователен к ресурсам} & 
                \greencell{Нет} & 
                \redcell{Да} & 
                \greencell{Нет} & 
                \redcell{Да} & 
                \greencell{Нет} \\ 
            \hline
                \makecell{Имеет продвинутую систему\\редактирования текста} & 
                \redcell{Нет} &
                \redcell{Нет} &
                \redcell{Нет} &
                \redcell{Нет} &
                \greencell{Да} \\
            \hline
                \makecell{Кроссплатформенность} & 
                \greencell{Есть} & 
                \greencell{Есть} &
                \greencell{Есть} &
                \greencell{Есть} &
                \greencell{Есть} \\
            \hline
                \makecell{Может работать\\без GUI} & 
                \redcell{Нет} &
                \redcell{Нет} &
                \redcell{Нет} &
                \redcell{Нет} &
                \greencell{Да} \\
            \hline
                \makecell{Восстановление после сбоев} & 
                \redcell{Нет} & 
                \greencell{Есть} &
                \greencell{Есть} &
                \greencell{Есть} &
                \greencell{Есть} \\
            \hline
                \makecell{Возможность выделять\\ключевые слова с помощью\\регулярных выражений} & 
                \redcell{Нет} & 
                \greencell{Есть} &
                \greencell{Есть} &
                \greencell{Есть} &
                \greencell{Есть} \\
            \hline
                \makecell{Опыт использования} & 
                \redcell{Нет} &
                \redcell{Нет} &
                \greencell{Есть} &
                \greencell{Есть} &
                \greencell{Есть} \\
            \hline
        \end{longtable}
    }
\end{table}

\vspace{-5ex}
{\scriptsize
\begin{itemize}
    \item \cite{aporia-ide} \fullcite{aporia-ide}
    \item \cite{atom-ide} \fullcite{atom-ide}
    \item \cite{sublime-ide} \fullcite{sublime-ide}
    \item \cite{vs-code-ide} \fullcite{vs-code-ide}
    \item \cite{vim-ide} \fullcite{vim-ide}
\end{itemize}
}
\end{frame}

\begin{frame}%[plain, noframenumbering, t]
\frametitle{Схема данных {\ProgModule}}
    \begin{figure}[!htbp]
        \tikzset{
    line/.style={draw, -latex'},
%     every join/.style={line},
    u/.style={anchor=south},
    r/.style={anchor=west},
    fxd/.style={text width = 6em},
    it/.style={font={\itshape}},
    bf/.style={font={\bfseries}}

}
\tikzstyle{base} =
    [
        draw,
%         on chain,
%         on grid, именно из-за этой опции у вас node distance было расстоянием между центрами, а не между блоками
%         align=center,
%         minimum width = 5ex,
%         node distance = 6mm and 60mm,
        text badly centered,
        text width=12em,
        minimum height=3ex,
        inner xsep = 1pt,
        inner ysep = 3pt,
    ]
\tikzstyle{coord} =
    [
        coordinate,
%         on chain,
%         on grid
    ]
\tikzstyle{cloud} =
    [
        base,
        ellipse,
%         node distance = 3cm,
%         minimum height = 2em
    ]
\tikzstyle{decision} =
    [
        base,
        diamond,
        aspect=2,
%         node distance = 2cm,
        inner sep = 0pt
    ]
\tikzstyle{block} =
    [
        rectangle,
        base,
        rounded corners,
%         minimum height = 2em
    ]
\tikzstyle{print_block} =
    [
        base,
        tape,
        tape bend top=none,
    ]
\tikzstyle{io} =
    [
        base,
        trapezium,
        trapezium left angle = 70,
        trapezium right angle = 110,
    ]
\tikzstyle{prompt} =
    [
        base,
        trapezium,
        trapezium left angle = 90,
        trapezium right angle = 80,
        shape border rotate = 90
    ]
\tikzstyle{disk file} =
    [
        base,
        cylinder,
        aspect=0.2,
        minimum width=4ex, % то, что у~лежачего цилиндра по вертикали — это ширина
    ]
\tikzstyle{process} =
    [
        rectangle,
        base,
    ]
\makeatletter
\pgfkeys{/pgf/.cd,
    subrtshape w/.initial=2mm,
    cycleshape w/.initial=2mm
}
\pgfdeclareshape{subrtshape}{
    \inheritsavedanchors[from=rectangle]
    \inheritanchorborder[from=rectangle]
    \inheritanchor[from=rectangle]{north}
    \inheritanchor[from=rectangle]{center}
    \inheritanchor[from=rectangle]{west}
    \inheritanchor[from=rectangle]{east}
    \inheritanchor[from=rectangle]{mid}
    \inheritanchor[from=rectangle]{base}
    \inheritanchor[from=rectangle]{south}
    \backgroundpath{
        \southwest \pgf@xa=\pgf@x \pgf@ya=\pgf@y
        \northeast \pgf@xb=\pgf@x \pgf@yb=\pgf@y
        \pgfmathsetlength\pgfutil@tempdima{\pgfkeysvalueof{/pgf/subrtshape w}}
        \def\ppd@offset{\pgfpoint{\pgfutil@tempdima}{0ex}}
        \def\ppd@offsetm{\pgfpoint{-\pgfutil@tempdima}{0ex}}
        \pgfpathmoveto{\pgfqpoint{\pgf@xa}{\pgf@ya}}
        \pgfpathlineto{\pgfqpoint{\pgf@xb}{\pgf@ya}}
        \pgfpathlineto{\pgfqpoint{\pgf@xb}{\pgf@yb}}
        \pgfpathlineto{\pgfqpoint{\pgf@xa}{\pgf@yb}}
        \pgfpathclose
        \pgfpathmoveto{\pgfpointadd{\pgfpoint{\pgf@xa}{\pgf@yb}}{\ppd@offsetm}}
        \pgfpathlineto{\pgfpointadd{\pgfpoint{\pgf@xa}{\pgf@ya}}{\ppd@offsetm}}
        \pgfpathlineto{\pgfpointadd{\pgfpoint{\pgf@xb}{\pgf@ya}}{\ppd@offset}}
        \pgfpathlineto{\pgfpointadd{\pgfpoint{\pgf@xb}{\pgf@yb}}{\ppd@offset}}
        \pgfpathclose
    }
}
\pgfdeclareshape{cyclebegshape}{
    \inheritsavedanchors[from=rectangle]
    \inheritanchorborder[from=rectangle]
    \inheritanchor[from=rectangle]{north}
    \inheritanchor[from=rectangle]{center}
    \inheritanchor[from=rectangle]{west}
    \inheritanchor[from=rectangle]{east}
    \inheritanchor[from=rectangle]{mid}
    \inheritanchor[from=rectangle]{base}
    \inheritanchor[from=rectangle]{south}
    \backgroundpath{
        \southwest \pgf@xa=\pgf@x \pgf@ya=\pgf@y
        \northeast \pgf@xb=\pgf@x \pgf@yb=\pgf@y
        \pgfmathsetlength\pgfutil@tempdima{\pgfkeysvalueof{/pgf/cycleshape w}}
        \pgfpathmoveto{\pgfqpoint{\pgf@xa}{\pgf@ya}}
\pgfpathlineto{\pgfpointadd{\pgfpoint{\pgf@xa}{\pgf@yb}}{\pgfpoint{0ex}{-\pgfutil@tempdima}}}
\pgfpathlineto{\pgfpointadd{\pgfpoint{\pgf@xa}{\pgf@yb}}{\pgfpoint{\pgfutil@tempdima}{0ex}}}
\pgfpathlineto{\pgfpointadd{\pgfpoint{\pgf@xb}{\pgf@yb}}{\pgfpoint{-\pgfutil@tempdima}{0ex}}}
\pgfpathlineto{\pgfpointadd{\pgfpoint{\pgf@xb}{\pgf@yb}}{\pgfpoint{0ex}{-\pgfutil@tempdima}}}
\pgfpathlineto{\pgfqpoint{\pgf@xb}{\pgf@ya}}
        \pgfpathclose
    }
}
\pgfdeclareshape{cycleendshape}{
    \inheritsavedanchors[from=rectangle]
    \inheritanchorborder[from=rectangle]
    \inheritanchor[from=rectangle]{north}
    \inheritanchor[from=rectangle]{center}
    \inheritanchor[from=rectangle]{west}
    \inheritanchor[from=rectangle]{east}
    \inheritanchor[from=rectangle]{mid}
    \inheritanchor[from=rectangle]{base}
    \inheritanchor[from=rectangle]{south}
    \backgroundpath{
        \southwest \pgf@xa=\pgf@x \pgf@ya=\pgf@y
        \northeast \pgf@xb=\pgf@x \pgf@yb=\pgf@y
        \pgfmathsetlength\pgfutil@tempdima{\pgfkeysvalueof{/pgf/cycleshape w}}
        \pgfpathmoveto{\pgfqpoint{\pgf@xb}{\pgf@yb}}
\pgfpathlineto{\pgfpointadd{\pgfpoint{\pgf@xb}{\pgf@ya}}{\pgfpoint{0ex}{\pgfutil@tempdima}}}
\pgfpathlineto{\pgfpointadd{\pgfpoint{\pgf@xb}{\pgf@ya}}{\pgfpoint{-\pgfutil@tempdima}{0ex}}}
\pgfpathlineto{\pgfpointadd{\pgfpoint{\pgf@xa}{\pgf@ya}}{\pgfpoint{\pgfutil@tempdima}{0ex}}}
\pgfpathlineto{\pgfpointadd{\pgfpoint{\pgf@xa}{\pgf@ya}}{\pgfpoint{0ex}{\pgfutil@tempdima}}}
\pgfpathlineto{\pgfqpoint{\pgf@xa}{\pgf@yb}}
        \pgfpathclose
    }
}
\makeatother
\tikzstyle{subroutine} =
    [
        base,
        subrtshape,
    ]
\tikzstyle{cyclebegin} =
    [
        base,
        cyclebegshape,
    ]
\tikzstyle{cycleend} =
    [
        base,
        cycleendshape,
    ]
\tikzstyle{connector} =
    [
        base,
        circle,
    ]

% \small
% \footnotesize
\scriptsize
\renewcommand{\baselinestretch}{0.8}
\sf

\noindent
\resizebox{\linewidth}{!}{
% !TEX encoding = UTF-8 Unicode
% Úτƒ-8 encoded
% http://www.linux.org.ru/forum/general/10357036
% \begin{figure}
% \hspace{-4cm}
% \small
\begin{tikzpicture}[%
    start chain=main_vert going below,    % General flow is top-to-bottom
    start chain=main_horz going right,  
    start chain=rev_vert going above,    
    node distance=1.ex and 1em, % Global setup of box spacing
%     scale=0.7, 
%     every node/.style={scale=0.72}
every on chain/.style=join,
    ] 

        \tikzstyle{fitblock}=[inner sep = 0ex]
        \tikzstyle{shortline}=[draw, thin]
        \tikzstyle{longline}=[shortline,-latex']
        \tikzstyle{revline}=[shortline,latex'-]
        \tikzstyle{nodraw}=[draw=none]

        \tikzset{every join/.style=shortline}
        \node [disk file ] (sources)         [on chain=main_vert                   ] {  Файлы с~исходным кодом};
        \node [disk file ] (makefile)        [right  = of sources ] {  Makefile};
%         \node [prompt    ] (makefile path)   [right  = of makefile                             ] {  Путь до папки с~Makefile};
        \coordinate                          [on chain=main_vert] (main_from_makefile);
%         \coordinate[on chain=main_vert] (main_from_makefile path);
%         \coordinate                          [on chain=main_vert] (no_used);
        \tikzset{every join/.style=shortline} %longline
        \node [process   ] (builder)         [on chain=main_vert              ] {  Сборка};
        \tikzset{every join/.style=shortline}
                
        \node [disk file ] (build log)       [on chain=main_vert        ] {  Файл с~информацией о~сборке};
        \node [process   ] (static analyzer) [on chain=main_vert        ] {  Модуль статического анализа};
        \node [disk file ] (stat result)     [on chain=main_vert        ] {  Результаты статического анализа};
        \node [process   ] (stat parser)     [on chain=main_vert        ] {  Модуль преобразования результатов статического анализа};
        \node [disk file ] (stat json)       [on chain=main_vert                ] {  Преобразованные результаты статического анализа};
        \node [process   ] (aggregator)      [on chain=main_vert        ] {  Модуль агрегирования результатов линковки и~статического анализа};
        \node [disk file ] (aggregator file) [on chain=main_vert           ] {  Агрегированные результаты линковки и~статического анализа};
        \node [process   ] (comparer)        [on chain=main_vert           ] {  Модуль сравнительного анализа};
        \node [disk file ] (summary)         [on chain=main_vert           ] {  Результаты сравнительного анализа};
        \tikzset{every join/.style=nodraw}

        
        \draw [longline] (makefile)   |- (main_from_makefile);
%         \draw [longline] (makefile path)         |- (main_from_makefile path);

%         \draw [longline] (sources)         |- (static analyzer);
         \coordinate                          [left = of static analyzer] (static_from_sources);
        \draw [longline] (sources)     -| (static_from_sources)    -- (static analyzer);
       
        \node [disk file ] (call map)        [right =  of build log                ] {  Файл с~информацией о~линковке};
        \draw [longline] (builder)         -| (call map);
        \draw [longline] (call map)        |- (aggregator);
       
        \node[fit=(call map.north) (sources) (summary), fitblock] (left_vert_base) {};
        
        
        
        % самая широкая часть правой вертикали
        \tikzset{every join/.style=nodraw}
%         \coordinate                          [right = of stat parser.north east-|left_vert_base.east, on chain=main_horz] (right_vert_anchorpoint); % right = of stat result?
        \coordinate                          [right = of stat result.south-|left_vert_base.east, on chain=main_horz] (right_vert_anchorpoint); % right = of stat result?
        \node [disk file ] (gdb script)      [on chain=main_horz] {  Скрипт для GDB};
        \coordinate                          [on chain=main_horz] (center_from_breakpointer);
        \node [disk file ] (modified exe)    [on chain=main_horz] {  Модифицированный исполняемый файл};
        \node[fit=(gdb script) (modified exe), fitblock] (right_vert_cross) {};
       
        % вверх
        \node [process   ] (breakpointer)    [on chain=rev_vert, above = of right_vert_cross] {  Модуль бинарного анализа};
        \draw [longline] (breakpointer)    -| (modified exe);
        \draw [longline] (breakpointer)    -| (gdb script);
%         \tikzset{every join/.style=revline}
%         \tikzset{every join/.style=shortline} %revline
%         \coordinate                          [on chain=rev_vert] (executable_to_breakpointer);
        \tikzset{every join/.style=shortline}
        \node [disk file ] (file executable) [on chain=rev_vert   ] {  Исполняемый\\файл};
        \draw [longline] (builder)         -| (file executable);
        
% %         \node [prompt    ] (executable)      [right = of  file executable          ] {  Путь до исполняемого файла};
%         \node [prompt    ] (executable)      [above = of  file executable.north-|modified exe          ] {  Путь до исполняемого файла};
%         \draw [revline] (executable_to_breakpointer)    -| (executable);            
        
        % вниз
        \tikzset{every join/.style=nodraw}
        \node [process   ] (gdb manager)     [on chain=main_vert, below = of right_vert_cross   ] {  Модуль управления отладчиком};
        \draw [longline] (modified exe)    |- (gdb manager);
        \draw [longline] (gdb script)      |- (gdb manager);

        \tikzset{every join/.style=shortline}
        \node [disk file ] (dyn result)      [on chain=main_vert                ] {  Результаты динамического анализа};
        \node [process   ] (dyn parser)      [on chain=main_vert                ] {  Модуль преобразования результатов динамического анализа};
        \node [disk file ] (dyn json)        [on chain=main_vert                ] {  Преобразованные результаты динамического анализа};
   
        \draw [longline] (dyn json)        |- (comparer);
     
        
       
% другой вариант, по раскладке ближе к исходному       
%         \coordinate[right = of sources, on chain=main_horz] (right_vert_anchorpoint);
%         \node [disk file ] (file executable) [on chain=main_vert, right = of right_vert_anchorpoint             ] {  Исполняемый\\файл};
%         \tikzset{every join/.style=shortline}
%         \coordinate[on chain=main_vert] (executable_to_breakpointer);
%         \node [prompt    ] (executable)      [right = of  file executable          ] {  Путь до исполняемого файла};
%         \draw [revline] (executable_to_breakpointer)    -| (executable);
% 
%         
%         \tikzset{every join/.style=longline}
%         \node [process   ] (breakpointer)    [on chain=main_vert   ] {  Модуль бинарного анализа};% решает последнее
%         \tikzset{every join/.style=nodraw}
%         \coordinate[on chain=main_vert] (center_from_breakpointer);
% %         \node[circle, fill=red] at  (center_from_breakpointer) {};
%       
%         
% %         \node [disk file ] (modified exe)    [right = of center_from_breakpointer, anchor = north west] {  Модифицированный исполняемый файл}; у~цилиндра north west почти на north
%         \node [disk file ] (modified exe)    [right = of center_from_breakpointer, anchor = after bottom] {  Модифицированный исполняемый файл};
%         \node [disk file ] (gdb script)      [left = of center_from_breakpointer|-modified exe] {  Скрипт для GDB};
%         \node [process   ] (gdb manager)     [on chain=main_vert, below = of breakpointer|-modified exe.south              ] {  Модуль управления отладчиком};
%         \tikzset{every join/.style=shortline}
%         \node [disk file ] (dyn result)      [on chain=main_vert                ] {  Результаты динамического анализа};
%         \node [process   ] (dyn parser)      [on chain=main_vert                ] {  Модуль преобразования результатов динамического анализа};
%         \node [disk file ] (dyn json)        [on chain=main_vert                ] {  Преобразованные результаты динамического анализа};
%         
%         
% %         \draw [line] (builder)         -| (build log);
% % 
% % %         \draw [line] (executable)      -| (breakpointer);
% % %         \draw [line] (file executable) -| (breakpointer);
%         \draw [longline] (breakpointer)    -| (modified exe);
%         \draw [longline] (breakpointer)    -| (gdb script);
% % 
%         \draw [longline] (modified exe)    |- (gdb manager);
%         \draw [longline] (gdb script)      |- (gdb manager);
% % %         \draw [-] (gdb manager)        -- (dyn result);
% % %         \draw [-] (dyn result)         -- (dyn parser);
% % %         \draw [-] (dyn parser)         -- (dyn json);
% % 
% % 
% % %         \draw [-] (build log)          -- (static analyzer);
% % %         \draw [-] (static analyzer)    -- (stat result);
% % %         \draw [-] (stat result)        -- (stat parser);
% % %         \draw [-] (stat parser)        -- (stat json);
% % %         \draw [-] (stat json)          -- (aggregator);
% % %         \draw [-] (aggregator)         -- (aggregator file);
% % 
% % %         \draw [-] (aggregator file)    -- (comparer);
%         \draw [longline] (dyn json)        |- (comparer);
% % %         \draw [-] (comparer)           -- (summary);
        
        
% контроль полей
%         \draw [red] (current bounding box.south east) rectangle (current bounding box.north west);
\end{tikzpicture}
}

    \end{figure}
\end{frame}

\begin{frame}%[plain, noframenumbering, t]
\frametitle{Схема алгоритма {\ProgModule}}

    \Put(149,-414){\includegraphics[height=2cm]{Presentation/images/algo_gost.png}}
    %\begin{figure}[!htbp]
    %    \begin{adjustbox}{max totalsize={1.0\textwidth}{0.8\textheight},left}
    %        \includegraphics{}
    %    \end{adjustbox}
    %\end{figure}

    \begin{figure}[!htbp]
        \vspace{-9ex} 
        \hspace{-13cm} 
        \begin{adjustbox}{max totalsize={1.6\textwidth}{1.6\textheight},left}
            % !TEX encoding = UTF-8 Unicode
% Úτƒ-8 encoded
% http://www.linux.org.ru/forum/general/10357036
\tikzset{
    line/.style={draw, -latex'},
    u/.style={anchor=south},
    r/.style={anchor=west},
    fxd/.style={text width = 6em},
    it/.style={font={\itshape}},
    bf/.style={font={\bfseries}}
}
\tikzstyle{base_long} =
    [
        draw,
        text badly centered,
        text width=12em,
        minimum height=3ex,
        inner xsep = 1pt,
        inner ysep = 3pt,
        minimum width = 10em,
    ]
\tikzstyle{base} =
    [
        draw,
        text badly centered,
        text width=12em,
        minimum height=3ex,
        inner xsep = 1pt,
        inner ysep = 3pt,
    ]
\tikzstyle{coord} =
    [
        coordinate,
    ]
\tikzstyle{cloud} =
    [
        base,
        ellipse,
        text width=5em,
    ]
\tikzstyle{decision} =
    [
        base,
        diamond,
        aspect=2,
        inner sep = 0pt
    ]
\tikzstyle{block} =
    [
        rectangle,
        base,
        rounded corners,
    ]
\tikzstyle{print_block} =
    [
        base,
        tape,
        tape bend top=none,
    ]
\tikzstyle{io} =
    [
        base,
        trapezium,
        trapezium left angle = 70,
        trapezium right angle = 110,
    ]
\tikzstyle{prompt} =
    [
        base,
        trapezium,
        trapezium left angle = 90,
        trapezium right angle = 80,
        shape border rotate = 90
    ]
\tikzstyle{disk file} =
    [
        base,
        cylinder,
        aspect=0.2,
    ]
\tikzstyle{process} =
    [
        rectangle,
        base,
    ]
\makeatletter
\pgfkeys{/pgf/.cd,
    subrtshape w/.initial=2mm,
    cycleshape w/.initial=2mm
}
\pgfdeclareshape{parallelshape}{
    \inheritsavedanchors[from=rectangle]
    \inheritanchorborder[from=rectangle]
    \inheritanchor[from=rectangle]{north}
    \inheritanchor[from=rectangle]{center}
    \inheritanchor[from=rectangle]{west}
    \inheritanchor[from=rectangle]{east}
    \inheritanchor[from=rectangle]{mid}
    \inheritanchor[from=rectangle]{base}
    \inheritanchor[from=rectangle]{south}
    \backgroundpath{
        \southwest \pgf@xa=\pgf@x \pgf@ya=\pgf@y
        \northeast \pgf@xb=\pgf@x \pgf@yb=\pgf@y
        \def\ppd@offset{\pgfpoint{\pgfutil@tempdima}{0ex}}
        \def\ppd@offsetm{\pgfpoint{-\pgfutil@tempdima}{0ex}}
        \pgfpathmoveto{\pgfqpoint{\pgf@xa}{\pgf@ya}}
            \pgfpathlineto{\pgfqpoint{\pgf@xb}{\pgf@ya}}
        \pgfpathclose
        \pgfpathmoveto{\pgfqpoint{\pgf@xb}{\pgf@yb}}
            \pgfpathlineto{\pgfqpoint{\pgf@xa}{\pgf@yb}}
        \pgfpathclose
    }
}
\pgfdeclareshape{subrtshape}{
    \inheritsavedanchors[from=rectangle]
    \inheritanchorborder[from=rectangle]
    \inheritanchor[from=rectangle]{north}
    \inheritanchor[from=rectangle]{center}
    \inheritanchor[from=rectangle]{west}
    \inheritanchor[from=rectangle]{east}
    \inheritanchor[from=rectangle]{mid}
    \inheritanchor[from=rectangle]{base}
    \inheritanchor[from=rectangle]{south}
    \backgroundpath{
        \southwest \pgf@xa=\pgf@x \pgf@ya=\pgf@y
        \northeast \pgf@xb=\pgf@x \pgf@yb=\pgf@y
        \pgfmathsetlength\pgfutil@tempdima{\pgfkeysvalueof{/pgf/subrtshape w}}
        \def\ppd@offset{\pgfpoint{\pgfutil@tempdima}{0ex}}
        \def\ppd@offsetm{\pgfpoint{-\pgfutil@tempdima}{0ex}}
        \pgfpathmoveto{\pgfqpoint{\pgf@xa}{\pgf@ya}}
        \pgfpathlineto{\pgfqpoint{\pgf@xb}{\pgf@ya}}
        \pgfpathlineto{\pgfqpoint{\pgf@xb}{\pgf@yb}}
        \pgfpathlineto{\pgfqpoint{\pgf@xa}{\pgf@yb}}
        \pgfpathclose
        \pgfpathmoveto{\pgfpointadd{\pgfpoint{\pgf@xa}{\pgf@yb}}{\ppd@offsetm}}
        \pgfpathlineto{\pgfpointadd{\pgfpoint{\pgf@xa}{\pgf@ya}}{\ppd@offsetm}}
        \pgfpathlineto{\pgfpointadd{\pgfpoint{\pgf@xb}{\pgf@ya}}{\ppd@offset}}
        \pgfpathlineto{\pgfpointadd{\pgfpoint{\pgf@xb}{\pgf@yb}}{\ppd@offset}}
        \pgfpathclose
    }
}
\pgfdeclareshape{cyclebegshape}{
    \inheritsavedanchors[from=rectangle]
    \inheritanchorborder[from=rectangle]
    \inheritanchor[from=rectangle]{north}
    \inheritanchor[from=rectangle]{center}
    \inheritanchor[from=rectangle]{west}
    \inheritanchor[from=rectangle]{east}
    \inheritanchor[from=rectangle]{mid}
    \inheritanchor[from=rectangle]{base}
    \inheritanchor[from=rectangle]{south}
    \backgroundpath{
        \southwest \pgf@xa=\pgf@x \pgf@ya=\pgf@y
        \northeast \pgf@xb=\pgf@x \pgf@yb=\pgf@y
        \pgfmathsetlength\pgfutil@tempdima{\pgfkeysvalueof{/pgf/cycleshape w}}
        \pgfpathmoveto{\pgfqpoint{\pgf@xa}{\pgf@ya}}
\pgfpathlineto{\pgfpointadd{\pgfpoint{\pgf@xa}{\pgf@yb}}{\pgfpoint{0ex}{-\pgfutil@tempdima}}}
\pgfpathlineto{\pgfpointadd{\pgfpoint{\pgf@xa}{\pgf@yb}}{\pgfpoint{\pgfutil@tempdima}{0ex}}}
\pgfpathlineto{\pgfpointadd{\pgfpoint{\pgf@xb}{\pgf@yb}}{\pgfpoint{-\pgfutil@tempdima}{0ex}}}
\pgfpathlineto{\pgfpointadd{\pgfpoint{\pgf@xb}{\pgf@yb}}{\pgfpoint{0ex}{-\pgfutil@tempdima}}}
\pgfpathlineto{\pgfqpoint{\pgf@xb}{\pgf@ya}}
        \pgfpathclose
    }
}
\pgfdeclareshape{cycleendshape}{
    \inheritsavedanchors[from=rectangle]
    \inheritanchorborder[from=rectangle]
    \inheritanchor[from=rectangle]{north}
    \inheritanchor[from=rectangle]{center}
    \inheritanchor[from=rectangle]{west}
    \inheritanchor[from=rectangle]{east}
    \inheritanchor[from=rectangle]{mid}
    \inheritanchor[from=rectangle]{base}
    \inheritanchor[from=rectangle]{south}
    \backgroundpath{
        \southwest \pgf@xa=\pgf@x \pgf@ya=\pgf@y
        \northeast \pgf@xb=\pgf@x \pgf@yb=\pgf@y
        \pgfmathsetlength\pgfutil@tempdima{\pgfkeysvalueof{/pgf/cycleshape w}}
        \pgfpathmoveto{\pgfqpoint{\pgf@xb}{\pgf@yb}}
\pgfpathlineto{\pgfpointadd{\pgfpoint{\pgf@xb}{\pgf@ya}}{\pgfpoint{0ex}{\pgfutil@tempdima}}}
\pgfpathlineto{\pgfpointadd{\pgfpoint{\pgf@xb}{\pgf@ya}}{\pgfpoint{-\pgfutil@tempdima}{0ex}}}
\pgfpathlineto{\pgfpointadd{\pgfpoint{\pgf@xa}{\pgf@ya}}{\pgfpoint{\pgfutil@tempdima}{0ex}}}
\pgfpathlineto{\pgfpointadd{\pgfpoint{\pgf@xa}{\pgf@ya}}{\pgfpoint{0ex}{\pgfutil@tempdima}}}
\pgfpathlineto{\pgfqpoint{\pgf@xa}{\pgf@yb}}
        \pgfpathclose
    }
}
\makeatother
\tikzstyle{subroutine} =
    [
        base,
        subrtshape,
    ]
\tikzstyle{cyclebegin} =
    [
        base,
        cyclebegshape,
    ]
\tikzstyle{cycleend} =
    [
        base,
        cycleendshape,
    ]
\tikzstyle{connector} =
    [
        base,
        circle,
    ]

\tikzstyle{parallel} =
    [
        base_long,
        parallelshape,
        text width=43em,
    ]
\begin{tikzpicture}[%
    start chain=main going below,    % General flow is top-to-bottom
    start chain=left_vert  going below,    % General flow is top-to-bottom
    start chain=right_vert going below,  
    node distance=2.ex and 1em, % Global setup of box spacing
    %every on chain/.style=join,
    ] 
        \node [cloud      ] (makefile)        [on chain=main] {Начало};
        \node [subroutine ] (builder)         [on chain=main, join] {Сборка};
        \node [parallel   ] (parallel)        [yscale=0.3, on chain=main, join] {\ };

        \node [subroutine ] (static analyzer) at (-15em,-15ex) [on chain=left_vert] {Статический анализ}; 
        \node [subroutine ] (stat parser)     [on chain=left_vert, join] {Преобразование результатов статического анализа};
        \node [subroutine ] (aggregator)      [on chain=left_vert, join] {Агрегирование результатов линковки и статического анализа};

        \node [subroutine ] (breakpointer) at (15em,-15ex)   [on chain=right_vert] {Бинарный анализ};
        \node [subroutine ] (gdb manager)     [on chain=right_vert, join] {Динамический анализ};
        \node [subroutine ] (dyn parser)      [on chain=right_vert, join] {Преобразование результатов динамического анализа};

        \node [parallel   ] (parallel aggr)   [yshift=-28ex, yscale=0.3,on chain=main] {\ };
        \node [subroutine ] (comparer)        [on chain=main,join] {Сравнительный анализ};
        \node [cloud      ] (end)             [on chain=main, join] {Конец};


        \coordinate (upper left)  at (-15em, -10.52ex);
        \coordinate (upper right) at (15em,  -10.52ex);

        \draw [-] (parallel) -- (upper left)  -- (static analyzer);
        \draw [-] (parallel) -- (upper right) -- (breakpointer);

        \coordinate (lower left)  at (-15em, -40.52ex);
        \coordinate (lower right) at (15em,  -40.52ex);

        \draw [-] (aggregator) -- (lower left)  -- (parallel aggr);
        \draw [-] (dyn parser) -- (lower right) -- (parallel aggr);


\end{tikzpicture}

        \end{adjustbox}
        %\caption{Схема алгоритма {\ProgModule}\label{fig:algorithm}}
    \end{figure}
\end{frame}

\begin{frame}%[plain, noframenumbering, t]
\frametitle{Пользовательский интерфейс {\ProgModule}}
    Пользователь может управлять {\ProgModule} как с помощью
    консольного интерфейса, так и графического.
    \begin{figure}[!htbp]
        \begin{adjustbox}{max totalsize={0.8\textwidth}{0.8\textheight}}
            \includegraphics[trim={0.2em 0.2ex 0.2ex 0.2ex},clip,width=\linewidth]{images/apndv-gui.png}
        \end{adjustbox}
        %\caption{Схема алгоритма {\ProgModule}\label{fig:algorithm}}
    \end{figure}

\end{frame}

\begin{frame}%[plain, noframenumbering, t]
\frametitle{Отладка и тестирование {\ProgModule}}
    В процессе разработки {\ProgModule} было написано 43 модульных теста, рассматривающих
    различные сценарии использования элементов {\ProgModule}. {\ProgModule} отлаживался
    с помощью отладчика GDB.
    
    \vspace{3ex}
    \begin{figure}[!htbp]
        \begin{adjustbox}{max totalsize={1.2\textwidth}{1.2\textheight}}
            \includegraphics[trim={0 24ex 0 0 0},clip,width=\linewidth]{images/running-gdb.png}
        \end{adjustbox}
        %\caption{Схема алгоритма {\ProgModule}\label{fig:algorithm}}
    \end{figure}

\end{frame}

\begin{frame}%[plain, noframenumbering, t]
\frametitle{Апробация}
    \begin{itemize}
        \item Уманский А.А. Разработка coreutils на языке Forth для встраиваемых систем.
            <<Актуальные проблемы информатизации в цифровой экономике и научных исследованиях>>
            Международная научно-практическая конференция 2019.
        \item Внедрение {\ProgModule} готовится в ООО Фирма <<Анкад>>.
    \end{itemize}
    \begin{figure}[!htbp]
        \begin{adjustbox}{max totalsize={0.6\textwidth}{0.6\textheight}}
            \includegraphics[trim={0 4ex 0 0 0},clip,width=\linewidth]{Presentation/images/conference.jpg}
        \end{adjustbox}
        %\caption{Схема алгоритма {\ProgModule}\label{fig:algorithm}}
    \end{figure}
\end{frame}

\begin{frame}%[plain, noframenumbering, t]
\frametitle{Результаты работы}
    \begin{itemize}
        \item исследована предметная область;
        \item проведен сравнительный анализ существующих программных решений;
        \item выборан язык и среда разработки;
        \item разработана схема данных {\ProgModule};
        \item разработана схема алгоритма {\ProgModule};
        \item программно реализован {\ProgModule};
        \item проведена отладка и тестирование {\ProgModule};
        \item разработано руководство оператора к {\ProgModule}.
    \end{itemize}

    Поиск НДВ в сертифицируемом ПО унифицирован и проходит автоматически,
    что значительно ускоряет процесс сертификации.
    
    Цель ВКР достигнута.
\end{frame}

\begin{frame}%[plain, noframenumbering, t]

    \begin{center}
        \Huge Спасибо за внимание!
    \end{center}
    
\end{frame}
