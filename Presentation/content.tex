\begin{frame}%[plain, noframenumbering, t]
    \frametitle{Проблемная ситуация}

    Создание аппаратного обеспечения -- трудоемкий процесс,
    непременно сязанный с созданием прикладного программного обеспечения.
    В свою очередь, отсутствие возможности исполнения ПО с использованием аппаратного обеспечения
    серьезно усложняет отладку и тестирование конечного продукта.

    \textbf{Причины сложившейся ситуации:}
    \begin{itemize}
        \item невозможность обеспечить всех разработчиков прикладного ПО физическими устройствами;
        \item отсутствие автоматического или полуавтоматического ПО для эмуляции перефирийных устройств;
        \item неизбежное появление программных ошибок при создании собственного эмулятора.
    \end{itemize}
\end{frame}


\begin{frame}%[plain, noframenumbering, t]
    \frametitle{Цель и задачи диссертации}
    \textbf{Цель:} снижение трудоемкости создания виртуальных устройств.

    \textbf{Задачи:}
    \begin{itemize}
        \item аналитический обзор существующих методов создания виртуального аппаратного обеспечения;
        \item анализ существующих подходов к разработке прикладного ПО при отсутствии аппаратного обеспечения;
        \item декомпозиция поставленной задачи для создания методики и алгоритма генерации виртуального аппаратного обеспечения;
        \item создание методики и алгоритма генерации виртуального аппаратного обеспечения на основе его характеристик;
        \item формализация задачи;
        \item разработка лингвистического аппарата (семантика, синтаксис) языка для создания программ по генерации виртуального
            аппаратного обеспечения.
    \end{itemize}
\end{frame}


\begin{frame}%[plain, noframenumbering, t]
    \frametitle{Положения, выносимые на защиту}
    \begin{enumerate}
        \item Формализованное представление алгоритма генерации виртуального аппаратного обеспечения;
        \item Алгоритм генерации виртуального аппаратного обеспечения;
        \item Лингвистический аппарат (синтаксис, семантика) языка для создания программ по генерации виртуального
            аппаратного обеспечения;
    \end{enumerate}
\end{frame}


\begin{frame}%[plain, noframenumbering, t]
    \frametitle{Формализованное представление}
    \begin{figure}[!htbp]
        \begin{adjustbox}{max totalsize={\textwidth}{\textheight}}
            \begin{minipage}{\linewidth}
                \begin{grammar}
                    <letter> ::= `a' ... `z' | `A' ... `Z'; \newline
                    <digit> ::= `0' ... `9' ; \newline
                    <symbol> ::= \symbol{92}x20 ... \symbol{92}x7E ; (* любой печатный символ, согласно кодам ASCII *) \newline
                    <const value> ::= <digit> | `"' \{ <symbol> \} `"'; \newline
                    <identifier> ::= <letter> [\{ <letter> | <digit> | `\_' \}] ; \newline
                    <block start> ::= `{'; \newline
                    <block end> ::= `}'; \newline
                    <field> ::= <identifier> `=' <identifier> | <block> ; \newline
                    <block> ::= <block start> <field> [\{ `,' <field> \}] <block end>; \newline
                    <device definition> ::= '\#' <identifier>; \newline
                    <device class inheritance> ::= `(' <identifier> `:' <identifier> [\{ `,' <identifier> \}] `)'; \newline
                    <device class block> ::= <device class inheritance> <block>; \newline
                    <bind block> ::= `@bind' <block>; \newline
                    <python block> ::= `@py' <block>; \newline
                    <program> ::= <device definition> <device class block> <bind block> <python block>;
                \end{grammar}
            \end{minipage}
        \end{adjustbox}
        \caption{Расширенная форма Бэкуса-Наура \mylanguage}\label{fig:qpydev-grammar}
    \end{figure}
\end{frame}


\begin{frame}%[plain, noframenumbering, t]
    \frametitle{Формализованное представление (продолжение)}
    \begin{figure}[!htbp]
        \begin{adjustbox}{max totalsize={\textwidth}{\textheight}}
        \begin{minipage}{\linewidth}
                \centering
                \begingroup
                \begin{gather*}
                    [[assignment]](x,y) = \lambda x.y \\
                    [[terminate]](m) \text{ Завершение работы компилятора} \\
                    [[if]](c,e_1,e_2) =
                    \begin{cases}
                        e_1, & \text{Если } c = true \\
                        e_2, & \text{Если } c \not= true
                    \end{cases} \\
                    [[throw\ error]](c, e) = if(c, e_g, terminate) \\
                    [[lookup]](o) = [[throw\ error]](o \in Q, o) \\
                    [[<device\ definition>]](i) = lookup(i) \\
                    [[<device\ class\ inheritance>]](i_1,...,i_n) = lookup(i_1) + ... + lookup(i_n) \\
                    [[<field>]](v_1, v_2) = assignment(v_1, v_2), \text{ Если } v_1 \in Q \text{ и } v_3 \in C \cup Q \\
                    [[<block>]](f_1,...,f_n) = field(f_1) + ... + field(f_n) \\
                    [[<python block>]](b) = assignment(B,B)\\
                \end{gather*}
                \endgroup
            \end{minipage}
        \end{adjustbox}
        \caption{Денотационная семантика языка {\mylanguage}}\label{fig:denotational-semantics}
    \end{figure}
\end{frame}%[plain, noframenumbering, t]

\begin{frame}%[plain, noframenumbering, t]
    \frametitle{Методика создания полнофункционального виртуального устройства}
    \begin{enumerate}
        \item С помощью разработанного языка описываются интерфейсы и характеристики устройства;
        \item Компилятор для разработанного языка преобразует данное описание в код на языке C;
        \item Пользователь заполняет в полученном файле функции обработки данных;
        \item Данный файл с описанием устройства на языке C встраивается как новое устройство в эмулятор QEMU;
    \end{enumerate}
\end{frame}


\begin{frame}%[plain, noframenumbering, t]
    \frametitle{Выбор метрики оценки эффективности}
    Основные метрики эффективности:
    \begin{enumerate}
        \item Время разработки виртуального устройства (в человеко-часах)
        \item Качество сгенерированного кода на языке C
    \end{enumerate}
\end{frame}


\begin{frame}%[plain, noframenumbering, t]
    \frametitle{Основные результаты диссертационной работы}
    \begin{enumerate}
        \item Проведен аналитический обзор существующих методов создания виртуального аппаратного обеспечения;
        \item Проведен анализ существующих подходов к эмуляции аппаратного обеспечения;
        \item Проведен анализ актуальности существующих подходов к разработке прикладного ПО при отсутствии аппаратного обеспечения;
        \item Произведена декомпозиция поставленной задачи для создания методики и алгоритма генерации виртуального аппаратного обеспечения;
        \item Разработан лингвистический аппарата (семантика, синтаксис) языка для создания программ по генерации виртуального
            аппаратного обеспечения;
    \end{enumerate}
\end{frame}


\begin{frame}%[plain, noframenumbering, t]
    \begin{center}
        \Huge Спасибо за внимание!
    \end{center}
\end{frame}
