%\subsection{Исследование предметной области}
\begin{frame}%[plain, noframenumbering, t, shrink=20]
    \begin{center}
        \textbf{Объект исследования:}

        Существующие методики создания виртуального аппаратного
        обеспечения.

        \textbf{Предмет исследования:}

        Программное обеспечение.
    \end{center}
\end{frame}


\begin{frame}%[plain, noframenumbering, t]
    \frametitle{Проблемная ситуация}

    Высокая трудоемкость и дороговизна создания прикладного ПО, ориентированного
    на использование конкретного физического устройства.

    \textbf{Причины сложившейся ситуации:}
    \begin{itemize}
        \item невозможность обеспечить всех разработчиков прикладного ПО физическими устройствами;
        \item отсутствие автоматического или полуавтоматического ПО для эмуляции физических устройств;
        \item неизбежное появление программных ошибок в "рукописном" устройстве;
    \end{itemize}
\end{frame}


\begin{frame}%[plain, noframenumbering, t]
    \frametitle{Цели и задачи диссертации}
    \textbf{Цель:} создание методики и алгоритма генерации виртуального аппаратного обеспечения на основе его характеристик.

    \textbf{Задачи:}
    \begin{itemize}
        \item аналитический обзор существующих методов создания виртуального аппаратного обеспечения;
        \item анализ существующих подходов к эмуляции аппаратного обеспечения;
        \item анализ актуальности существующих подходов к разработке прикладного ПО при отсутствии аппаратного обеспечения;
        \item декомпозиция поставленной задачи для создания методики и алгоритма генерации виртуального аппаратного обеспечения;
        \item разработка лингвистического аппарата (семантика, синтаксис) языка для создания программ по генерации виртуального
            аппаратного обеспечения;
    \end{itemize}
\end{frame}


\begin{frame}%[plain, noframenumbering, t]
    \frametitle{Положения, выносимые на защиту}
    \begin{enumerate}
        \item Формализованное представление алгоритма генерации виртуального аппаратного обеспечения;
        \item Алгоритма генерации виртуального аппаратного обеспечения;
        \item Программная реализация разработанных методики и алгоритма;
        \item Лингвистический аппарат (синтаксис, семантика) языка для создания программ по генерации виртуального
            аппаратного обеспечения;
    \end{enumerate}
\end{frame}


\begin{frame}%[plain, noframenumbering, t]
    \frametitle{Анализ существующих средств макетирования реальных устройств}
    \begin{table}[!htbp]
        {\footnotesize
            \setlength{\tabcolsep}{2pt}
            \begin{longtable}{*{4}{| c}|}
                \hline
                \diagbox[width=4cm]{Свойства}{Название\\программы}                                       &
                \makecell{umockdev} &
                \makecell{Verilator}                       &
                \makecell{GHDL} \\
                \hline
                \makecell{Кроссплатформенность}             & \greencell{Да} & \greencell{Да} & \greencell{Да}\\
                \hline
                \makecell{Открытость\\исходного кода}        & \greencell{Да} & \greencell{Да}   & \greencell{Да}\\
                \hline
                \makecell{Бесплатность}                      & \greencell{Да} & \greencell{Да}   & \greencell{Да}\\
                \hline
                \makecell{Анализ языков\\описания аппаратуры} & \redcell{Нет} & \greencell{Да}   & \greencell{Да}\\
                \hline
                \makecell{Создание виртуального\\устройства в системе} & \yellowcell{Есть воспроизведение поведения} & \redcell{Нет}   & \redcell{Нет}\\
                \hline
                \makecell{Графический интерфейс}             & Нет & Нет & Нет \\
                \hline
            \end{longtable}
        }
    \end{table}
\end{frame}


\begin{frame}%[plain, noframenumbering, t]
    \frametitle{Формализованное представление}
\end{frame}


\begin{frame}%[plain, noframenumbering, t]
    \frametitle{Методика создания полнофункционального виртуального устройства}
    Из описания аппаратуры:
    \begin{enumerate}
        \item С помощью Verilator или GDHL генерируется модель устройства;
        \item Из модели удаляется все, кроме общения устройства с внешним миром и обработки этой информации;
        \item Полученная модель встраивается как новое устройство в эмулятор QEMU;
    \end{enumerate}
    Полуавтоматически:
    \begin{enumerate}
        \item С помощью разработанного языка описываются интерфейсы и характеристики устройства;
        \item Компилятор для разработанного языка преобразует данное описание в код на языке C;
        \item Пользователь заполняет в полученном файле функции обработки данных;
        \item Данный файл с описанием устройства на языке C встраивается как новое устройство в эмулятор QEMU;
    \end{enumerate}
\end{frame}


\begin{frame}%[plain, noframenumbering, t]
    \begin{center}
        \Huge Спасибо за внимание!
    \end{center}
\end{frame}
