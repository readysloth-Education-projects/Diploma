%\subsection{Исследование предметной области}
\begin{frame}%[plain, noframenumbering, t, shrink=20]
    \textbf{Объект исследования:}

    Существующие методики поиска НДВ в программном обеспечении.

    \textbf{Предмет исследования:}

    Программное обеспечение.
\end{frame}


\begin{frame}%[plain, noframenumbering, t]
\frametitle{Проблемная ситуация}

Большое количество ложноположительных и ложноотрицательных срабатываний при проведении анализа ПО на НДВ не отвечает современным требованиям безопасности.
\end{frame}

\begin{frame}%[plain, noframenumbering, t]
\frametitle{Причины сложившейся ситуации:}
\begin{itemize}
    \item у современных интерфейсов, как программных, так и пользовательских, большая «поверхность» для атаки;
    \item современные компиляторы производят большое количество изменений кода, таких как: встраивание тел функций, разворачивание циклов, объединение функций;
    \item требования и методики нахождения НДВ в ПО разрабатывались во времена с другим уровнем и сложностью технологий;
    \item новые атаки на ПО появляются постоянно, нет гибкого механизма их диагностирования, который соответствовал бы текущему уровню развития технологий.
\end{itemize}
\end{frame}

\begin{frame}%[plain, noframenumbering, t]
    \begin{center}
        \Huge Спасибо за внимание!
    \end{center}
\end{frame}
