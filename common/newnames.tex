% Новые переменные, которые могут использоваться во всём проекте
% ГОСТ 7.0.11-2011
% 9.2 Оформление текста автореферата диссертации
% 9.2.1 Общая характеристика работы включает в себя следующие основные структурные
% элементы:
% актуальность темы исследования;
\newcommand{\actualityTXT}{Актуальность темы.}
% степень ее разработанности;
\newcommand{\progressTXT}{Степень разработанности темы.}
% цели и задачи;
\newcommand{\aimTXT}{Целью}
\newcommand{\tasksTXT}{задачи}
% научную новизну;
\newcommand{\noveltyTXT}{Научная новизна:}
% теоретическую и практическую значимость работы;
%\newcommand{\influenceTXT}{Теоретическая и практическая значимость}
% или чаще используют просто
\newcommand{\influenceTXT}{Практическая значимость}
% методологию и методы исследования;
\newcommand{\methodsTXT}{Методология и методы исследования.}
% положения, выносимые на защиту;
\newcommand{\defpositionsTXT}{Основные положения, выносимые на~защиту:}
% степень достоверности и апробацию результатов.
\newcommand{\reliabilityTXT}{Достоверность}
\newcommand{\probationTXT}{Апробация работы.}

\newcommand{\contributionTXT}{Личный вклад.}
\newcommand{\publicationsTXT}{Публикации.}


%%% Заголовки библиографии:

% для автореферата:
\newcommand{\bibtitleauthor}{Публикации автора по теме диссертации}

% для стиля библиографии `\insertbiblioauthorgrouped`
\newcommand{\bibtitleauthorvak}{В изданиях из списка ВАК РФ}
\newcommand{\bibtitleauthorscopus}{В изданиях, входящих в международную базу цитирования Scopus}
\newcommand{\bibtitleauthorwos}{В изданиях, входящих в международную базу цитирования Web of Science}
\newcommand{\bibtitleauthorother}{В прочих изданиях}
\newcommand{\bibtitleauthorconf}{В сборниках трудов конференций}

% для стиля библиографии `\insertbiblioauthorimportant`:
\newcommand{\bibtitleauthorimportant}{Наиболее значимые \protect\MakeLowercase\bibtitleauthor}

% для списка литературы в диссертации и списка чужих работ в автореферате:
\newcommand{\bibtitlefull}{Список литературы} % (ГОСТ Р 7.0.11-2011, 4)

\newcommand{\ProgModule}{ПМ АПНДВ} % Название программного модуля

\newcommand\diag[4]{%
    \multicolumn{1}{p{#2}|}{
        \hskip-\tabcolsep
        $\vcenter{
            \begin{tikzpicture}[baseline=0,anchor=south west,inner sep=#1]
                \path[use as bounding box] (0,0) rectangle (#2+2\tabcolsep,\baselineskip);
                \node[minimum width={#2+2\tabcolsep},minimum height=\baselineskip+\extrarowheight] (box) {};
                \draw (box.north west) -- (box.south east);
                \node[anchor=south west] at (box.south west) {#3};
                \node[anchor=north east] at (box.north east) {#4};
            \end{tikzpicture}
        }$\hskip-\tabcolsep
   }
}

\newcommand{\redcell}[1]{
    \cellcolor{red!30} #1 }

\newcommand{\greencell}[1]{
    \cellcolor{green!30} #1 }

\newcommand{\yellowcell}[1]{
    \cellcolor{yellow!30} #1 }

\newcommand{\rptf}[2][1]{%
    \newcounter{loopcntr1}
    \forloop{loopcntr1}{0}{\value{loopcntr1}<#1}{#2}%
  }

\newcommand{\rpts}[2][1]{%
    \newcounter{loopcntr2}
    \forloop{loopcntr2}{0}{\value{loopcntr2}<#1}{#2}%
  }
