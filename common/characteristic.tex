\textbf{Актуальность исследования:}

Подтверждение безопасности программного обеспечения -- важный этап
в продвижении программного продукта.

Сертификация не является универсальным способом
решения всех существующих проблем в
области информационной безопасности, однако
сегодня это единственный реально функционирующий
механизм, который обеспечивает независимый
контроль качества средств защиты информации.
При грамотном применении механизм сертификации
позволяет достаточно успешно решать задачу
достижения гарантированного уровня защищенности автоматизированных систем.

Отсутствие недекларированных возможностей в скомпилированном
объектном файле является ключевым аспектом сертификации ПО.
Сертификация программного обеспечения необходима для
подтверждения требований заказчика к защите
информации, к выполнению функциональных
и технических задач и к обеспечению работы ПО в целом.


\textbf{Проблемная ситуация в области объекта исследований:}

Большое количество ложноположительных и ложноотрицательных срабатываний при проведении анализа ПО на НДВ не отвечает современным требованиям безопасности.

\textbf{Причины сложившейся ситуации:}
\begin{enumerate}[label={\arabic*)}]
    \item у современных интерфейсов, как программных, так и пользовательских, большая «поверхность» для атаки;
    \item современные компиляторы производят большое количество изменений кода, таких как: встраивание тел функций, разворачивание циклов, объединение функций;
    \item требования и методики нахождения НДВ в ПО разрабатывались во времена с другим уровнем и сложностью технологий;
    \item новые атаки на ПО появляются постоянно, нет гибкого механизма их диагностирования, который соответствовал бы текущему уровню развития технологий.
\end{enumerate}

\textbf{Объект исследования:}

Существующие методики поиска НДВ в программном обеспечении.

\textbf{Предмет исследования:}

Программное обеспечение.

\textbf{Цель исследования:}
уменьшение количества ложноположительных и ложноотрицательных срабатываний поиска НДВ.

\textbf{Задачи исследования:}
\begin{enumerate}[label={\arabic*)}]
    \item анализ возможностей нарушителя;
    \item анализ критичности и применимости воздействия нарушителя на ПО;
    \item создание методики выбора инструментов для проверки воздействия нарушителя на ПО;
    \item разработка алгоритма поиска НДВ;
\end{enumerate}
