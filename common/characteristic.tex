Сертификация -- процесс подтверждения соответствия характеристик товара определенным стандартам.

Разработка {\ProgModule} проводилась в ООО Фирма <<Анкад>>.
ООО Фирма <<Анкад>> занимается разработкой доверенных средств защиты информации и
средств криптографической защиты информации.

{\aim} данной работы является ускорение проведения процесса сертификации
программ, написанных на языках программирования C/C++.

Для~достижения поставленной цели необходимо решить следующие {\tasks}:
\begin{enumerate}[label={\arabic*)}]
    \item исследование предметной области;
    \item сравнительный анализ существующих программных решений;
    \item выбор языка и среды разработки;
    \item разработка схемы данных {\ProgModule};
    \item разработка схемы алгоритма {\ProgModule};
    \item программная реализация {\ProgModule};
    \item разработка пользовательского интерфейса;
    \item отладка и тестирование {\ProgModule};
    \item разработка руководства оператора {\ProgModule}.
\end{enumerate}


% {\novelty}
% \begin{enumerate}
%   \item Впервые \ldots
%   \item Впервые \ldots
%   \item Было выполнено оригинальное исследование \ldots
% \end{enumerate}

{\influence} проекта состоит в ускорении и унификации процесса сертификации ПО на отсутствие НДВ.


%{\methods} \ldots

%{\defpositions}
%\begin{enumerate}
%  \item Первое положение
%\end{enumerate}
%В папке Documents можно ознакомиться в решением совета из Томского ГУ
%в~файле \verb+Def_positions.pdf+, где обоснованно даются рекомендации
%по~формулировкам защищаемых положений.

% {\reliability} полученных результатов обеспечивается \ldots \ Результаты находятся в соответствии с результатами, полученными другими авторами.
% 
% 
% {\probation}
% Основные результаты работы докладывались~на:
% перечисление основных конференций, симпозиумов и~т.\:п.
% 
% {\contribution} Автор принимал активное участие \ldots
% 
% \ifnumequal{\value{bibliosel}}{0}
% {%%% Встроенная реализация с загрузкой файла через движок bibtex8. (При желании, внутри можно использовать обычные ссылки, наподобие `\cite{vakbib1,vakbib2}`).
%     {\publications} Основные результаты по теме диссертации изложены в XX печатных изданиях,
%     X из которых изданы в журналах, рекомендованных ВАК,
%     X "--- в тезисах докладов.
% }%
% {%%% Реализация пакетом biblatex через движок biber
%     \begin{refsection}[bl-author]
%         % Это refsection=1.
%         % Процитированные здесь работы:
%         %  * подсчитываются, для автоматического составления фразы "Основные результаты ..."
%         %  * попадают в авторскую библиографию, при usefootcite==0 и стиле `\insertbiblioauthor` или `\insertbiblioauthorgrouped`
%         %  * нумеруются там в зависимости от порядка команд `\printbibliography` в этом разделе.
%         %  * при использовании `\insertbiblioauthorgrouped`, порядок команд `\printbibliography` в нём должен быть тем же (см. biblio/biblatex.tex)
%         %
%         % Невидимый библиографический список для подсчёта количества публикаций:
%         \printbibliography[heading=nobibheading, section=1, env=countauthorvak,          keyword=biblioauthorvak]%
%         \printbibliography[heading=nobibheading, section=1, env=countauthorwos,          keyword=biblioauthorwos]%
%         \printbibliography[heading=nobibheading, section=1, env=countauthorscopus,       keyword=biblioauthorscopus]%
%         \printbibliography[heading=nobibheading, section=1, env=countauthorconf,         keyword=biblioauthorconf]%
%         \printbibliography[heading=nobibheading, section=1, env=countauthorother,        keyword=biblioauthorother]%
%         \printbibliography[heading=nobibheading, section=1, env=countauthor,             keyword=biblioauthor]%
%         \printbibliography[heading=nobibheading, section=1, env=countauthorvakscopuswos, filter=vakscopuswos]%
%         \printbibliography[heading=nobibheading, section=1, env=countauthorscopuswos,    filter=scopuswos]%
%         %
%         \nocite{*}%
%         %
%         {\publications} Основные результаты по теме диссертации изложены в~\arabic{citeauthor}~печатных изданиях,
%         \arabic{citeauthorvak} из которых изданы в журналах, рекомендованных ВАК\sloppy%
%         \ifnum \value{citeauthorscopuswos}>0%
%             , \arabic{citeauthorscopuswos} "--- в~периодических научных журналах, индексируемых Web of~Science и Scopus\sloppy%
%         \fi%
%         \ifnum \value{citeauthorconf}>0%
%             , \arabic{citeauthorconf} "--- в~тезисах докладов.
%         \else%
%             .
%         \fi
%     \end{refsection}%
%     \begin{refsection}[bl-author]
%         % Это refsection=2.
%         % Процитированные здесь работы:
%         %  * попадают в авторскую библиографию, при usefootcite==0 и стиле `\insertbiblioauthorimportant`.
%         %  * ни на что не влияют в противном случае
%         \nocite{vakbib2}%vak
%         \nocite{bib1}%other
%         \nocite{confbib1}%conf
%     \end{refsection}%
%         %
%         % Всё, что вне этих двух refsection, это refsection=0,
%         %  * для диссертации - это нормальные ссылки, попадающие в обычную библиографию
%         %  * для автореферата:
%         %     * при usefootcite==0, ссылка корректно сработает только для источника из `external.bib`. Для своих работ --- напечатает "[0]" (и даже Warning не вылезет).
%         %     * при usefootcite==1, ссылка сработает нормально. В авторской библиографии будут только процитированные в refsection=0 работы.
%         %
%         % Невидимый библиографический список для подсчёта количества внешних публикаций
%         % Используется, чтобы убрать приставку "А" у работ автора, если в автореферате нет
%         % цитирований внешних источников.
%         % Замедляет компиляцию
%     \ifsynopsis
%     \ifnumequal{\value{draft}}{0}{
%       \printbibliography[heading=nobibheading, section=0, env=countexternal,          keyword=biblioexternal]%
%     }{}
%     \fi
% }
% 
% При использовании пакета \verb!biblatex! будут подсчитаны все работы, добавленные
% в файл \verb!biblio/author.bib!. Для правильного подсчёта работ в~различных
% системах цитирования требуется использовать поля:
% \begin{itemize}
%         \item \texttt{authorvak} если публикация индексирована ВАК,
%         \item \texttt{authorscopus} если публикация индексирована Scopus,
%         \item \texttt{authorwos} если публикация индексирована Web of Science,
%         \item \texttt{authorconf} для докладов конференций,
%         \item \texttt{authorother} для других публикаций.
% \end{itemize}
% Для подсчёта используются счётчики:
% \begin{itemize}
%         \item \texttt{citeauthorvak} для работ, индексируемых ВАК,
%         \item \texttt{citeauthorscopus} для работ, индексируемых Scopus,
%         \item \texttt{citeauthorwos} для работ, индексируемых Web of Science,
%         \item \texttt{citeauthorvakscopuswos} для работ, индексируемых одной из трёх баз,
%         \item \texttt{citeauthorscopuswos} для работ, индексируемых Scopus или Web of~Science,
%         \item \texttt{citeauthorconf} для докладов на конференциях,
%         \item \texttt{citeauthorother} для остальных работ,
%         \item \texttt{citeauthor} для суммарного количества работ.
% \end{itemize}
% % Счётчик \texttt{citeexternal} используется для подсчёта процитированных публикаций.
% 
% Для добавления в список публикаций автора работ, которые не были процитированы в
% автореферате требуется их~перечислить с использованием команды \verb!\nocite! в
% \verb!Synopsis/content.tex!.
