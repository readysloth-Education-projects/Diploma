%%% Реализация библиографии пакетами biblatex и biblatex-gost с использованием движка biber %%%
\usepackage{csquotes} % biblatex рекомендует его подключать. Пакет для оформления сложных блоков цитирования.
%%% Загрузка пакета с основными настройками %%%
\makeatletter
\ifnumequal{\value{draft}}{0}{% Чистовик
\usepackage[%
backend=biber,% движок
bibencoding=utf8,% кодировка bib файла
sorting=none,% настройка сортировки списка литературы
style=gost-numeric,% стиль цитирования и библиографии (по ГОСТ)
language=autobib,% получение языка из babel/polyglossia, default: autobib % если ставить autocite или auto, то цитаты в тексте с указанием страницы, получат указание страницы на языке оригинала
autolang=other,% многоязычная библиография
clearlang=true,% внутренний сброс поля language, если он совпадает с языком из babel/polyglossia
defernumbers=true,% нумерация проставляется после двух компиляций, зато позволяет выцеплять библиографию по ключевым словам и нумеровать не из большего списка
sortcites=true,% сортировать номера затекстовых ссылок при цитировании (если в квадратных скобках несколько ссылок, то отображаться будут отсортированно, а не абы как)
doi=false,% Показывать или нет ссылки на DOI
isbn=false,% Показывать или нет ISBN, ISSN, ISRN
]{biblatex}[2016/09/17]
\ltx@iffilelater{biblatex-gost.def}{2017/05/03}%
{\toggletrue{bbx:gostbibliography}%
\renewcommand*{\revsdnamepunct}{\addcomma}}{}
}
{%Черновик
\usepackage[%
backend=biber,% движок
bibencoding=utf8,% кодировка bib файла
style=gost-numeric,% стиль цитирования и библиографии (по ГОСТ)
language=autobib,% получение языка из babel/polyglossia, default: autobib % если ставить autocite или auto, то цитаты в тексте с указанием страницы, получат указание страницы на языке оригинала
sorting=true,% настройка сортировки списка литературы
defernumbers=true, % откомментируйте, если требуется правильная нумерация ссылок на литературу в режиме черновика. Замедляет сборку
]{biblatex}[2016/09/17]%
}
\makeatother

\ifxetexorluatex
\else
% Исправление случая неподдержки знака номера в pdflatex
    \DefineBibliographyStrings{russian}{number={\textnumero}}
\fi

\ifsynopsis
\ifnumgreater{\value{usefootcite}}{0}{
    \ExecuteBibliographyOptions{autocite=footnote}
    \newbibmacro*{cite:full}{%
        \printtext[bibhypertarget]{%
            \usedriver{%
                \DeclareNameAlias{sortname}{default}%
            }{%
                \thefield{entrytype}%
            }%
        }%
        \usebibmacro{shorthandintro}%
    }
    \DeclareCiteCommand{\smartcite}[\mkbibfootnote]{%
        \usebibmacro{prenote}%
    }{%
        \usebibmacro{citeindex}%
        \usebibmacro{cite:full}%
    }{%
        \multicitedelim%
    }{%
        \usebibmacro{postnote}%
    }
}{}
\fi
